% ------------------ PACKAGES ------------------ 
% Packages add extra commands and features to your LaTeX document. 
% In here, some of the most common packages for a thesis document have been added 

% LaTeX's float package
\usepackage{float}
% LaTeX's color package
\usepackage{color}
\usepackage{epigraph}
% Add color to Tables
\usepackage[table,xcdraw]{xcolor}

% LaTeX's main math package
\usepackage{amsthm}
\newtheorem{thm}{Theorem}
\newtheorem{lemma}{Lemma}
\usepackage[ruled,vlined]{algorithm2e}
\usepackage{esvect}
\usepackage{tikz}
\usetikzlibrary{patterns}
\usetikzlibrary{bayesnet}
\usetikzlibrary{arrows}
\usetikzlibrary{backgrounds}
\usepackage{algorithmic}
\usepackage{hyperref}
\usepackage[capitalize,noabbrev]{cleveref}

\usepackage{multirow}

% LaTeX's Caption and subcaption packages
\usepackage[format=hang,font=normalsize,labelfont=bf,labelsep=colon,singlelinecheck=off, justification=centering]{caption}
\usepackage{subcaption}

% Longtable allows you to write tables that continue to the next page.
\usepackage{longtable}

% Font encoding
\usepackage[T1]{fontenc}

% This package allows the user to specify the input encoding
\usepackage[utf8]{inputenc}

% This package allows you to add empty pages
\usepackage{emptypage}

% Allows inputs to be imported from a directory
\usepackage{import}

% Provides control over the typography of the Table of Contents, List of Figures and List of Tables
\usepackage{tocloft}

% Allows the customization of Latex's title styles
\usepackage{titlesec}

% Allows the customization of Latex's table of contents title styles
\usepackage{titletoc}

% The package provides functions that offer alternative ways of implementing some LATEX kernel commands
\usepackage{etoolbox}

% Provides extensive facilities for constructing and controlling headers and footers
\usepackage{fancyhdr} 

% Typographical extensions, namely character protrusion, font expansion, adjustment 
%of interword spacing and additional kerning
\usepackage{microtype}

% Generates PDF bookmarks
\usepackage{bookmark}

% Use these two packages together -- they define symbols
%  for e.g. units that you can use in both text and math mode.
\usepackage{gensymb}
\usepackage{textcomp}

% Bibliography package
% \usepackage[backend=biber,style=nature]{biblatex}
\usepackage[natbib,style=authoryear]{biblatex}
\addbibresource{References.bib} % Add the .bib file that contains the references

% This package provides an easy way to input latin sample text (for the template only)
\usepackage{blindtext}
\usepackage{booktabs}
\usepackage{listings}

% for svg
\usepackage{graphicx}
\usepackage{svg}
% for restating theorem
\usepackage{thmtools}
\usepackage{thm-restate}

\usepackage{setspace}

\usepackage{minitoc} %for creating TOC at beginning of each chapter

\newcommand\addtotoc[1]{

  \refstepcounter{dummy}

  \addcontentsline{toc}{chapter}{#1}}

 