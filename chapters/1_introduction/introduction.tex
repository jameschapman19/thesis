\chapter{Introduction}
\label{Introduction}

\section{Motivation and Challenges of Self-Supervised Learning for Biomedical Data}

Biomedical data are essential for advancing our knowledge and practice of medicine and healthcare. Biomedical data can help us understand the mechanisms and causes of diseases, diagnose and monitor patients' conditions, develop and evaluate treatments, and discover new insights and opportunities for improving health outcomes. However, biomedical data are also challenging to analyze due to their complexity, heterogeneity, high-dimensionality, and scarcity of labels.

Biomedical data are complex because they capture various aspects of biological systems and processes at different levels, such as molecular, cellular, tissue, organ, and organism. Biomedical data are heterogeneous because they come from different sources and modalities, such as genomic sequences, gene expression profiles, proteomic spectra, metabolic pathways, brain images, electrocardiograms, clinical records, questionnaire responses, and wearable device measurements. Biomedical data are high-dimensional because they often contain thousands or millions of features or variables that describe each sample or individual. Biomedical data are scarce of labels because they often require expert annotation or validation that is costly, time-consuming, and error-prone.

To overcome these challenges, self-supervised learning (SSL) has emerged as a promising paradigm for learning from unlabeled data by leveraging inherent structures or patterns in the data. SSL methods can exploit different forms of supervision signals derived from the data itself, such as contrastive learning, reconstruction, prediction, or clustering. SSL methods can also benefit from deep neural networks that can learn expressive and flexible representations from complex and high-dimensional data. SSL methods have shown remarkable performance on various biomedical tasks, such as segmentation\cite{krishnan2022self}, classification\cite{serra2019multiview}, detection\cite{wang2023multitask}, and generation. The recent and phenomenal success of large language models (LLMs), such as BERT\cite{devlin2018bert} and GPT-3\cite{brown2020language}, which are based on SSL, has demonstrated the power of SSL for learning general and transferable representations from massive amounts of unlabeled text data. Similarly, we aim to leverage SSL for learning from large-scale unlabeled biomedical data.

A subset of SSL methods that are particularly relevant for biomedical data are multiview SSL methods. Multiview SSL methods are designed to handle multiview data, which consist of multiple sources of information that describe the same phenomenon or entity, such as brain imaging and gene expression for mental health. Multiview data offer several advantages for SSL methods: 

\begin{itemize} \item They can provide complementary or redundant information that can enhance the representation quality or robustness. \item They can enable cross-view learning that can transfer knowledge from one view to another. \item They can reveal latent structures or patterns that are shared or specific across views. \item They can uncover causal relationships or effects among views that can improve interpretability or discovery. \end{itemize}

However, existing multiview SSL methods also face several limitations when applied to biomedical data, such as scalability, flexibility, and interpretability.

\textbf{Scalability} refers to the ability of multiview SSL methods to handle large-scale and high-dimensional data efficiently and effectively. Many classical multiview SSL methods such as Principal Component Analysis (PCA), Canonical Correlation Analysis (CCA), and Partial Least Squares (PLS) are based on subspace learning techniques that require solving eigenvalue problems or matrix factorizations that are computationally expensive and memory-intensive. Moreover, many multiview SSL methods rely on batch optimization that cannot exploit the advantages of modern optimisation techniques like stochastic gradient descent (SGD) and engineering methods like parallel computing. In this thesis, we propose novel approaches to multiview SSL that are scalable and efficient by reformulating subspace learning as an unconstrained optimization problem that can be solved by SGD and by exploiting deep neural networks that can handle large-scale and high-dimensional data.

\textbf{Flexibility} refers to the ability of multiview SSL methods to adapt to different types of data with different structures and properties. Many classical multiview SSL methods are limited by linear models that cannot capture the nonlinear and complex relationships among different types of biomedical data. Moreover, many multiview SSL methods cannot incorporate different forms of regularization or prior knowledge that may be useful for improving the quality or robustness of the representations. In this thesis, we propose novel approaches to multiview SSL that are flexible and adaptable by using deep neural networks that can learn nonlinear associations from complex data and by developing a general framework for regularized multiview SSL that can handle different types of regularization forms on each view.

\textbf{Interpretability} refers to the ability of multiview SSL methods to provide meaningful and understandable explanations for the learned representations or associations. Many classical multiview SSL methods learn dense and entangled representations that are difficult to analyze or visualize. Moreover, many multiview SSL methods do not account for the covariance structures or sparsity patterns of the data that may reveal important features or factors. In this thesis, we propose novel approaches to multiview SSL that are interpretable and explainable by using causal inference techniques that can learn disentangled representations that can reveal causal relationships among views and by exploiting covariance structures or regularization techniques that can induce sparsity or disentanglement in the representations.

The core idea of this thesis is to formulate certain SSL problems as unconstrained objective functions that can be optimized by gradient descent. This idea enables us to extend the classical subspace learning methods, such as Canonical Correlation Analysis (CCA), to nonlinear functions, such as deep neural networks, and to incorporate structured priors, such as regularization, that can improve the interpretability and robustness of the learned representations. To facilitate the understanding of our work, we will provide a background on the related work in multiview learning and SSL in the next chapter. This thesis will focus on solving problems related to CCA, which is arguably the most versatile and powerful of the classical subspace learning methods.

% Chapter \ref{chap:literature} will give a general background to the literature around a number of classical subspace learning algorithms including Principal Components Analysis (PCA), Partial Least Squares (PLS), and Canonical Correlation Analysis (CCA). We will review prior work attempting to apply these methods to high-dimensional data including regularisation and kernel-based methods. Finally, we will review efforts to extend these methods to the deep learning setting in both Deep CCA and more recent work on self-supervised learning.

% Chapter \ref{chap:alternating least squares} will demonstrate the benefits of using flexible regularisation in the context of brain-behaviour data in order to motivate the later work. We adapt recent work from NLP. 

% The workhorse of this thesis is introduced in chapter \ref{chap:gradient descent} where we reformulate Generalized Eigenvalue Problems as unconstrained objectives which can be solved by gradient descent. Gradient descent allows two further developments: proximal gradient descent for implementing regularisation (explored in chapter \ref{chap:proximal gradient descent}, and extension to Deep CCA and related problems (chapter \ref{chap:deep).



