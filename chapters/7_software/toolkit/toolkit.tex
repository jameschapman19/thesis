\section{MATLAB: PLS/CCA Toolkit}

The PLS/CCA Toolkit is designed to offer a comprehensive framework for multivariate association analysis involving different data modalities such as brain imaging and behavior or brain imaging and genetics. It supports various types of PLS/CCA models including PLS, CCA, PCA-CCA, SPLS, KCCA, as well as several options for statistical inference and hyperparameter optimization.

\subsection{Contribution}

My main contribution to the PLS/CCA Toolkit lies in the development of a MATLAB function that generates simulated data as outlined in chapter \ref{chap:framework}. This function is available for user exploration and is also employed in some of the documentation examples.

\subsubsection{Function Overview}

The \texttt{generate\_data} function produces synthetic datasets based on a sparse latent variable model, following the model by Witten et al., 2009. It produces two sets of data modalities with specified dimensions, active features, and noise levels.

\begin{lstlisting}[language=Matlab]
function [X, Y, wX, wY] = generate_data(nexamples, nfeatx, nfeaty, activex, activey, noise)
\end{lstlisting}

\subsubsection{Parameters and Outputs}

The function accepts six parameters:

\begin{itemize}
    \item \texttt{nexamples}: Number of data samples
    \item \texttt{nfeatx} and \texttt{nfeaty}: Number of features in each data modality
    \item \texttt{activex} and \texttt{activey}: Number of active features correlated with a latent variable
    \item \texttt{noise}: Noise level in the generative model
\end{itemize}

It returns four outputs:

\begin{itemize}
    \item \texttt{X} and \texttt{Y}: Simulated datasets for two different modalities
    \item \texttt{wX} and \texttt{wY}: True weight vectors used in generating the data
\end{itemize}

\subsubsection{Utility and Applications}

The availability of this function enriches the PLS/CCA Toolkit by providing a valuable utility for:

\begin{itemize}
    \item Validating the performance of different PLS/CCA algorithms.
    \item Assisting users in understanding the behavior of the models in a controlled setting.
\end{itemize}

\subsubsection{Documentation and Example}

The function is well-documented and accompanied by example usages for clarity. For instance, one can use the function to generate a dataset with 1000 samples, 100 features in each modality, 10 active features, and a noise level of 1:

\begin{lstlisting}[language=Matlab]
[X, Y, wX, wY] = generate_data(1000, 100, 100, 10, 10, 1);
\end{lstlisting}
