\graphicspath{{chapters/gradient_descent/}}
\chapter{Gradient Descent: Accelerating CCA for Large Datasets}\label{chap:gradient_descent}
\epigraph{It seems easier to train a bi-directional LSTM with attention than to compute the SVD of a large matrix.\cite{gemp}}{Chris Ré}

The content of this chapter is based on a series of papers~\citep{chapman2022generalized, chapman2023efficient} as well as a NeurIPS workshop paper~\citep{chapman2023neurips}.
I am grateful to my co-authors Lennie Wells and Ana Lawry Aguila for their contributions to this work.
In particular, Lennie's mathematical expertise improved the theoretical grounding of the idea greatly and Ana's access to the UK Biobank dataset enabled the application of our methods to a real-world biomedical dataset.
In this thesis I include much of the work from these papers, but I exclude many of Lennie's extensive proofs where I can make no claim to have contributed beyond proofreading.
It was a joy to bring this work to fruition and I am proud of the results we achieved.
\minitoc

\section{Introduction} %Introduce the problem statement and its importance

CCA methods learn highly correlated representations of multi-view data.
The original CCA method of \citet{hotelling1933analysis} learns low-dimensional representations from linear transformations.
Notable extensions to ridge-regularized CCA \citep{vinod1976canonical}, Partial Least Squares (PLS), and multi-view CCA  \citep{wong2021deep} allow one to use CCA in high dimensional regimes, and with three or more views of data.
More recently, a variety of Deep CCA methods \citep{andrew2013deep} have been proposed which learn representations obtained from non-linear transformations of the data, parameterized by deep neural networks; Deep CCA has seen excellent empirical results, and is so foundational for deep multi-view learning that it secured a runner-up position for the test-of-time award at ICML 2023 \citep{ICML2023TOT}.

However, there are significant computational challenges when applying these CCA methods to large scale data.
Classical algorithms for linear CCA methods require computing full covariance matrices and so scale quadratically with dimension, becoming intractable for many large-scale datasets of practical interest.
There is therefore great interest in approximating solutions for CCA in stochastic or data-streaming settings \citep{arora2012stochastic}.
Large scale data also challenges existing full-batch algorithms for Deep CCA, and their stochastic counterparts are not only complex to implement but also difficult to train \citep{wang2015stochastic}.
% want to say something about downstream tasks, not sure how best to

Self-supervised learning (SSL) methods are now state of the art in image classification.
% need citation
They also learn useful representations of data, usually from pretext tasks or objectives that exploit some inherent structure or property of the data.
Remarkably, SSL methods can even perform zero-shot classification: where the representations are learnt without any explicit labels or supervision \citep{balestriero2023cookbook}.
Of particular interest to us is the so-called CCA family %\cite{balestriero2023cookbook}
of SSL methods.
Like CCA, these aim to transform a pair of views of data to a pair of similar representations.
It is known that a number of algorithms in this class are closely related to CCA \citep{balestriero2022contrastive}, notably including Barlow Twins \citep{zbontar2021barlow} and VICReg \citep{bardes2021vicreg}.
However, the specific details of this relationship are poorly understood.

In \cref{sec:background-unified} we provide a unified approach to all the CCA methods introduced above, emphasizing objectives which are functions of the joint distributions of the transformed variables.
% These methods all learn transformations optimizing some function of the joint distribution of the transformed variables.
Versions of all the linear CCA methods can be defined by solutions to certain Generalized Eigenvalue Problems (GEPs); this provides a particularly convenient way to relate a large number of equivalent objectives.

Section \ref{sec:contributions} outlines our core conceptual contributions.
Firstly, with \cref{prop:EY-charac} we present an unconstrained loss function that characterizes solutions to GEPs; this is based on the Eckhart--Young inequality and has appealing geometrical properties.
We apply this to the GEP formulation of CCA and construct unbiased estimates of the loss and its gradients from mini-batches of data.
These loss functions can therefore be optimized out-of-the-box using standard frameworks for deep learning. % and gradient descent-based optimisers.
This immediately gives a unified family of algorithms for CCA, Deep CCA, and indeed SSL.

Our CCA algorithms dominate existing state-of-the-art methods across a wide range of benchmarks, presented in \cref{Experiments}.
For stochastic CCA, our method not only converges faster but also achieves higher correlation scores than existing techniques.
For Deep CCA and Deep Multiview CCA our unbiased stochastic gradients yield significantly better validation correlations and allow the use of smaller mini-batches in memory constrained applications.
We also demonstrate how useful our algorithms are in practice with an pioneering real-world case study. We apply stochastic Partial Least Squares (PLS) to an extremely high-dimensional dataset from the UK Biobank dataset -- all executed on a standard laptop.
Our algorithm opens the door to problems previously deemed intractable in biomedical data analytics.
Finally, our SSL method achieves comparable performance to VICReg and Barlow Twins, despite having no hyperparameters in the objective.
This frees computational resources to tune more critical hyperparameters, such as architecture, optimizer or augmentations.
In addition, our method appears more robust to these other hyperparameters, has a clear theoretical foundation, and naturally generalizes to the multi-view setting.
Furthermore, we also provide the first rigorous proofs that Barlow Twins and VICReg are equivalent to CCA in the linear case, opening up avenues to better theoretical understanding.

\section{Background}
\subsection{A unified approach to the CCA family}\label{sec:background-unified}

Suppose we have a sequence of vector-valued random variables $X^{(i)} \in \R^{D_i}$ for $i \in \{1, \dots, I \}$\footnote{Yes, there are I (eye) views}.
We want to learn meaningful $K$-dimensional representations
\begin{equation}\label{eq:general-form-of-representations}
    Z\sps{i} = f\sps{i}( X\sps{i}; \theta\sps{i}).
\end{equation}
For convenience, define $D = \sum_{i=1}^I D_i$ and $\theta = \left(\theta\sps{i}\right)_{i=1}^I$.
Without loss of generality take $D_1 \geq D_2 \geq \cdots \geq D_I$.
We will consistently use the subscripts $i,j \in [I]$ for views;
$d \in [D_i]$ for dimensions of input variables;
and $l,k \in [K]$ for dimensions of representations - i.e. to subscript dimensions of $Z\sps{i}, f\sps{i}$.
Later on we will introduce total number of samples $N$ and mini-batch size $M$.
%so we will subscript samples with $n,m$ respectively.
% don't know if actually ended up using this, so can comment out if not!

\subsection{Background: GEPs in linear algebra}
A Generalized Eigenvalue Problem (GEP) is defined by two symmetric matrices $A,B\in \mathbb{R}^{D\times D}$ \citep{stewart_matrix_1990}\footnote{more generally, $A,B$ can be Hermitian, but we are only interested in the real case}.
They are usually characterized by the set of solutions to the equation:
\begin{align}\label{eq:igep}
Au=\lambda Bu
\end{align}
with $\lambda \in \R, u \in \R^D$, called (generalized) eigenvalue and (generalized) eigenvector respectively. We shall only consider the case where $B$ is positive definite to avoid degeneracy.
Then the GEP becomes equivalent to an eigendecomposition of the symmetric matrix $B^\mhalf A B^\mhalf$. This is key to the proof of our new characterization.
In addition, one can find a basis of eigenvectors spanning $\R^d$.
%the eigenvectors are not always unique, but the eigenvalues $\lambda_1 \geq \dots \geq \lambda_d$ are.
We define a \textit{top-$k$ subspace} to be one spanned by some set of eigenvectors {$u_1,\dots,u_K$} with the top-$K$ associated eigenvalues $\lambda_1 \geq \dots \geq \lambda_K$.
We say a matrix $U \in \R^{D \times K}$ \textit{defines} a top-$K$ subspace if its columns span one.
%hoping this finesses non-uniqueness for main text.
GEPs have a rich geometric structure, and lead to many alternative characterizations; for context we state a standard constrained characterization \citep{stewart_matrix_1990}.

\begin{restatable}[Constrained objective for GEPs]{proprep}{GEPconstrained}
\label{prop:GEP-constrained}
    The top-$k$ subspace of the GEP $(A,B)$ can be characterized by the following maximization \begin{equation}\label{eq:gep-constrained-subspace-formulation}
        \max_{U \in \R^{d\times k}} \tr(U^{\top} A U) \quad \text{ subject to: }  U^{\top} B U = I_K.
        \end{equation}
    Moreover, the maximal value of this objective is $\sum_{k=1}^K \lambda_k$.
\end{restatable}

\subsection{The CCA Family}\label{sec:CCA Definition}
The classical notion CCA \citep{hotelling1992relations} considers two views, $I=2$, and constrains the representations to be linear transformations
\begin{equation}\label{eq:cca-linear-function-def}
    Z\sps{i}_k = \langle u_k\sps{i} , X\sps{i} \rangle.
\end{equation}

The objective is to find the \textit{weights} or \textit{canonical directions} $u_k\sps{i}$ which maximize the \textit{canonical correlations} $\rho_k = \Corr(Z\sps{1}_k, Z\sps{2}_k)$
sequentially, subject to orthogonality with the previous pairs of the transformed variables.
It is well known that CCA is equivalent to a singular value decomposition (SVD) of the matrix $\Var(X\sps{1})^\mhalf \Cov(X\sps{1}, X\sps{2}) \Var(X\sps{2})^\mhalf$.
It is slightly less well known \citep{borga_learning_1998} that this is equivalent to a GEP where:
\begin{equation}\label{eq:cca-GEV}\small
	A = \begin{pmatrix} 0 &\Cov(X\sps{1}, X\sps{2}) \\ \Cov(X\sps{2}, X\sps{1}) & 0 \end{pmatrix}, \quad
	B = \begin{pmatrix}\Var(X\sps{1}) & 0 \\ 0 & \Var(X\sps{2}) \end{pmatrix}, \quad
	u =\begin{pmatrix}	u\sps{1} \\ u\sps{2} \end{pmatrix}.
\end{equation}

CCA therefore has notions of uniqueness similar to those for SVD or GEPs: the weights are not in general unique, but the canonical correlations $1 \geq \rho_1 \geq \rho_2 \geq \dots \geq 0$ are unique \citep{mills1988calculation}. Therefore, we can write:
%\footnote{Slightly abusing notation: $\CCA_K$ is a function of the joint distribution $\mathcal{L}_{X\sps{1}, \sps{2}}$ not of the random vectors themselves}:
\begin{align}\label{eq:cca-vector-of-correlations-def}\small
    \CCA_K(X^{(1)},X^{(2)}) \defeq (\rho_k)_{k=1}^K
    %= \sigma_{1, \dots, K} \left(\Var(X\sps{1})^\mhalf \Cov(X\sps{1}, X\sps{2}) \Var(X\sps{2})^\mhalf\right).
\end{align}

\textbf{Unified GEP formulation:}
this GEP formulation of MCCA can be presented in a unified framework generalizing CCA and ridge-regularized extensions. Indeed, we now take $A,B \in \R^{D \times D}$ to be block matrices $A = (A\sps{ij})_{i,j=1}^I, B_\alpha = (B_\alpha\sps{ij})_{i,j=1}^I$ where the diagonal blocks of $A$ are zero, the off-diagonal blocks of $B$ are zero, and the remaining blocks are defined by:
\begin{align}\label{eq:gep-most-general-formulation}%\small
    A^{(ij)} &= \Cov(X\sps{i}, X\sps{j}) \text{ for } i \neq j, \quad % \text{ for } i,j \in [I], ; \:\:
    B_\alpha^{(ii)} = \alpha_i I_{D_i} + (1-\alpha_i) \Var(X\sps{i})  %\text{ for } i \in [I]
\end{align}
Where $\alpha \in [0,1]^I$ is a vector of ridge penalty parameters: taking $\alpha_i = 0 \: \forall i$ recovers CCA and $\alpha = 1 \: \forall i$ recovers PLS.
We may omit the subscript $\alpha$ when $\alpha=0$ and we recover the `pure CCA' setting; in this case, following \cref{eq:cca-vector-of-correlations-def} we can define $\MCCA_K(X\sps{1},\dots,X\sps{I})$ to be the vector of the top-$K$ generalized eigenvalues.
% This can also be posed as a GEP where $A, B$ are block matrices analogous to those in \cref{eq:cca-GEV}\footnote{I.e. $A$ has zeros on the diagonal blocks and $\Cov(X\sps{i}, X\sps{j})$ on the off-diagonal blocks; $B$ has zeros off the diagonal and $\Var(X\sps{i}, X\sps{i})$ on the diagonal blocks.}


\textbf{Deep CCA:} was originally introduced in \cite{andrew2013deep}; this was extended to an GEP-based formulation of \textbf{Deep Multi-view CCA} (DMCCA) in \cite{somandepalli2019multimodal}. This can be defined using our $\MCCA$ notation as maximizing
\begin{align}\label{eq:DMCCA-def}
    \norm{\MCCA_K\left(Z\sps{1}, ... Z\sps{I}\right)}_2
    %=\norm{V(\theta)^\mhalf C(\theta) V(\theta)^\mhalf}_F
\end{align}
over parameters $\theta$ of neural networks defining the representations $Z\sps{i}=f\sps{i}(X\sps{i};\theta\sps{i})$ for $i\in[I]$.
% which of course generalizes Andrew version
% $\norm{\CCA_K\left(Z\sps{1}, Z\sps{2}\right)}_2 =
%     \norm{\Var(Z\sps{1})^\mhalf \Cov(Z\sps{1}, Z\sps{2}) \Var(Z\sps{2})^\mhalf}_F$

\subsection{Related Work}

To the best of our knowledge, the state-of-the-art in Stochastic PLS and CCA are the subspace Generalized Hebbian Algorithm (\textbf{SGHA}) of \citep{chen2019constrained} and \textbf{$\gamma$-EigenGame} from \citep{gemp20,gemp2021}.
Specifically, SGHA utilizes a Lagrange multiplier heuristic along with saddle-point analysis, albeit with limited convergence guarantees.
EigenGame focuses on top-k subspace learning but introduces an adaptive whitening matrix in the stochastic setting with an additional hyperparameter.

\section{Methods}

In this section, we introduce a new class of algorithms based on matrix analysis results which allow us to efficiently solve a number of interesting problems including but not limited to CCA, PLS, DCCA, and SSL.

\subsection{GEP-GHA}
First, we present GEP-GH.

\begin{restatable}[Eckhart-Young inspired objective for GEPs]{proprep}{GHAcharac}
    \label{prop:GHA-charac}
    The top-$k$ subspace of the GEP $(A,B)$ can be characterised by maximising the following objective over $W \in \R^{d \times k}$:
    \begin{align}
        \mathcal{U}^\text{GEP-GHA}(W) \defeq \tr \left( 2\, W^T A W - \left(W^T A W\right) \left(W^T B W\right) \right)
    \end{align}
    Moreover, the maximum value is precisely $\sum_{j=1}^k \lambda_j^2$, where $(\lambda_j)$ are the generalised eigenvalues.
\end{restatable}

\subsection{Proof of Proposition~\ref{prop:GHA-charac}}

\subsection{GEP-EY: an unconstrained objective for GEPs}

First, we present GEP-EY, a formulation of GEP problems by matrix analysis which permits efficient optimization by gradient descent.
This is summarised by proposition~\ref{prop:EY-charac}, a consequence of applying the Eckhart-Young-Minsky inequality \citep{stewart_matrix_1990} to the eigen-decomposition of $B^\mhalf A B^\mhalf$. Detailed statements and proofs can be found in supplement~\ref{supp:proofs}.

\begin{restatable}[Eckhart-Young inspired objective for GEPs]{proprep}{EYcharac}
    \label{prop:EY-charac}
    The top-$k$ subspace of the GEP $(A,B)$ can be characterised by maximising the following objective over $W \in \R^{d \times k}$:
    \begin{align}
        \mathcal{U}^\text{GEP-EY}(W) \defeq \tr \left( 2\, W^T A W - \left(W^T B W\right) \left(W^T B W\right) \right)
    \end{align}
    Moreover, the maximum value is precisely $\sum_{j=1}^k \lambda_j^2$, where $(\lambda_j)$ are the generalised eigenvalues.
\end{restatable}

In the following sections, we describe a number of algorithms for solving GEPs using proposition~\ref{prop:EY-charac}.
These will be denoted with \textbf{EY} and \textbf{GHA}.

\subsection{Stochastic Optimization}

We next show how GEP-EY and CCA-SVD can be efficiently optimized using stochastic methods, which makes them suitable for large-scale and online settings.
Suppose we have an unbiased estimate $\hat{A}$ for $A$ and a pair of independent and unbiased estimates $(\hat{B},\hat{B}')$ of the matrix $B$ (for example obtained from a minibatch); then one can estimate the objective in proposition~\ref{prop:constr-charac} by:
\begin{align}
    \hat{\mathcal{U}}^\text{GEP-EY}(W) \defeq \tr \left( 2\, W^T \hat{A} W - \left(W^T \hat{B} W\right) \left(W^T \hat{B}' W\right) \right)
\end{align}
which we can differentiate to give a gradient estimate:
\begin{align}\label{eq:GEP-EY-grad}
    \hat{\Delta}^{\text{GEP-EY}}(W)
    \defeq \nabla_W \hat{\mathcal{U}}^\text{GEP-EY}(W)
    % old = 4 \left\{ \hat{A} W - \hat{B} W \left(W^T \hat{B}' W \right) \right\}
    = 2 \left\{ 2 \hat{A} W - \hat{B} W \left(W^T \hat{B}' W \right) - \hat{B}' W \left(W^T \hat{B} W \right) \right\}
\end{align}

By the independence of $(\hat{B},\hat{B}')$ both of these estimates are unbiased.

The unbiased estimate immediately leads to Algorithm~\ref{alg:Delta}.
The key advantage of this is that the computational complexity of the algorithm is only $\mathcal{O}(b d k)$; this is far less than the $\mathcal{O}(N d^2)$ complexity of any methods which evaluate the matrices $A,B$ on the full batch.

\textit{Why is the complexity so low?} Because $A,B$ are linear functions of covariance matrices, we can construct our unbiased estimates by plugging in sample covariances on a minibatch.
These estimates will then be low rank; indeed we can factorise these estimates in the form $\hat{M} \hat{M}^T$ where $\hat{M} \in \R^{d \times b}$. The dominant cost in evaluating $\mathcal{U}^\text{GEP-EY}$ is then just the matrix multiplications of the form $\hat{M}^T W$; see supplement~\ref{supp:analyticsubspace} for full details.

\begin{algorithm}
    \caption{GEP-EY: A Stochastic Gradient Descent Algorithm for GEP subspace}
    \label{alg:Delta}
    \begin{algorithmic}
        \STATE {\bfseries Input:} data stream $Z_t$ consisting of B samples from $z_n$. Learning rate $(\eta_t)_t$. Number of time steps $T$. Number of eigenvectors to compute $k$.
        \STATE {\bfseries Initialise:} $(\hat{\mathbf {W}})_{i=1}^K$ with random uniform entries
        \FOR{$t=1$ {\bfseries to} $T$}
        \STATE Construct an independent unbiased estimate of $\hat{A}$ and two independent unbiased estimates of $\hat{B}$ from $Z_t$
        \STATE $\hat{\mathbf {W}} \leftarrow \hat{\mathbf {W}}+\eta_{t} \hat{\Delta}^{\text{GEP-EY}}$
        \COMMENT{As defined in (\ref{eq:GEP-EY-grad})}
        \ENDFOR
    \end{algorithmic}
\end{algorithm}


\section{Conclusion}

We presented a new unconstrained objective to characterise GEPs; this immediately gave efficient algorithms to solve many subspace learning methods in the stochastic setting with SGD.
Crucially, the gradient estimates are unbiased and cheap to compute.
Moreover, the only hyperparameter is the choice of optimiser.
We will later show how this work can be adapted and applied to optimise regularised GEPs and, in particular, CCA.

\appendix

\section{Appendix}