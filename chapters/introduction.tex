\graphicspath{{chapters/introduction/}}


\chapter{Introduction}\label{chap:introduction}

In the middle of my PhD journey, in June 2021, I self-referred to the Community Living Well service in London, UK, for help with my mental health.
I was assigned a therapist, who I met with weekly for 12 weeks.
During our sessions, we discussed my mental health and the challenges I was facing.
I was also asked to complete a questionnaire at the beginning and end of each session, which asked me to rate my mood and answer questions about my mental health.
Each time I did this, I questioned how well these subjective numbers truly represented my feelings.

A keen sportsperson, I also wear a Garmin watch that tracks my heart rate, my sleep, and my activity levels.
I use this data to monitor my health and fitness, and I have found it to be a useful tool in my training.
Using a physical `stress level' metric based on Heart Rate Variability (HRV), I can see how alcohol affects my sleep\footnote{badly}, how well I have slept, and I know I am about to get sick before I feel it.

Furthermore, as a type 1 diabetic, I rely on a continuous glucose monitor.
This tool provides real-time blood sugar readings every five minutes, offering insights into trends and helping me fine-tune my insulin management.

These personal experiences underscore a broader issue in health data analysis: the challenge of integrating diverse health metrics, from subjective self-assessments to objective biometric readings, in a meaningful and interpretable way.
This thesis focuses on methods to resolve this challenge using self-supervised learning.
By applying these techniques to Brain-Behavior associations, I aim to demonstrate how integrating various health data streams can improve personal health management and understanding.

\section{Thesis Structure and Contributions}

This thesis presents new methods that can scale multiview learning to large datasets, revolutionizing the analysis and comprehension of biomedical data.
Using advancements in self-supervised and multiview learning, I explore the integration of diverse data sources, as exemplified by my mental health, physical activity, and diabetes management data.

A key goal is to develop practical, user-friendly methodological improvements.
We focus on creating tools and methods that are not only theoretically sound but also intuitive for use in real-world scenarios.
This will enable practitioners in biomedical research and other fields to focus on their domain expertise, rather than the technical details of the methods.

This thesis contributes in four ways:

\begin{itemize}
    \item Developing a framework for regularised Canonical Correlation Analysis (CCA) using structured priors, like the Elastic Net, enhancing interpretability.
    \item Unifying proposed simulated data generation methods for CCA from the literature, demonstrating that they can all be viewed as latent variable models, and improving our ability to interpret CCA results.
    \item Formulating a new gradient descent approach for CCA and other fundamental generalised eigenvalue problems, tailored for large datasets.
    \item Extending the gradient descent approach to Deep CCA and Joint Embedding Self-Supervised Learning.
\end{itemize}

These contributions offer practical benefits.
For instance, our regularised CCA framework allows practitioners to more accurately correlate brain imaging data with behavioral assessments and interpret the (possibly sparse) model parameters, aiding in the diagnosis and treatment of neurological disorders.
Our perspective on simulated data generation methods for CCA will help researchers better understand the relationship between loadings and weights in CCA, and simplify the process of generating high-dimensional simulated data for CCA.
The gradient descent approach for large datasets enables researchers to analyze extensive health databases, like the UK Biobank, more efficiently, leading to faster and more accurate health insights.
Finally, the Deep CCA and Self-Supervised Learning extensions will allow researchers to integrate diverse data sources, such as images and text, using modern deep learning techniques.

\subsection{Chapter Summaries}

\textbf{Chapter \ref{ch:background}} reviews multiview and self-supervised learning techniques, focusing on their application in biomedical data.

\textbf{Chapter \ref{ch:als}} introduces a method to regularize CCA using structured priors, demonstrated with Human Connectome Project and Alzheimer's Disease Neuroimaging Initiative data.

\textbf{Chapter \ref{ch:loadings}} examines the relationship between loadings and weights in CCA, using simulated data to show the advantages of loadings for interpretability.

\textbf{Chapter \ref{ch:gradient_descent}} presents a new gradient descent algorithm for generalized eigenvalue problems, demonstrated with Multiview CCA and PLS. We show how our algorithm can be applied to large datasets, using the UK Biobank as an example.

\textbf{Chapter \ref{ch:deep_learning}} extends the algorithm from Chapter \ref{ch:gradient_descent} to deep learning, showing its application in scaling deep CCA. We demonstrate state-of-the-art results on CIFAR-10 and CIFAR-100 benchmarks, illustrating the potential of Deep CCA in Self-Supervised Learning.

\textbf{Chapter \ref{ch:software}} introduces CCA-Zoo, a Python package implementing the methodologies of this thesis, and discusses its role in the Python ecosystem and biomedical research.

\textbf{Chapter \ref{ch:discussion}} discusses the implications, challenges, and future directions for the research presented in this thesis.

Through this thesis, I aspire to bridge the gap between the potential of biomedical data and the current capabilities of analytical methods, enhancing our ability to understand and manage personal health.
