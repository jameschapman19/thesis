\graphicspath{{chapters/background/}}


\chapter{Background: Multiview Machine Learning: Concepts, Methods, and Limitations}\label{chap:background}
% \epigraph{Principal Component Analysis is a dimensionally invalid method that gives people a delusion that they are doing something useful with their data. If you change the units that one of the variables is measured in, it will change all the “principal components”! It’s for that reason that I made no mention of PCA in my book. I am not a slavish conformist, regurgitating whatever other people think should be taught. I think before I teach.}{Professor David MacKay}
\minitoc


\section{Introduction to Machine Learning and Multiview Learning}

In this chapter, we gather the necessary background knowledge needed to motivate and understand the contributions of this thesis.

Machine learning enables models to automatically learn patterns and make decisions from data.
Machine learning comprises three primary paradigms: supervised, self-supervised (or unsupervised), and reinforcement learning, each distinct in its approach to learning from data.
This thesis focuses on \textit{multiview self-supervised machine learning}, which aims to develop robust representations by uncovering associations between various data types within datasets.
These data types, known as \gls{views} may include distinct sources of information such as neuroimaging modalities, genomics, and clinical records in the context of patient data analysis.

\subsection{Multiview Machine Learning}

Multiview machine learning encompasses a variety of techniques aimed at learning from data that have multiple sources or modalities, also known as \gls{views}.
These techniques can be broadly classified into supervised and self-supervised (or sometimes, equivalently, unsupervised) multiview learning, with some algorithms straddling the boundary between the two.

\subsubsection{Supervised Multiview Learning}

In supervised multiview learning, one view serves as the input while the other view is treated as the target label.
The algorithm learns to predict the target view based on the input view, leveraging the information from both to enhance the predictive performance \citep{zong2023self}. These can in principal encompass both regression and classification tasks depending on the nature of the target view (continuous or discrete). In a sense, we can think of this in exactly the same way as supervised learning, where the features are one view and the target is the other view. However, multiview learning is not restricted to a single view as input or target.

\subsubsection{Self-Supervised Multiview Learning}

Self-Supervised Learning (SSL) is a paradigm where the training signal is derived from the data itself, rather than relying on external labels \citep{balestriero2023cookbook}.
The cornerstone of SSL is the concept of a `pretext task', a learning task created from the data that trains the model to capture useful features or representations.
In the context of multiview machine learning, self-supervised learning often operates under the assumption that different \gls{views} are generated from a common, but unobserved, \textit{latent} source.
A natural pretext task, in this case, is to predict or estimate this source from the given \gls{views}.
Dimensionality reduction is a common example of this, where the model learns to estimate a low-dimensional representation of the data from the \gls{views}.
In this case, the model is forced to learn the underlying latent variable structure of the data without any direct supervision.
This not only enables the model to learn associations between \gls{views} but also allows it to derive robust and informative representations for subsequent tasks like classification or regression.

\subsection{Types of Multiview Data}

In neuroscience and genetics, two specific types of multiview studies are particularly relevant to this thesis: brain-behavior studies and imaging-genetics. Both involve the integration of data from multiple sources, offering rich insights into complex phenomena.

Brain-behavior studies typically involve pairing neuroimaging data, such as that obtained from Structural MRI (sMRI) or Functional MRI (fMRI), with non-imaging data like responses from questionnaires, cognitive test results, and other behavioral assessments. sMRI provides detailed anatomical brain images, essential for understanding brain structure and neurological disorders *citation*, while fMRI focuses on brain function by mapping activity during cognitive tasks *citation*. The integration of these imaging techniques with behavioral data offers a comprehensive view of how brain structures and functions correlate with behavioral and cognitive patterns *citation*.

Imaging-Genetics, another critical multiview approach, combines neuroimaging data with genetic information. This interdisciplinary field seeks to understand the genetic influences on brain structure and function, thereby illuminating the genetic basis of neuropsychiatric disorders and cognitive traits *citation*. Studies in this area might explore how specific genetic variations correlate with differences in brain morphology or activity patterns observed in neuroimaging *citation*.

Together, these multiview approaches are fundamental in advancing our understanding of the brain's structure, function, and its interactions with genetic and behavioral factors. They represent key applications of SSL in neuroscience and genetics, providing comprehensive insights that underpin developments in these fields *citation*.

\subsection{Conditional Independence, Causality, and Multiview Learning}

Consider the graphical model depicted in Figure~\ref{fig:mentalhealthselfsupervised}.
It comprises two distinct observed \gls{views}: a brain modality and a behavioral modality.
The graphical represents the assumption that the brain and behaviour are conditionally independent given the severity of an unobserved `latent' mental health condition.

\begin{figure}
    \centering
    \tikz{
    % nodes
        \node[latent, align=center, minimum size=2cm] (Z) {Severity\\z};
        %
        \node[obs, below left=of Z, minimum size=2cm, align=center] (x1) {Brain\\$x^{(1)}$};
        \node[obs, below right=of Z, minimum size=2cm, align=center] (x2) {Behaviour\\$x^{(2)}$};
        % edges
        \edge{Z} {x1}
        \edge{Z} {x2}}
    \caption[Latent Variable Model of Mental Health]{\textit{\textbf{Latent Variable Model of Mental Health:}} From this perspective the neuroimaging modality and behavioural data are both considered to have been generated with distributions conditioned on the severity of a mental health condition}\label{fig:mentalhealthselfsupervised}
\end{figure}

In multiview machine learning, the relationship between conditional independence and causality is nuanced but crucial.
When examining dependencies between events, such as those observed between brain activity and behavior, several scenarios emerge:

\begin{itemize}
    \item direct causation (brain influencing behavior or vice versa or even both)
    \item both being influenced by a common, possibly unobserved, cause
    \item no direct causal link between them
\end{itemize}

Importantly, if a common cause does exist, conditioning on it renders brain and behavior independent; this 'screens off' their dependence, revealing key insights for our models \citep{reichenbach1956direction}.
However, it is essential to recognize that the presence of a common latent variable, inferred from these \gls{views}, does not automatically imply causality in the observed data.

\subsubsection{Complementary and Redundant Information}
The nature of the information provided by different \gls{views} (such as neuroimaging and behavioral data) is important for understanding multiview learning models.
A particularly useful distinction is between \textit{complementary} and \textit{redundant} information \citep{nguyen2020multiview}.
When views contain complementary information, they provide different perspectives on the same subject or sample.
For example, we can understand different aspects of a mental health condition by examining both neuroimaging and behavioral data.
On the other hand, when \gls{views} contain redundant information about the latent variables, they provide the same information from different perspectives.
For example, a disease diagnosis might be encoded in both neuroimaging and blood test results.
This does not make the \gls{views} useless, however, because they can be used to denoise each other, enhancing the clarity and reliability of the data. We can be more confident that a diagnosis is correct if it is supported by both neuroimaging and blood test results.
A particularly famous example of this principle is the `Wisdom of Crowds' effect, where the average of multiple noisy estimates is more accurate than any individual estimate \citep{galton1907vox}.
This process exploits the overlap in information to correct or reduce noise and errors, a principle fundamental to many denoising techniques in machine learning.

In this thesis we will work with Canonical Correlation Analysis, a multiview learning method which assumes that the \gls{views} contain complementary information about latent variables.
The next section builds a formal understanding of the principles behind Canonical Correlation Analysis and its variants.

\section{Learning Representations: Definitions and Notation}

Suppose we have a sequence of vector-valued random variables $X\sps{i} \in \R^{D_i}$ for $i \in \{1, \dots, I \}$
We want to learn meaningful $K$-dimensional representations
\begin{equation}
    \label{eq:general-form-of-representations}
    Z\sps{i} = f\sps{i}( X\sps{i}; \theta\sps{i}).
\end{equation}
For convenience, define $D = \sum_{i=1}^I D_i$ and $\theta = \left(\theta\sps{i}\right)_{i=1}^I$.
Without loss of generality take $D_1 \geq D_2 \geq \cdots \geq D_I$.
We will consistently use the subscripts $i,j \in [I]$ for \gls{views};
$d \in [D_i]$ for dimensions of input variables;
and $l,k \in [K]$ for dimensions of representations - i.e. to subscript dimensions of $Z\sps{i}, f\sps{i}$.
Later on we will introduce total number of samples $N$.

In this report, when the functions $f$ are linear, we will typically refer to $u_k$ as \textit{\gls{weights}}, $Z_k = X_k u_k$ as \textit{\gls{representations}} or \textit{\gls{latent variables}}, depending on the context. We will sometimes consider a
matrix $U = \left(u_1, \dots, u_K\right) \in \R^{D \times K}$ of \gls{weights}, and a
matrix $Z = \left(Z_1, \dots, Z_K\right) \in \R^{N \times K}$ of representations.
We will refer to the Pearson correlation between features and their respective latent variable $\Corr(X\sps{i}_j, Z_k)$ as the \textit{\gls{loadings}} of $X\sps{i}_j$ on $Z_k$ \citep{rosipal2005overview, alpert1972interpretation}, noting that the same concept has also been referred to as \textit{structure correlations} \citep{meredith1964canonical}.

\subsection{Background: Generalized Eigenvalue Problems in linear algebra}
A Generalized Eigenvalue Problem (GEP) is defined by two symmetric matrices $A,B\in \mathbb{R}^{D\times D}$ \citep{stewart_matrix_1990}\footnote{more generally, $A,B$ can be Hermitian, but we are only interested in the real case}. They are usually characterized by the set of solutions to the equation:
\begin{align}
    \label{eq:igep}
    Au=\lambda Bu
\end{align}
with $\lambda \in \R, u \in \R^D$, called (generalized) eigenvalue and (generalized) eigenvector respectively. We shall only consider the case where $B$ is positive definite to avoid degeneracy.
Then the GEP becomes equivalent to an eigen-decomposition of the symmetric matrix $B^\mhalf A B^\mhalf$. This is key to the proof of our new characterization.
In addition, one can find a basis of eigenvectors spanning $\R^D$.
We define a \textit{top-$K$ subspace} to be one spanned by some set of eigenvectors {$u_1,\dots,u_K$} with the top-$K$ associated eigenvalues $\lambda_1 \geq \dots \geq \lambda_K$.
We say a matrix $U \in \R^{D \times K}$ \textit{defines} a top-$K$ subspace if its columns span one.

\paragraph{Uniqueness}
In GEPs, the eigenvectors $u$ are not in general unique, but the canonical correlations $1 \geq \rho_1 \geq \rho_2 \geq \dots \geq 0$ are unique \citep{mills1988calculation}.

\subsection{Principal Components Analysis}

Principal Components Analysis \citep{hotelling1933analysis} (\acrshort{pca}) is a classical method in unsupervised machine learning for representation learning.
It is widely used for dimensionality reduction and feature extraction.
The primary goal of \acrshort{pca} is to transform the original high-dimensional data into a new coordinate system defined by orthogonal axes, capturing the most relevant aspects of the data.

In \acrshort{pca}, the representations are constrained to be linear transformations of the form:
\begin{equation}
    \label{eq:pca-linear-function-def}
    Z_k = X u_k,
\end{equation}
where $u_k$ are the orthonormal basis vectors such that:
\begin{equation}
    \label{eq:pca-orthonormality-constraint}
    u_k^\top u_k = 1, \quad
    u_k^\top u_l = \delta_{kl} \text{ for } k \neq l.
\end{equation}

The primary goal of \acrshort{pca} is to maximize the variance of the representations \(Z_k\).

\subsubsection{Optimization and Solution}
Mathematically, for the first principal component, this can be formulated as:

\begin{align}
    u_{\text{opt}} & = \underset{u}{\text{argmax}} \left( u^\top \Sigma u \right) \\
    \text{subject to:} \notag                                                     \\
    u^\top u       & = 1 \notag
\end{align}

Where \(\Sigma = \mathbb{E}[X^\top X]\) is the covariance matrix of the data.

The Lagrangian for this problem is:
\begin{equation}
    f(u,\lambda) = u^\top \Sigma u + \lambda(1 - u^\top u),
\end{equation}
where \(\lambda\) is the Lagrange multiplier. Differentiating the Lagrangian yields the first-order conditions:
\begin{align}
    \Sigma u & = \lambda u, \\
    u^\top u & = 1.
\end{align}

\paragraph{Eigenvalue Problem}

This transforms the problem into an eigenvalue equation for the covariance matrix \(\Sigma\), which can be efficiently solved using standard libraries such as scikit-learn \citep{pedregosa2011scikit}.

The first principal component therefore corresponds to the eigenvector associated with the largest eigenvalue \(\lambda\). Subsequent components are the remaining eigenvectors ordered by their corresponding eigenvalues.

\subsubsection{Limitations}
However, when applying \acrshort{pca} to datasets such as high-dimensional neuroimaging and behavioral
data, \acrshort{pca}'s main limitation arises: it only accounts for variance within a single dataset, so it cannot take advantage of either the redundancy or the complementary information in multiview data.

\subsection{Partial Least Squares}

Partial Least Squares (PLS) \citep{wold1975path} aims to maximize the shared covariance between two paired sets of data, referred to as \gls{views}. \acrshort{pls} can be seen as a generalization of \acrshort{pca}, where \acrshort{pca} becomes a special case when the two \gls{views} are identical.

\subsubsection{Optimization and Solution}

The optimization problem for \acrshort{pls} can be formulated as:

\begin{align}
    u\sps{1}_{\text{opt}} & = \underset{u\sps{1}}{\mathrm{argmax}} \{ u\spsT{1} \Sigma_{12} u\sps{2} \} \\
    \text{subject to:} \notag                                                                             \\
    u\spsT{1}u\sps{1}   & = 1 \notag                                                                    \\
    u\spsT{2}u\sps{2}   & = 1 \notag
\end{align}

where \( X\sps{1} \in \mathbb{R}^{n \times p_1} \) and \( X\sps{2} \in \mathbb{R}^{n \times p_2} \), meaning we have two \gls{views} with the same number of samples but potentially different number of features.

The Lagrangian for this optimization problem can be formulated as:

\begin{equation}
    f(u\sps{1}, \lambda) = u\spsT{1} \Sigma_{12} u\sps{2} + \lambda_1 (1 - u\spsT{1}u\sps{1}) + \lambda_2 (1 - u\spsT{2}u\sps{2})
\end{equation}

Upon deriving the first order conditions, we get:

\begin{align}
    \Sigma_{21} u\sps{1} & = \lambda_2 u\sps{2} \\
    \Sigma_{12} u\sps{2} & = \lambda_1 u\sps{1} \\
    u\spsT{1}u\sps{1}  & = 1                  \\
    u\spsT{2}u\sps{2}  & = 1
\end{align}

By substituting the constraint conditions into these equations, we find that \( \lambda_1 = \lambda_2 = \lambda \) by symmetry. Further simplification yields:

\begin{align}
    \Sigma_{21} \Sigma_{12} u\sps{2} & = \lambda^2 u\sps{2} \\
    \Sigma_{12} \Sigma_{21} u\sps{1} & = \lambda^2 u\sps{1}
\end{align}

\paragraph{Eigenvalue Problem}

Once again, we see that solving these equations will yield the \( u\sps{1} \) and \( u\sps{2} \) vectors as eigenvectors, this time of \( \Sigma_{12} \Sigma_{21} \) and \( \Sigma_{21} \Sigma_{12} \), respectively \citep{hoskuldsson1988pls}.

\paragraph{Generalized Eigenvalue Problem}

We can also represent the system of equations in matrix form as follows:

\begin{align}
    \begin{pmatrix}
        0           & \Sigma_{12} \\
        \Sigma_{21} & 0
    \end{pmatrix}
    \begin{pmatrix}
        u\sps{1} \\
        u\sps{2}
    \end{pmatrix}
    =
    \lambda
    I
    \begin{pmatrix}
        u\sps{1} \\
        u\sps{2}
    \end{pmatrix}
\end{align}

Which is of the form $A v = \lambda B v$. \acrshort{pls} is therefore also defined by the solution to a single generalized eigenvalue problem.

Given the notions of uniqueness in GEPs, the weights $u$ are not in general unique but we can write the vector of generalized eigenvalues, here representing covariances, as
\begin{align}
    \label{eq:cca-vector-of-correlations-def}
    \PLS_K(X^{(1)},X^{(2)}) \defeq (\rho_k)_{k=1}^K
\end{align}

\subsubsection{Limitations} The problem with applying \acrshort{pls} to neuroimaging and behavioural modalities is that \acrshort{pls} is not scale invariant and
is therefore biased towards the largest principal components in the data \citep{helmer2020stability}.
This is particularly problematic when there is a low signal to noise ratio since \acrshort{pls} may find directions in either dataset which correspond to the largest directions of noise in the other.
Additionally, \acrshort{pls} assumes that the structures contributing to variance in both datasets are linearly related, which
may not be the case in complex biological systems like the brain or in intricate behavioral patterns \citep{rosipal2005overview}.
The linearity assumption can sometimes be overly restrictive, failing to capture more complicated, nonlinear relationships between the data modalities.
Another issue is the lack of sparsity in the \acrshort{pls} solution.
Traditional \acrshort{pls} methods do not provide sparse weight vectors, which makes the interpretation of results challenging in high-dimensional settings such as neuroimaging where only a subset of features might be relevant.
There are sparse variants of \acrshort{pls} available, but these typically introduce additional complexity and may require fine-tuning of regularization parameters \citep{chun2010sparse, witten2009penalized}.
Furthermore, \acrshort{pls} can be sensitive to outliers, which are not uncommon in neuroimaging data due to motion artifacts or other sources of noise.
Since the method aims to maximize covariance, extreme values in one dataset can disproportionately affect the resulting latent variables \citep{Wold1973}.

\subsection{Canonical Correlation Analysis}\label{sec:cca}

In Canonical Correlation Analysis (\acrshort{cca}), we aim to find the directions that maximize correlation, as opposed to maximizing covariance between two \gls{views} of a dataset.
This nuance renders \acrshort{cca} invariant to feature scale.

\subsubsection{Optimization and Solution}
The optimization problem for \acrshort{cca} can be expressed as:

\begin{align}
    & u_{\text{opt}}=\underset{u}{\mathrm{argmax}}\{ u\spsT{1}X\spsT{1}X\sps{2}u\sps{2} \} \\
    & \text{subject to:} \notag                                                                \\
    & u\spsT{1}\Sigma_{11}u\sps{1}=1 \notag                                                  \\
    & u\spsT{2}\Sigma_{22}u\sps{2}=1 \notag
\end{align}

Although non-convex, numerous methods exist for solving the \acrshort{cca} problem, including eigendecomposition and generalized eigendecomposition solvers \citep{uurtio2017tutorial} and block coordinate descent via alternating least squares regressions \citep{golub1995canonical,sun2008least}.

The first-order conditions derived in the same manner as the \acrshort{pls} case are:

\begin{align}
    \label{CCA:FOCs}
    & \Sigma_{21}u\sps{1}=\lambda\sps{2} \Sigma_{22}u\sps{2} \\
    & \Sigma_{12}u\sps{2}=\lambda\sps{1} \Sigma_{11}u\sps{1} \\
    & u\spsT{1}\Sigma_{11}u\sps{1}=1                       \\
    & u\spsT{2}\Sigma_{22}u\sps{2}=1
\end{align}

\paragraph{Eigenvalue Problems}

Substituting the second two conditions into the first two, we get \(\lambda\sps{1}=\lambda\sps{2}=\lambda\). Then, recognizing \(X_i^{\top}X_i\) as the covariance matrix \(\Sigma_{ii}\) and \(X_i^{\top}X_j\) as the cross-covariance matrix \(\Sigma_{ij}\), we obtain another pair of eigenvalue problems:

\begin{align}
    & \Sigma_{11}^{-1}\Sigma_{12}\Sigma_{22}^{-1}\Sigma_{21}u\sps{1}=\lambda^2u\sps{1} \notag \\
    & \Sigma_{22}^{-1}\Sigma_{21}\Sigma_{11}^{-1}\Sigma_{12}u\sps{2}=\lambda^2u\sps{2} \notag
\end{align}

An alternative form of the \acrshort{cca} problem can be developed by reparameterizing \(u\sps{i*}=\Sigma_{ii}^{-\frac{1}{2}}u\sps{i}\). The optimization problem then becomes:

\begin{align}
    & u_{\text{opt}}=\underset{u}{\mathrm{argmax}}\{ u\spsT{1}\Sigma_{11}^{-\frac{1}{2}}\Sigma_{12}\Sigma_{22}^{-\frac{1}{2}}u\sps{2} \} \\
    & \text{subject to:} \notag                                                                                                            \\
    & u\spsT{1}u\sps{1}=1 \notag                                                                                                         \\
    & u\spsT{2}u\sps{2}=1 \notag
\end{align}

This reparameterized form will later underpin Deep Canonical Correlation Analysis (\acrshort{dcca}).

This form also shows that \acrshort{pls} and \acrshort{cca} can be made equivalent by whitening the data matrices before constructing the covariance matrix. When the number of features exceeds the number of samples (\(p>n\)), \acrshort{cca} becomes degenerate because the within-view covariance matrices cannot be inverted—contrasting with \acrshort{pls}, which is always computable.

\paragraph{Generalized Eigenvalue Problem}

We can also represent the system of equations in equation~\ref{CCA:FOCs} as a matrix equation:

\begin{align}
    \begin{pmatrix}
        0           & \Sigma_{12} \\
        \Sigma_{21} & 0
    \end{pmatrix}
    \begin{pmatrix}
        u\sps{1} \\
        u\sps{2}
    \end{pmatrix}
    =
    \lambda
    \begin{pmatrix}
        \Sigma_{11} & 0           \\
        0           & \Sigma_{22}
    \end{pmatrix}
    \begin{pmatrix}
        u\sps{1} \\
        u\sps{2}
    \end{pmatrix}
\end{align}

Which is once again of the form $A v = \lambda B v$. \acrshort{cca}, like \acrshort{pls}, is therefore also defined by the solution to a single generalized eigenvalue problem.

Given the notions of uniqueness in GEPs, the weights $u$ are not in general unique but we can write the vector of generalized eigenvalues, here representing covariances, as
\begin{align}
    \label{eq:cca-vector-of-correlations-def}\small
    \CCA_K(X^{(1)},X^{(2)}) \defeq (\rho_k)_{k=1}^K
\end{align}

\subsection{Multiview \acrshort{cca}}

Multiview \acrshort{cca} or \acrshort{mcca} is a straightforward extension of \acrshort{cca} to the case of 3-or more datasets.
The goal is to find a set of directions \(u\sps{i}\) such that the pairwise correlations between the views are maximized.

\subsubsection{Optimization and Solution}

The optimization problem for \acrshort{mcca} can be stated as:
\begin{align}
    & u_{\text{opt}} = \underset{u}{\mathrm{argmax}} \sum_{i=1}^{m} \sum_{j=1, j \neq i}^{m} u\spsT{i} \Sigma_{ij} u\sps{j} \\
    & \text{subject to:} \notag                                                                                               \\
    & \sum_{i=1}^{m} u\spsT{i} \Sigma_{ii} u\sps{i} = 1 \notag
\end{align}

\paragraph{Generalized Eigenvalue Problem}

The generalized eigenvalue problem (GEP) for MCCA can be written in matrix form as follows:

\begin{align}
    \small
    \begin{pmatrix}
        0           & \Sigma_{12} & \cdots & \Sigma_{1m} \\
        \Sigma_{21} & 0           & \cdots & \Sigma_{2m} \\
        \vdots      & \vdots      & \ddots & \vdots      \\
        \Sigma_{m1} & \Sigma_{m2} & \cdots & 0
    \end{pmatrix}
    \begin{pmatrix}
        u\sps{1} \\
        u\sps{2} \\
        \vdots   \\
        u\sps{m}
    \end{pmatrix}.
    =
    \lambda
    \begin{pmatrix}
        \Sigma_{11} & 0           & \cdots & 0           \\
        0           & \Sigma_{22} & \cdots & 0           \\
        \vdots      & \vdots      & \ddots & \vdots      \\
        0           & 0           & \cdots & \Sigma_{mm}
    \end{pmatrix}
    \begin{pmatrix}
        u\sps{1} \\
        u\sps{2} \\
        \vdots   \\
        u\sps{m}
    \end{pmatrix}.
\end{align}

This GEP formulation of MCCA can be presented in a unified framework generalizing CCA and ridge-regularized extensions. Indeed, we now take $A,B_\alpha \in \R^{D \times D}$ to be block matrices $A = (A\sps{ij})_{i,j=1}^I, B_\alpha = (B_\alpha\sps{ij})_{i,j=1}^I$ where the diagonal blocks of $A$ are zero, the off-diagonal blocks of $B_\alpha$ are zero, and the remaining blocks are defined by:
\begin{align}
    \label{eq:gep-most-general-formulation}%\small
    A^{(ij)} &= \Cov(X\sps{i}, X\sps{j}) \text{ for } i \neq j, \quad % \text{ for } i,j \in [I], ; \:\:
    B_\alpha^{(ii)} = \alpha_i I_{D\sps{i}} + (1-\alpha_i) \Var(X\sps{i})  %\text{ for } i \in [I]
\end{align}
Where $\alpha \in [0,1]^I$ is a vector of ridge penalty parameters: taking $\alpha_i = 0 \: \forall i$ recovers CCA and $\alpha = 1 \: \forall i$ recovers PLS.
We may omit the subscript $\alpha$ when $\alpha=0$ and we recover the `pure CCA' setting; in this case, following \cref{eq:cca-vector-of-correlations-def} we can define $\MCCA_K(X\sps{1},\dots,X\sps{I})$ to be the vector of the top-$K$ generalized eigenvalues.

\subsection{Linear Discriminant Analysis LDA}

Linear Discriminant Analysis (LDA) can be viewed as a special case of Canonical Correlation Analysis (CCA) where \(X^{(2)}\) is a one-hot encoded matrix representing the class labels.
This allows us to draw a connection between the unsupervised learning framework of \acrshort{cca} and the supervised framework of LDA, thus expanding the understanding of both algorithms.

\textbf{Intuition:} In LDA, the aim is to find a lower-dimensional subspace where the classes are maximally separated. This objective can be viewed through the lens of \acrshort{cca}, where the optimal directions \(u^{(1)}\) and \(u^{(2)}\) in the original and one-hot encoded spaces aim to maximize correlation. In the LDA context, \(u^{(1)}\) would maximize the separation between classes.

\subsubsection{Optimization and Solution}

Mathematically, LDA is reduced to solving a generalized eigenvalue problem involving the between-class scatter matrix \(S_B\) and the within-class scatter matrix \(S_W\):

\[
    \hat{S_B} = \sum_{i=1}^{c} n_i (\mu_i - \mu)(\mu_i - \mu)\top
\]

\[
    \hat{S_W} = \sum_{i=1}^{c} \sum_{x \in X_i} (x - \mu_i)(x - \mu_i)\top
\]

\textbf{Connection to \acrshort{cca}:} When \(X^{(2)}\) is the one-hot encoded matrix of class labels, the \acrshort{cca} problem effectively tries to maximize the correlation between the feature vectors and their corresponding labels.
This turns out to be equivalent to maximizing the between-class variance in LDA while minimizing the within-class variance.
Thus, LDA can be thought of as a constrained form of \acrshort{cca}, tailored to classification tasks.

This perspective unifies the two algorithms and shows that the core objective—finding meaningful relationships or directions in the data—is shared between both \acrshort{cca} and LDA.

\subsection{Sample Covariance and Population Covariance}
In the previous sections, the methods were described in terms of population covariance matrices such as \(\Sigma_{11}=\mathbb{E}[X\spsT{1} X\sps{1}]\), \(\Sigma_{22}=\mathbb{E}[X\spsT{2} X\sps{2}]\), and \(\Sigma_{12}=\mathbb{E}[X\spsT{1} X\sps{2}]\). These population covariances assume an underlying probability distribution from which the data are drawn.

\textbf{Sample Covariance:} In practical settings, we often do not have access to the entire population but only to a sample. Hence, we can utilize the Sample Average Approximation to estimate these covariances:

\[
    \hat{\Sigma}\sps{12} = \frac{1}{b-1} \bar{\mathbf{X}\sps{1}} \bar{\mathbf{X}\sps{2}}\top
\]

Here, \(b\) denotes the size of the minibatch, and \(\mathbf{X}\sps{1} \in \mathbb{R}^{p \times b}\) and \(\mathbf{X}\sps{2} \in \mathbb{R}^{q \times b}\) are the data matrices for the samples from \(X\sps{1}\) and \(X\sps{2}\), respectively. The bar over \(\mathbf{X}\sps{1}\) and \(\mathbf{X}\sps{2}\) signifies that these are centered versions of the matrices, i.e., the mean has been subtracted from each column.

\textbf{Practical Implications:} Using sample covariance matrices introduces some estimation error but allows us to apply the methods in real-world scenarios where population-level data are unattainable. Additionally, the use of minibatches provides a computationally efficient way to estimate these covariances in large-scale problems, at the cost of some additional statistical noise.

\textbf{Connection to Previous Methods:} The use of sample covariance matrices is directly applicable to algorithms like \acrshort{cca} and LDA. When replacing the population covariances \(\Sigma\sps{ij}\) with sample estimates, the optimization problems remain structurally similar but are solved using the sample data.

This dual perspective—considering both population and sample covariance matrices—enables a more robust and flexible approach to the methods discussed, bridging the gap between theoretical analysis and practical application.
It will be particularly useful in the context of chapter \ref{chap:loadings} where we will use population variables as ground truth while estimating the models using sample data.


\section{Practical Frameworks for Multiview Learning}

At this point, we have introduced the theoretical foundations of multiview learning, including \acrshort{cca} and its variants.
However, it is not yet clear how we should apply these methods to real-world datasets.

\subsection{Machine Learning and Statistical Inference}

Canonical Correlation Analysis (\acrshort{cca}) has been studied from both machine learning and statistical inference perspectives.
In this section, we will explore the differences between these two approaches and their implications for multiview learning.

\subsubsection{Statistical Inference Evaluation Framework}

Statistical inference approaches provide a contrasting perspective to machine learning methods, focusing on understanding and quantifying the underlying data structure:

\paragraph{Parameter Estimation}

In statistical inference, parameter estimation involves estimating model parameters and their uncertainties. This process is fundamental to understanding the data and the model's fit.

\paragraph{Hypothesis Testing}

Hypothesis testing assesses the statistical significance of the relationships found by the model. It tests whether the observed data patterns are likely to have occurred under the null hypothesis.

\paragraph{Confidence Intervals}

Confidence intervals provide ranges within which the true parameter values are likely to fall, considering uncertainty. They are essential for understanding the reliability of parameter estimates.

\paragraph{Permutation Testing}

Permutation testing is a non-parametric method that evaluates the significance of models. It compares model performance on the original data with performance on randomly shuffled data, helping to ascertain the results' robustness.

\subsubsection{Machine Learning Evaluation Framework}

\paragraph{Training, Validation, and Test Sets}

In machine learning, data is typically partitioned into training, validation, and test sets, each serving a specific purpose in the model development process:

\begin{itemize}
    \item Training Set: Used for fitting the model.
    \item Validation Set: Assists in model parameter tuning.
    \item Test Set: Evaluates the model's generalization capability.
\end{itemize}

\paragraph{Cross-Validation}

A fundamental technique in machine learning, cross-validation involves dividing the training dataset into smaller subsets for training and validation. This approach provides insights into the model's performance across different data segments.

\paragraph{Holdout Method}

The holdout method involves using a separate dataset, not involved in training or validation, for final model assessment. This ensures an unbiased performance evaluation.

\paragraph{Out of Sample Correlation}

Specific to canonical correlation analysis, this involves measuring the correlation between latent variables in new datasets, assessing the model's ability to uncover relationships in unseen data.

\paragraph{Downstream Tasks}

Evaluating model performance on downstream tasks like classification or prediction can offer practical insights into the utility of the learned representations.

\subsection{Components and Subspaces in CCA: A Subspace Perspective}

\subsubsection{Context: Eigenvalue Problems in CCA}\label{subsec:orthogonality}

While our focus so far has primarily been on the top-1 eigenvector-eigenvalue pair, it's important to note that the methodology also extends to the top-k subspace problem. This broader approach involves identifying the top-k eigenvectors and their corresponding eigenvalues.

\subsubsection{Addressing the Top-k Problem}

Transitioning from a focus on the top-1 component to exploring the top-k subspace introduces additional complexities. One common method to solve the top-k problem is to identify the top-1 component and then apply a deflation process to find subsequent orthogonal components.
Deflation involves removing the top-1 component from the data and then repeating the process to find the next top-1 component. This process is repeated until the desired number of components is found.
For instance, Hotelling's Deflation \citep{hotelling1933analysis} involves removing the top-1 component from the data, while Projection Deflation \citep{mackey2008deflation} involves projecting the data onto the orthogonal complement of the top-1 component.
Different deflation methods enforce different forms of orthogonality, which can impact the resulting components and their interpretation, particularly when the first component is not a true eigenvector.

\subsubsection{Non-Uniqueness of Components}

Furthermore, non-uniqueness is a significant challenge in CCA, particularly when eigenvectors have repeated eigenvalues. Imagine a scenario where the top-1 eigenvalue is repeated \(k\) times. In this case, there are \(k\) possible eigenvectors that can be associated with the top-1 eigenvalue. While this is unlikely to occur in practice, the eigenvalues can in practice be very close to each other, leading to numerical instability and non-uniqueness in the components. Particularly true in cross-validation settings, this non-uniqueness can lead to instability in the components, complicating their interpretation and comparison.
For example, the top-1 component in one analysis might be the second component in another analysis, making it difficult to compare the results.

This non-uniqueness also has a grounding in the probabilistic perspectives on PCA and CCA, where the latent variables are considered unique only up to a rotation.
This perspective further reinforces the subspace approach, emphasizing the identification of a subspace rather than specific directions within it.

\paragraph{Thesis Approach: Concentrating on the Top-1 Component}

In this thesis, we focus on the top-1 component in CCA to align with and facilitate comparison with typical componentwise studies in brain-behavior research. This choice is driven by the complexity associated with the top-k problem and the variety of methods available to address it. Under the assumption of a significant eigengap\footnote{An `eigengap' refers to the difference in magnitude between consecutive eigenvalues in an eigenvalue problem. A significant eigengap between the first and second eigenvalues suggests that the first eigenvalue (and its corresponding eigenvector) is distinctly more significant than the next, lending credence to its uniqueness and importance.}, the first component can be considered equivalent to the top-1 subspace. This equivalence allows for a clear and interpretable analysis, making the top-1 subspace a straightforward and reliable choice for studying multivariate data. It's important to note that while we focus on the top-1 component, the later sections of the thesis introduce a method for simultaneously solving the complete subspace, addressing broader subspace analyses.


\section{Multiview Learning in Neuroimaging}

There have been a number of applications of \acrshort{cca} and related methods to multiview problems in neuroimaging.
Using resting state fMRI data, modes of correlation have been found that relate to differences in sex and age relating to drug and alcohol abuse, depression and self harm \citep{mihalik2019brain}.
A similar mode relating to `positive-negative' wellbeing has been found across studies \citep{smith2015positive} suggesting that mental wellbeing has a relationship (though not necessarily causally) with functional connectivity between networks in the brain.
Later in this dissertation we will replicate and build on the findings from this paper by using regularised and non-linear \acrshort{cca} methods.
Owing to the high dimensionality of neuroimaging data, regularisation has been a particular focus of multiview learning in neuroimaging. \citet{mihalik2022canonical} reviews the application of \acrshort{cca} to neuroimaging data and highlights the importance of regularisation in this context. \citet{bilenko2016pyrcca} 
CCA has also been used as a preprocessing step in order to identify groups of subjects in the latent variable space.

CCA has also been used to identify relationships between neuroimaging data and other modalities such as genetics \citep{chen2016imaging}, and behavioural data \citep{chen2015imaging}.

In particular, \acrshort{cca} and clustering have been used to identify depression using fMRI data \citep{dinga2019evaluating,drysdale2017resting}.
CCA has also been used in the manner we described to denoise two \gls{views} of a dataset such as separate measures of neuroimaging data \citep{zhuang2020technical} to remove artefacts.
Deep \acrshort{cca} has recently been used to extract features for the diagnosis of schizophrenia\citep{qi2016deep}.


\section{Open challenges in Multiview Learning and CCA}

This thesis has been motivated by a number of open challenges in multiview learning and canonical correlation analysis.
Chapter \ref{chap:als} and \ref{chap:loadings} will address the first challenge, which is the regularisation of \acrshort{cca} in high dimensional settings and the interpretation of the resulting components.
Chapters \ref{chap:gradient_descent}, \ref{chap:deep_learning}, and \ref{chap:software} will address the second challenge, the efficient application of \acrshort{cca} to big data.
Finally \ref{chap:deep} will also address the third challenge, extending \acrshort{cca} to Deep Self-Supervised Learning.

\subsection{Interpretability and Regularization}

\textcolor{red}{TODO: Add a paragraph on interpretability and regularization}

\subsection{Efficient Algorithms for High-Dimensional Data}

\textcolor{red}{TODO: Add a paragraph on efficient algorithms for high-dimensional data}

\subsection{Non-linear \acrshort{cca} and Joint Embedding Self-Supervised Learning}

\textcolor{red}{TODO: Add a paragraph on non-linear \acrshort{cca} and Joint Embedding Self-Supervised Learning}

%As is standard in Kernel methods, \acrshort{kcca} can be extended to non-linear \acrshort{cca} by using a non-linear kernel function \(k\).
%This allows for the discovery of non-linear relationships between the data modalities.
%However, the non-linear kernel function \(k\) is typically fixed a priori, which can be problematic since the optimal kernel function may be unknown.
%
%\subsubsection{Deep \acrshort{cca}}
%
%Deep \acrshort{cca} (\acrshort{dcca}) is a non-linear extension of \acrshort{cca} that uses deep neural networks to find non-linear relationships between variables.
%
%\subsubsection{Joint Embedding Self-Supervised Learning}
%
%Joint Embedding Self-Supervised Learning is a method for learning representations of data that are useful for downstream tasks.
%
%\begin{figure}
%    \centering
%    \tikz{
%    % nodes
%        \node[latent, align=center, minimum size=2cm] (x) {Original Image\\x};
%        %
%        \node[obs, below left=of Z, minimum size=2cm, align=center] (x1) {Augmentated Image \\$x^{(1')}$};
%        \node[obs, below right=of Z, minimum size=2cm, align=center] (x2) {Augmentated Image \\$x^{(2')}$};
%        % edges
%        \edge{x} {x1}
%        \edge{x} {x2}}
%    \caption[Joint Embedding Data Generation Process]{\textit{\textbf{Joint Embedding Data Generation Process:}} The original image \(x\) is augmented twice to produce two augmentated images \(x^{(1')}\) and \(x^{(2')}\).}
%\end{figure}
%
%In many self-supervised learning methods, the data is augmented to produce multiple views of the same data.
%For example, in the case of images, the image can be cropped, rotated, or flipped to produce multiple views of the same image.
%The goal is to learn representations of the data that are invariant to these augmentations and thus capture the underlying structure of the data (e.g. features which are generally useful).
%From the background we have given on \acrshort{cca}, it is clear that \acrshort{cca} is a natural fit for this problem, precisely because all of the information is redundant rather than complementary; the augmentations are all views of the same data and they are chosen randomly.





