\chapter{Flexible, Scalable Regularization for Generalized Eigenvalue Problems for Multiview Subspace Learning}
\label{Regularised}

The content relating to regularised alternating least squares as a method for optimizing regularised CCA is based on an abstract which I presented in poster form at OHBM. The work benefitted from extensive theoretical discussions with John Shawe-Taylor and Janaina Mourao-Miranda as well as Janaina and Rick Adams' help interpreting the results.

\section{Introduction}

\section{Data}
\subsection{Human Connectome Project (HCP)}

The Human Connectome Project (https://www.humanconnectome.org/study/hcp-young-adult/data-releases) contains structural and resting-state fMRI data as well as behavioural, demographic, and lifestyle measures for 1,200 subjects including 1003 health subjects and a number with mental health conditions such as depression. The original goal of the HCP dataset was to understand the relationships between functional and structural images of the brain. We hypothesize that multiview machine learning will illuminate variations in depression and how it is manifested in the brain by uncovering associations between brain connectivity and non-imaging features (e.g., demographics, psychometrics and other behavioural features)

\subsection{Alzheimer's Disease Neuroimaging Initiative (ADNI)}

The Alzheimer's Disease Neuroimaging Initiative (ADNI) (https://adni.loni.usc.edu/) contains 3 cohorts of subjects with significant overlap. Subjects are labelled as Alzheimer's disease patients, Mild Cognitive Impairment subjects, and elderly controls. The ADNI dataset contains MRI and PET images as well as genetics and cognitive tests. It is aimed at determining the relationships between different biomarkers of Alzheimer's disease and, owing to the strength of the disease signal, has been a benchmark dataset for machine learning and deep learning since its release in 2017. We hypothesize that there are associations between regions of the brain which are damaged by the disease and specific behavioural outcomes like memory.


\section{Contributions}

\subsection{Simulated Data for Comparing Sparse CCA Models}

A necessary 

\subsection{Flexible Regularised ALS}

We can solve CCA by alternating minimisation over each view, based on the alternating least squares form. This form finds a variable $\bold{T}$ that is close to the latent variables $\bold{X}_i\bold{W}_i$, where $\bold{X}_i$, $\bold{W}_i$ are the matrix and weights for each view $i$. The closer $\bold{T}$ is to $\bold{X}_i\bold{W}_i$, the higher the correlation between them. The constraint  $\bold{T^{\top}T}=\bold{I}$ ensures that the latent space is orthogonal to find different effects.

\[ \underset{\bold{W},\bold{T}}{\mathrm{argmin}}\left\{\sum_i \|\bold{X}_i\bold{W}_i-\bold{T}\|_{F}^2 \right\} \]
  \[ \text{subject to: } \bold{T^{\top}T}=\bold{I} \]

To regularise the projection matrices, we add penalty terms to the objective function, such as $P(\bold{W}_i)=\lambda_i\|\bold{W}_i\|_F$ for ridge regression or $P(\bold{W}_i)=\lambda_i\|\bold{W}_i\|_1$ for lasso. This can help us avoid overfitting and improve the interpretability of the results. This means that \textbf{any regularised least squares solver} can be used to solve each subproblem, such as ridge regression, lasso, elastic net, etc. making our framework substantially more flexible than prior work.

\[ \underset{\bold{W},\bold{T}}{\mathrm{argmin}}\left\{\sum_i \|\bold{X}_i\bold{W}_i-\bold{T}\|_{F}^2 + \textcolor{red}{\lambda_i P(\bold{W}_i)}\}\right\} \]
  \[ \text{subject to: } \bold{T^{\top}T}=\bold{I} \]

A benefit of this approach is that neuroimaging practitioners may have existing code for implementing specific regularisation methods in a regression context which can be reused. Furthermore, they may have large dimensions both in features and increasingly in number of samples which benefit from using highly optimised software. 

\subsection{Flexible Regularised GEPs by Proximal Gradient Descent}

\section{Experiments: Flexible Regularised ALS for Brain-Behaviour Associations in the Human Connectome Project}

\subsection{Experiment Design}

Using FRALS, we solve a CCA problem using alternating elastic net regressions with tuned l1 and l2 regularization. We compare our proposed framework to prior work including ridge Regularised CCA (RCCA), Penalized Matrix Decomposition (PMD), commonly referred to in the literature as Sparse CCA but which strictly optimises covariance rather than correlation, and separate Principal Components Analysis (PCA) of the brain and behavioural data. We show that our method improves on RCCA by finding sparse solutions and improves on PMD by optimising a correlation rather than covariance based objective.

We used resting-state fMRI and non-imaging subject measures of 1003 healthy subjects from the 1200-subject data release of the Human Connectome Project (HCP). Following the main steps from\cite{smith2015positive}, we processed the rs-fMRI data into brain connectivity matrices. Like\cite{smith2015positive} we used 145 items of the non-imaging subject measures and removed the same confounding variables. We split the data into 80\% train and 20\% test and used cross-validation within the training data in order to tune hyperparameters.

\subsection{Results}

\subsubsection{Out of sample generalization}

\subsubsection{Model Similarities}

\subsubsection{Brain Loadings}

\subsubsection{Behaviour Loadings}

\section{Experiments: Total Variation Regularisation using Simulated Data}
\subsection{Experiment Design}
\subsection{Results}


\section{Experiments: Total Variation Regularisation of Brain-Behaviour Assocations in the ADNI Data}
\subsection{Experiment Design}
\subsection{Results}

\section{Comparison of Sparse CCA Methods in Simulated Data}
\subsection{Experiment Design}
\subsection{Results}


\section{Discussion and Conclusion}


