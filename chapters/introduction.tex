\chapter{Introduction}\label{chap:introduction}

It was June 2021, and I had self-referred to the Community Living Well service in London, UK, seeking help for my mental health. Each week, I met with my therapist and dutifully filled out the questionnaires, rating my mood and answering questions about my well-being. Yet, I couldn't shake the feeling that these snapshots were inadequate in capturing the complexity of my mental state. As a keen sportsperson, I relied on my Garmin watch to track my heart rate, sleep, and activity levels, providing a continuous stream of biometric data that painted a more nuanced picture of my physical health. Moreover, as a type 1 diabetic, my continuous glucose monitor offered real-time insights into my blood sugar levels, helping me fine-tune my insulin management. The stark contrast between these diverse data streams, each offering unique perspectives on my overall health, highlighted the challenges and opportunities that lie in integrating multimodal health data for personalized healthcare.

This challenge of integrating diverse health data extends far beyond my personal experience. In biomedical research, there is a growing need to develop methods that can effectively combine and analyze data from various sources, such as electronic health records, imaging data, and patient-reported outcomes. By leveraging the power of self-supervised learning, a machine learning approach that learns from unlabeled data by predicting parts of the input from other parts, we can potentially uncover hidden patterns and relationships in these complex datasets. Self-supervised learning is particularly well-suited for this task, as it can learn meaningful representations from vast amounts of unlabeled data, which is abundant in the biomedical domain.

This thesis focuses on developing and applying novel machine learning methods to address the challenge of integrating diverse health metrics. By harnessing the power of self-supervised and multiview learning, we aim to revolutionize the analysis and comprehension of biomedical data, ultimately enhancing our ability to understand and manage personal health. The methods developed in this thesis are not only theoretically grounded but also designed to be practical and user-friendly, enabling practitioners to focus on their domain expertise rather than the technical details of the methods.

\section{Thesis Contributions}

The main contributions of this thesis are fourfold:
\begin{enumerate}
    \item Developing a framework for regularised Canonical Correlation Analysis (CCA) using structured priors;
    \item Unifying simulated data generation methods for CCA under a latent variable model perspective;
    \item Formulating a new gradient descent approach for CCA and other generalised eigenvalue problems, tailored for large datasets;
    \item Extending the gradient descent approach to Deep CCA and Joint Embedding Self-Supervised Learning.
\end{enumerate}
These contributions have significant practical implications, from aiding in the diagnosis and treatment of neurological disorders to enabling efficient analysis of extensive health databases like the UK Biobank.

\section{Thesis Structure}

The thesis is structured as follows:
\begin{itemize}
    \item \textbf{Chapter \ref{ch:background}} reviews multiview and self-supervised learning techniques, focusing on their application in biomedical data.
    \item \textbf{Chapter \ref{ch:als}} introduces a method to regularize CCA using structured priors, demonstrated with Human Connectome Project and Alzheimer's Disease Neuroimaging Initiative data.
    \item \textbf{Chapter \ref{ch:loadings}} examines the relationship between loadings and weights in CCA, using simulated data to show the advantages of loadings for interpretability.
    \item \textbf{Chapter \ref{ch:gradient_descent}} presents a new gradient descent algorithm for generalized eigenvalue problems, demonstrated with Multiview CCA and PLS. We show how our algorithm can be applied to large datasets, using the UK Biobank as an example.
    \item \textbf{Chapter \ref{ch:deep_learning}} extends the algorithm from Chapter \ref{ch:gradient_descent} to deep learning, showing its application in scaling deep CCA. We demonstrate state-of-the-art results on CIFAR-10 and CIFAR-100 benchmarks, illustrating the potential of Deep CCA in Self-Supervised Learning.
    \item \textbf{Chapter \ref{ch:software}} introduces CCA-Zoo, a Python package implementing the methodologies of this thesis, and discusses its role in the Python ecosystem and biomedical research.
    \item \textbf{Chapter \ref{ch:discussion}} discusses the implications, challenges, and future directions for the research presented in this thesis.
\end{itemize}

Through this thesis, we aspire to bridge the gap between the potential of biomedical data and the current capabilities of analytical methods. By developing novel, scalable, and interpretable machine learning approaches, we aim to unlock the full potential of diverse health metrics, paving the way for advancements in biomedical research and personalized healthcare.