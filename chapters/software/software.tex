\graphicspath{{chapters/software/}}


\chapter{CCA-Zoo: A collection of Regularized, Deep Learning-based, Kernel, and Probabilistic methods in a scikit-learn style framework}\label{chap:ccazoo}

% \epigraph{And that's really the essence of programming. By the time you've sorted out a complicated idea into little steps that even a stupid machine can deal with, you've learned something about it yourself.}{\textit{Douglas Adams}}
\section*{Preface}

This work was published in the Journal of Open Source Software \citep{chapman2021cca}.
I have been the lead developer of the \texttt{CCA-Zoo} package since its inception in 2020.
All of the methods we have described in this thesis are implemented in \texttt{CCA-Zoo} and are immediately available for use by the research community.

\section{Introduction}

The Python programming language has seen a surge in popularity in the machine learning community due to its versatility and extensive libraries.
However, when it comes to the domain of multiview learning, there is a noticeable void in the Python ecosystem.
Existing libraries, such as \texttt{scikit-learn}\cite{pedregosa2011scikit}, offer basic implementations for CCA and PLS, yet fall short of providing a comprehensive toolkit for multiview learning techniques.
This is particularly striking given the widespread recognition that the availability of quality software implementations often acts as a catalyst for the adoption of novel methodologies in the statistical learning community.

One glaring example of this trend is Sparse PLS. Despite its known limitations, Sparse PLS has effectively become the go-to method for sparse CCA applications, primarily due to its robust implementation in the R programming language.
The discrepancy between the availability of multiview learning tools in R and Python has not only hindered the diversification of methodologies but also impeded the community from leveraging the more recent advances in the field.

\section{Background}

The research community continues to show a heightened interest in multiview learning.
Traditionally, this field has been dominated by contributions from statistical learning researchers who predominantly utilized R and MATLAB for their work.
These platforms have been the birthplace of many state-of-the-art algorithms and methodologies, including Sparse PLS.

However, this posed a challenge for Python-oriented researchers and practitioners, leaving them with two less-than-ideal options: either port existing R or MATLAB code into Python, often a non-trivial task requiring domain expertise, or resort to using the limited set of methods available in native Python libraries like \texttt{scikit-learn}.
This fragmentation has, in effect, created barriers to entry and possibly slowed down the progress in applying multiview learning techniques in Python-based projects.

The \texttt{CCA-Zoo} package aims to bridge this divide by offering a broad range of multiview learning algorithms, creating a unified platform that fosters both academic research and practical applications in Python.

\section{Methods}

In this section, we describe the implementation of \texttt{CCA-Zoo} as depicted in Figure \ref{fig:cca-zoo-api} and the design decisions that were made during its development.
We highlight the package's optimization for use with high-dimensional biomedical data and elaborate on its compatibility with standard machine learning packages.

\subsection{API}

\begin{figure}[ht]
    \centering
    \includegraphics[width=0.8\textwidth]{figures/CCA_Zoo_map}
    \caption[The \texttt{CCA-Zoo} compatibility map]{The \texttt{CCA-Zoo} compatibility map showcases integration with various machine learning packages. The deep learning module is built upon \texttt{PyTorch} and \texttt{Lightning}, reflecting their status as industry standards for neural network implementations. The probabilistic module employs \texttt{NumPyro} for its Bayesian inference capabilities, enhancing the application of probabilistic approaches in CCA.}
    \label{fig:cca-zoo-api}
\end{figure}

The \texttt{scikit-learn} API is familiar to many machine learning practitioners and researchers, and is the de facto standard for machine learning in Python. \texttt{CCA-Zoo} has been designed to be consistent with the \texttt{scikit-learn} API, inheriting its user-friendly characteristics and ensuring compatibility with the \texttt{scikit-learn} ecosystem. Furthermore, the deep module within \texttt{CCA-Zoo} integrates \texttt{PyTorch} and \texttt{Lightning}, harnessing their powerful features for deep learning research and applications. The probabilistic module takes advantage of \texttt{NumPyro}, which offers advanced features for probabilistic programming and Bayesian methods, further extending the versatility and functionality of \texttt{CCA-Zoo}.


\subsection{Usage}

Use of the \texttt{CCA-Zoo} package is straightforward and intuitive, as demonstrated in the following example, which implements a regularized CCA model with a ridge penalty.

\begin{minted}{python}
    # Import required libraries
    import numpy as np
    from cca_zoo.datasets import LatentVariableData
    from cca_zoo.linear import rCCA
    from cca_zoo.model_selection import GridSearchCV
    
    # Generate synthetic multiview data
    data = LatentVariableData(view_features=[10,10],latent_dims: int = 2)
    (X,Y) = data.sample(n_samples=100)

    # Define grid of potential regularization parameters
    c1 = [0.1, 0.3, 0.7, 0.9]
    c2 = [0.1, 0.3, 0.7, 0.9]
    param_grid = {'c': [c1, c2]}

    cv = 5  # Number of folds in cross-validation

    # Conduct grid search
    ridge = GridSearchCV(rCCA(latent_dimensions=2), param_grid=param_grid,
                        cv=cv, verbose=True, scoring=scorer).fit((train_view_1, train_view_2)).best_estimator_
    \end{minted}

\subsection{Linear}

\begin{table}[ht]
    \centering
    \begin{tabular}{|c|c|}
        \hline
        Class Name & Method Name \\
        \hline
        "MCCA" & Multiview CCA \\
        "CCA" & Canonical Correlation Analysis \\
        "rCCA" & Ridge CCA \\
        "PLS" & Partial Least Squares \\
        "MPLS" & Multiview Partial Least Squares \\
        "GCCA" & Generalized CCA \\
        "GRCCA" & Group Ridge Regularized CCA \\
        "PartialCCA" & Partial CCA \\
        "PRCCA" &  \\
        "TCCA" & Tensor CCA \\
        "PCACCA" & PCA CCA \\
        "SCCA_IPLS" & Sparse CCA using Iterative Lasso \\
        "ElasticCCA" & Elastic CCA using FRALS \\
        "PLS_ALS" & PLS using Alternating Least Squares \\
        "SPLS" & Sparse PLS \\
        "SCCA_Parkhomenko" & Penalized CCA \\
        "SCCA_Span" & Sparse CCA using Span Bound \\
        "CCA_EY" & CCA by Eckart-Young \\
        "PLS_EY" & PLS by Eckart-Young \\
        "CCA_GHA" & CCA using Generalized Hebbian Algorithm \\
        "CCA_SVD" & CCA using SVD \\
        "PLSStochasticPower" & PLS using Stochastic Power Method \\
        \hline
    \end{tabular}
    \caption{Class Names and Method Names}
    \label{tab:class_method}
\end{table}


\subsection{Deep}

\begin{table}[ht]
    \centering
    \begin{tabular}{|c|c|}
        \hline
        Class Name & Method Name \\
        \hline
        "DCCA" & Deep CCA \\
        "DCCA_GHA" & Deep CCA by Generalized Hebbian Algorithm \\
        "DCCA_SVD" & Deep CCA by SVD \\
        "DMCCA" & Deep Multiview CCA \\
        "DGCCA" & Geep Generalised CCA \\
        "DCCAE" & Deep Canonically Correlated Autoencoders \\
        "DCCA_NOI" & Deep CCA by nonlinear orthogonal iterations \\
        "DCCA_SDL" & Deep CCA by stochastic decorrelation loss \\
        "DVCCA" & Deep Variaitional CCA \\
        "BarlowTwins" & Barlow Twins \\
        "VICReg" & VICReg \\
        "DTCCA" & Deep Tensor CCA \\
        "DCCA_EY" & Deep CCA by Eckart-Young \\
        "architectures" & \\
        "objectives" & \\
        \hline
    \end{tabular}
    \caption{Class Names and Method Names}
    \label{tab:class_method_2}
\end{table}

\subsection{Model Selection Utilities}



\subsection{Code Availability}

The code for \texttt{CCA-Zoo} is available at.

\texttt{CCA-Zoo} has received 155 stars and 30 forks on GitHub, and has nearly 500 downloads per month on PyPI \footnote{https://pypistats.org/packages/cca-zoo}.

Documentation for \texttt{CCA-Zoo} is available at \footnote{https://cca-zoo.readthedocs.io/en/latest/}.
The documentation includes a user guide, API reference, and examples.

The package can be installed using \texttt{pip install cca-zoo} or \texttt{poetry add cca-zoo}.



\section{Benchmarking}

In this section, we compare the performance of \texttt{CCA-Zoo} against \texttt{scikit-learn}, focusing on the efficiency of the basic CCA and PLS methods.
We conducted experiments on synthetic datasets with varying dimensions to evaluate their average execution time.
The datasets consisted of random matrices with a varying number of dimensions: \(50\), \(100\), \(200\), \(400\), and \(800\). Each matrix had \(100\) samples. We set the latent dimensions for both CCA and PLS to \(10\). For each dimension, the experiment was repeated \(10\) times to obtain reliable performance metrics.

\paragraph{Libraries Used:}
\begin{itemize}
    \item \texttt{CCA-Zoo} (version: 2.4.0)
    \item \texttt{Scikit-learn} (version: 1.3.0)
\end{itemize}

\subsection{Canonical Correlation Analysis:}
Figure~\ref{fig:cca_benchmark} presents the comparison between \texttt{CCA-Zoo} and \texttt{scikit-learn} for Canonical Correlation Analysis. We observe that \texttt{CCA-Zoo} exhibits a competitive runtime profile when compared to \texttt{scikit-learn} across all dimensions.

\begin{figure}[h]
    \centering
    \includegraphics[width=0.8\textwidth]{figures/CCA_Speed_Benchmark}
    \caption{Performance comparison for CCA methods}
    \label{fig:cca_benchmark}
\end{figure}

\subsection{Partial Least Squares:}
The comparison for Partial Least Squares is shown in Figure \ref{fig:pls_benchmark}.
Like the CCA experiment, \texttt{CCA-Zoo} maintains a robust performance profile that is competitive with \texttt{scikit-learn}.

\begin{figure}[h]
    \centering
    \includegraphics[width=0.8\textwidth]{figures/CCA_Speed_Benchmark}
    \caption{Performance comparison for PLS methods}
    \label{fig:pls_benchmark}
\end{figure}

The results indicate that \texttt{CCA-Zoo} is an efficient Python package for both CCA and PLS methods, holding its own against the widely-used \texttt{scikit-learn} library.
These experiments underscore the capability of \texttt{CCA-Zoo} to handle high-dimensional data efficiently, making it a suitable choice for applications in bioinformatics, natural language processing, and other high-dimensional data domains.

\subsection{Conclusion}

\texttt{CCA-Zoo} has not only served as a tool for my research but aims to be a community resource that can accelerate research and application in multiview learning.
Its design decisions, such as API compatibility and focus on both linear and deep models, reflect a comprehensive understanding of the challenges and opportunities in this field.
