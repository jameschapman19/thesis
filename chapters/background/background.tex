\graphicspath{{chapters/background/}}
\chapter{Background: Multiview Machine Learning: Concepts, Methods, and Limitations}\label{chap:background}
\minitoc
%\epigraph{Principal Component Analysis is a dimensionally invalid method that gives people a delusion that they are doing something useful with their data. If you change the units that one of the variables is measured in, it will change all the “principal components”! It’s for that reason that I made no mention of PCA in my book. I am not a slavish conformist, regurgitating whatever other people think should be taught. I think before I teach.}{Professor David MacKay}
\section{Introduction to Machine Learning and Multiview Learning}

In this chapter, we gather the necessary background knowledge needed to motivate and understand the contributions of this thesis.

Machine learning enables models to automatically learn patterns and make decisions from data.
Machine learning comprises three primary paradigms: supervised, self-supervised (in the past called unsupervised), and reinforcement learning, each distinct in its approach to learning from data.
This thesis focuses on \textit{multiview machine learning}, which aims to develop robust representations by uncovering associations between various data types within datasets.
These data types, known as \gls{views} may include distinct sources of information such as MRI images, genomic data, and clinical records in the context of patient data analysis.

\subsection{Types of Multiview Machine Learning}

Multiview machine learning encompasses a variety of techniques aimed at learning from data that have multiple sources or modalities, also known as \gls{views}.

\subsection{Conditional Independence, Causality, and Multiview Learning}

Consider the graphical model depicted in Figure~\ref{fig:mentalhealthselfsupervised}.
It comprises two distinct \gls{views}: a brain modality and a behavioral modality.
The graphical model assumes that the brain and behaviour are conditionally independent given the severity of an (unobserved or `latent') mental health condition.

\begin{figure}
    \centering
    \tikz{
        % nodes
        \node[latent, align=center, minimum size=2cm] (Z) {Severity\\z};
        %
        \node[obs, below left=of Z, minimum size=2cm, align=center] (x1) {Brain\\$x^{(1)}$};
        \node[obs, below right=of Z, minimum size=2cm, align=center] (x2) {Behaviour\\$x^{(2)}$};
        % edges
        \edge{Z} {x1}
        \edge{Z} {x2}}
    \caption[Latent Variable Model of Mental Health]{\textit{\textbf{Latent Variable Model of Mental Health:}} From this perspective the neuroimaging modality and behavioural data are both considered to have been generated with distributions conditioned on the severity of a mental health condition}\label{fig:mentalhealthselfsupervised}
\end{figure}

In multiview machine learning, the relationship between conditional independence and causality is nuanced but crucial.
When examining dependencies between events, such as those observed between brain activity and behavior, several scenarios emerge:

\begin{itemize}
    \item direct causation (brain influencing behavior or vice versa or even both)
    \item both being influenced by a common, possibly unobserved, cause
    \item no direct causal link between them
\end{itemize}

Importantly, if a common cause does exist, conditioning on it renders brain and behavior independent; this 'screens off' their dependence, revealing key insights for our models\cite{reichenbach1956direction}.
However, it is essential to recognize that the presence of a common latent variable, inferred from these \gls{views}, does not automatically imply causality in the observed data.

\subsubsection{Complementary and Redundant Information}
The nature of the information provided by different \gls{views} (such as neuroimaging and behavioral data) is critical in multiview machine learning.

If these \gls{views} contain redundant information about the latent variables, they can be used to denoise each other, enhancing the clarity and reliability of the data.
This process exploits the overlap in information to correct or reduce noise and errors, a principle fundamental to many denoising techniques in machine learning.

Conversely, if the \gls{views} provide complementary information, they offer a richer and more complete understanding of the latent variables than either view alone.
In this case, the integration of diverse but related datasets leads to more robust and insightful models.
Complementary \gls{views} expand our perspective, revealing aspects of the latent variables that are not apparent when \gls{views} are considered in isolation.

This understanding is critical in mental health research, where accurate interpretations of complex relationships are key.

%These techniques can be broadly classified into supervised and self-supervised (or sometimes, equivalently, unsupervised) multiview learning, with some algorithms straddling the boundary between the two.

\subsubsection{Supervised Multiview Learning}

In supervised multiview learning, one view serves as the input while the other view is treated as the target label.
The algorithm learns to predict the target view based on the input view, leveraging the information from both to enhance the predictive performance.
This approach is especially useful when each view contains complementary information that can improve the model's accuracy or robustness.

\subsubsection{Self-Supervised Multiview Learning}

Self-Supervised Learning (SSL) is a paradigm where the training signal is derived from the data itself, rather than relying on external labels.
The cornerstone of SSL is the concept of a `pretext task,' a learning task created from the data that trains the model to capture useful features or representations.
In the context of multiview machine learning, self-supervised learning often operates under the assumption that different \gls{views} are generated from a common source.
A natural pretext task, in this case, is to predict these source from the given \gls{views}.
This not only enables the model to learn associations between \gls{views} but also allows it to derive robust and informative representations for subsequent tasks like classification or regression.
This is particularly true if the source is unavailable at test time (and, in the case of latent variables, even at training time), but knowledge of the source is useful for downstream tasks.

\section{Learning Representations: Definitions and Notation}

Suppose we have a sequence of vector-valued random variables $X\sps{i} \in \R^{D_i}$ for $i \in \{1, \dots, I \}$
We want to learn meaningful $K$-dimensional representations
\begin{equation}\label{eq:general-form-of-representations}
    Z\sps{i} = f\sps{i}( X\sps{i}; \theta\sps{i}).
\end{equation}
For convenience, define $D = \sum_{i=1}^I D_i$ and $\theta = \left(\theta\sps{i}\right)_{i=1}^I$.
Without loss of generality take $D_1 \geq D_2 \geq \cdots \geq D_I$.
We will consistently use the subscripts $i,j \in [I]$ for \gls{views};
$d \in [D_i]$ for dimensions of input variables;
and $l,k \in [K]$ for dimensions of representations - i.e. to subscript dimensions of $Z\sps{i}, f\sps{i}$.
Later on we will introduce total number of samples $N$.

In this report, we will typically refer to $u_k$ as \textbf{\gls{weights}}, $Z_k = X_k u_k$ as \gls{representations} or \gls{latent variables}, depending on the context. We will sometimes consider a
matrix $U = \left(u_1, \dots, u_K\right) \in \R^{D \times K}$ of \gls{weights}, and a
matrix $Z = \left(Z_1, \dots, Z_K\right) \in \R^{N \times K}$ of representations.
We will refer to $\Corr(X\sps{i}_j, Z_k)$ as the loadings of $X\sps{i}_j$ on $Z_k$.

\subsection{Principal Components Analysis}

Principal Components Analysis\cite{hotelling1933analysis} (\acrshort{pca}) is a classical method in unsupervised machine learning for representation learning.
It is widely used for dimensionality reduction and feature extraction.
The primary goal of \acrshort{pca} is to transform the original high-dimensional data into a new coordinate system defined by orthogonal axes, capturing the most relevant aspects of the data.

In \acrshort{pca}, the representations are constrained to be linear transformations of the form:
\begin{equation}\label{eq:pca-linear-function-def}
    Z_k = X u_k,
\end{equation}
where $u_k$ are the orthonormal basis vectors:
\begin{equation}\label{eq:pca-orthonormality-constraint}
    u_k^\top u_l = \delta_{kl}.
\end{equation}

The primary goal of \acrshort{pca} is to maximize the variance of the projections \(Z_k\). Mathematically, this can be formulated
as:
\begin{align}
    u_{\text{opt}} & = \underset{u}{\text{argmax}} \left( u^\top \Sigma u \right) \\
    \text{subject to:} \notag                                                     \\
    u^\top u       & = 1 \notag
\end{align}

Where \(\Sigma = \mathbb{E}[X^\top X]\) is the covariance matrix of the data. The solution to this optimization problem is the eigenvector associated with the largest eigenvalue of \(\Sigma\).

\paragraph{Optimization and Solution}
The Lagrangian for this problem is:
\begin{equation}
    f(u,\lambda) = u^\top \Sigma u + \lambda(1 - u^\top u),
\end{equation}
where \(\lambda\) is the Lagrange multiplier. Differentiating the Lagrangian yields the first-order conditions:
\begin{align}
    \Sigma u & = \lambda u, \\
    u^\top u & = 1.
\end{align}

This transforms the problem into an eigenvalue equation for the covariance matrix \(\Sigma\), which can be efficiently solved using standard libraries such as scikit-learn\citep{pedregosa2011scikit}.

The first principal component corresponds to the eigenvector associated with the largest eigenvalue \(\lambda\). Subsequent components are the remaining eigenvectors ordered by their corresponding eigenvalues.

\textbf{Limitations: }However, when applying \acrshort{pca} to datasets such as high-dimensional neuroimaging and behavioral
data, \acrshort{pca}'s main limitation arises: it only accounts for variance within a single dataset, potentially discarding features that are relevant for cross-modal analysis.

\subsection{Partial Least Squares}

Partial Least Squares (PLS)\citep{wold1975path} aims to maximize the shared covariance between two paired sets of data, referred to as \gls{views}. \acrshort{pls} can be seen as a generalization of \acrshort{pca}, where \acrshort{pca} becomes a special case when the two \gls{views} are identical. The optimization problem for \acrshort{pls} can be formulated as:

\begin{align}
    u\sps{1}_{\text{opt}} & = \underset{u\sps{1}}{\mathrm{argmax}} \{ u\spstop{1} \Sigma_{12} u\sps{2} \} \\
    \text{subject to:} \notag                                                                             \\
    u\spstop{1}u\sps{1}   & = 1 \notag                                                                    \\
    u\spstop{2}u\sps{2}   & = 1 \notag
\end{align}

where \( X\sps{1} \in \mathbb{R}^{n \times p_1} \) and \( X\sps{2} \in \mathbb{R}^{n \times p_2} \), meaning we have two \gls{views} with the same number of samples but potentially different number of features.

\subsubsection{Eigenvalue Problem}

The Lagrangian for this optimization problem can be formulated as:

\begin{equation}
    f(u\sps{1}, \lambda) = u\spstop{1} \Sigma_{12} u\sps{2} + \lambda_1 (1 - u\spstop{1}u\sps{1}) + \lambda_2 (1 - u\spstop{2}u\sps{2})
\end{equation}

Upon deriving the first order conditions, we get:

\begin{align}
    \Sigma_{21} u\sps{1} & = \lambda_2 u\sps{2} \\
    \Sigma_{12} u\sps{2} & = \lambda_1 u\sps{1} \\
    u\spstop{1}u\sps{1}  & = 1                  \\
    u\spstop{2}u\sps{2}  & = 1
\end{align}

By substituting the constraint conditions into these equations, we find that \( \lambda_1 = \lambda_2 = \lambda \) by symmetry. Further simplification yields:

\begin{align}
    \Sigma_{21} \Sigma_{12} u\sps{2} & = \lambda^2 u\sps{2} \\
    \Sigma_{12} \Sigma_{21} u\sps{1} & = \lambda^2 u\sps{1}
\end{align}

Thus, solving these equations will yield the \( u\sps{1} \) and \( u\sps{2} \) vectors as the eigenvectors of \( \Sigma_{12} \Sigma_{21} \) and \( \Sigma_{21} \Sigma_{12} \), respectively \citep{hoskuldsson1988pls}.

\textbf{Limitations: } The problem with applying \acrshort{pls} to neuroimaging and behavioural modalities is that \acrshort{pls} is not scale invariant and
is therefore biased towards the largest principal components in the data \citep{helmer2020stability}.
This is particularly problematic when there is a low signal to noise ratio since \acrshort{pls} may find directions in either dataset which correspond to the largest directions of noise in the other.
Additionally, \acrshort{pls} assumes that the structures contributing to variance in both datasets are linearly related, which
may not be the case in complex biological systems like the brain or in intricate behavioral patterns \citep{rosipal2005overview}.
The linearity assumption can sometimes be overly restrictive, failing to capture more complicated, nonlinear relationships between the data modalities.
Another issue is the lack of sparsity in the \acrshort{pls} solution.
Traditional \acrshort{pls} methods do not provide sparse weight vectors, which makes the interpretation of results challenging in high-dimensional settings such as neuroimaging where only a subset of features might be relevant \citep{leurgans1993canonical}.
There are sparse variants of \acrshort{pls} available, but these typically introduce additional complexity and may require fine-tuning of regularization parameters \citep{chun2010sparse}.
Furthermore, \acrshort{pls} can be sensitive to outliers, which are not uncommon in neuroimaging data due to motion artifacts or other sources of noise.
Since the method aims to maximize covariance, extreme values in one dataset can disproportionately affect the resulting latent variables \citep{wold1975path}.

\subsection{Canonical Correlation Analysis}\label{sec:cca}

In \acrshort{cca}, we aim to find the directions that maximize correlation, as opposed to maximizing covariance between two \gls{views} of a dataset.
This nuance renders \acrshort{cca} invariant to feature scale. The optimization problem for \acrshort{cca} can be expressed as:

\begin{align}
     & u_{\text{opt}}=\underset{u}{\mathrm{argmax}}\{ u\spstop{1}X\spstop{1}X\sps{2}u\sps{2} \} \\
     & \text{subject to:} \notag                                                                \\
     & u\spstop{1}\Sigma_{11}u\sps{1}=1 \notag                                                  \\
     & u\spstop{2}\Sigma_{22}u\sps{2}=1 \notag
\end{align}

Although non-convex, numerous methods exist for solving the \acrshort{cca} problem, such as eigenvalue problems, generalized eigenvalue problems, block coordinate descent via alternating least squares regressions \citep{golub1995canonical,sun2008least} , and gradient descent \citep{via2007learning}.

\subsubsection{Eigenvalue Problem}

The first-order conditions derived in the same manner as the \acrshort{pls} case are:

\begin{align}\label{CCA:FOCs}
     & \Sigma_{21}u\sps{1}=\lambda\sps{2} \Sigma_{22}u\sps{2} \\
     & \Sigma_{12}u\sps{2}=\lambda\sps{1} \Sigma_{11}u\sps{1} \\
     & u\spstop{1}\Sigma_{11}u\sps{1}=1                       \\
     & u\spstop{2}\Sigma_{22}u\sps{2}=1
\end{align}

Substituting the second two conditions into the first two, we get \(\lambda\sps{1}=\lambda\sps{2}=\lambda\). Then, recognizing \(X_i^{\top}X_i\) as the covariance matrix \(\Sigma_{ii}\) and \(X_i^{\top}X_j\) as the cross-covariance matrix \(\Sigma_{ij}\), we obtain another pair of eigenvalue problems:

\begin{align}
     & \Sigma_{11}^{-1}\Sigma_{12}\Sigma_{22}^{-1}\Sigma_{21}u\sps{1}=\lambda^2u\sps{1} \notag \\
     & \Sigma_{22}^{-1}\Sigma_{21}\Sigma_{11}^{-1}\Sigma_{12}u\sps{2}=\lambda^2u\sps{2} \notag
\end{align}

An alternative form of the \acrshort{cca} problem can be developed by reparameterizing \(u\sps{i*}=(\Sigma_{ii})^{-\frac{1}{2}}u\sps{i}\). The optimization problem then becomes:

\begin{align}
     & u_{\text{opt}}=\underset{u}{\mathrm{argmax}}\{ u\spstop{1}\Sigma_{11}^{-\frac{1}{2}}\Sigma_{12}\Sigma_{22}^{-\frac{1}{2}}u\sps{2} \} \\
     & \text{subject to:} \notag                                                                                                            \\
     & u\spstop{1}u\sps{1}=1 \notag                                                                                                         \\
     & u\spstop{2}u\sps{2}=1 \notag
\end{align}

This reparameterized form will later underpin Deep Canonical Correlation Analysis (\acrshort{dcca}).

This form also shows that \acrshort{pls} and \acrshort{cca} can be made equivalent by whitening the data matrices before constructing the covariance matrix. When the number of features exceeds the number of samples (\(p>n\)), \acrshort{cca} becomes degenerate because the within-view covariance matrices cannot be inverted—contrasting with \acrshort{pls}, which is always computable.

\subsubsection{Generalized Eigenvalue Problems}

We can also represent the system of equations in equation~\ref{CCA:FOCs} as a matrix equation:

\begin{align}
    \begin{pmatrix}
        0           & \Sigma_{12} \\
        \Sigma_{21} & 0
    \end{pmatrix}
    \begin{pmatrix}
        u\sps{1} \\
        u\sps{2}
    \end{pmatrix}
    =
    \lambda
    \begin{pmatrix}
        \Sigma_{11} & 0           \\
        0           & \Sigma_{22}
    \end{pmatrix}
    \begin{pmatrix}
        u\sps{1} \\
        u\sps{2}
    \end{pmatrix}
\end{align}

Which is of the form $\mathbf{A v} = \lambda \mathbf{B v}$. \acrshort{cca} is therefore often referred to as a generalized eigenvalue problem for which there are a number of publicly available solvers.

\subsubsection{LDA as a Special Case of \acrshort{cca}}

Linear Discriminant Analysis (LDA) can be viewed as a special case of Canonical Correlation Analysis (CCA) where \(X^{(2)}\) is a one-hot encoded matrix representing the class labels.
This allows us to draw a connection between the unsupervised learning framework of \acrshort{cca} and the supervised framework of LDA, thus expanding the understanding of both algorithms.

\textbf{Intuition:} In LDA, the aim is to find a lower-dimensional subspace where the classes are maximally separated. This objective can be viewed through the lens of \acrshort{cca}, where the optimal directions \(u^{(1)}\) and \(u^{(2)}\) in the original and one-hot encoded spaces aim to maximize correlation. In the LDA context, \(u^{(1)}\) would maximize the separation between classes.

Mathematically, LDA is reduced to solving a generalized eigenvalue problem involving the between-class scatter matrix \(S_B\) and the within-class scatter matrix \(S_W\):

\[
    \hat{S_B} = \sum_{i=1}^{c} n_i (\mu_i - \mu)(\mu_i - \mu)\top
\]

\[
    \hat{S_W} = \sum_{i=1}^{c} \sum_{x \in X_i} (x - \mu_i)(x - \mu_i)\top
\]

\textbf{Connection to \acrshort{cca}:} When \(X^{(2)}\) is the one-hot encoded matrix of class labels, the \acrshort{cca} problem effectively tries to maximize the correlation between the feature vectors and their corresponding labels.
This turns out to be equivalent to maximizing the between-class variance in LDA while minimizing the within-class variance.
Thus, LDA can be thought of as a constrained form of \acrshort{cca}, tailored to classification tasks.

This perspective unifies the two algorithms and shows that the core objective—finding meaningful relationships or directions in the data—is shared between both \acrshort{cca} and LDA.

\textbf{Multiview \acrshort{cca}} is a straightforward extension of \acrshort{cca} to the case of 3-or more datasets.
The optimization problem for \acrshort{mcca} can be stated as:
\begin{align}
     & u_{\text{opt}} = \underset{u}{\mathrm{argmax}} \sum_{i=1}^{m} \sum_{j=1, j \neq i}^{m} u\spstop{i} \Sigma_{ij} u\sps{j} \\
     & \text{subject to:} \notag                                                                                               \\
     & \sum_{i=1}^{m} u\spstop{i} \Sigma_{ii} u\sps{i} = 1 \notag
\end{align}

The generalized eigenvalue problem (GEP) can be written in matrix form as follows:

\begin{align}
    \mathbf{A} \mathbf{U} & = \lambda \mathbf{B} \mathbf{U}                    \\
    \mathbf{A}            & = \begin{pmatrix}
                                  0           & \Sigma_{12} & \cdots & \Sigma_{1m} \\
                                  \Sigma_{21} & 0           & \cdots & \Sigma_{2m} \\
                                  \vdots      & \vdots      & \ddots & \vdots      \\
                                  \Sigma_{m1} & \Sigma_{m2} & \cdots & 0
                              \end{pmatrix}, \\
    \mathbf{B}            & = \begin{pmatrix}
                                  \Sigma_{11} & 0           & \cdots & 0           \\
                                  0           & \Sigma_{22} & \cdots & 0           \\
                                  \vdots      & \vdots      & \ddots & \vdots      \\
                                  0           & 0           & \cdots & \Sigma_{mm}
                              \end{pmatrix}, \\
    \mathbf{U}            & = \begin{pmatrix}
                                  u\sps{1} \\
                                  u\sps{2} \\
                                  \vdots   \\
                                  u\sps{m}
                              \end{pmatrix}.
\end{align}

\subsection{Sample Covariance and Population Covariance}
In the previous sections, the methods were described in terms of population covariance matrices such as \(\Sigma_{11}=\mathbb{E}[X\spstop{1} X\sps{1}]\), \(\Sigma_{22}=\mathbb{E}[X\spstop{2} X\sps{2}]\), and \(\Sigma_{12}=\mathbb{E}[X\spstop{1} X\sps{2}]\). These population covariances assume an underlying probability distribution from which the data are drawn.

\textbf{Sample Covariance:} In practical settings, we often do not have access to the entire population but only to a sample. Hence, we can utilize the Sample Average Approximation to estimate these covariances:

\[
    \hat{\Sigma}\sps{12} = \frac{1}{b-1} \bar{\mathbf{X}\sps{1}} \bar{\mathbf{X}\sps{2}}\top
\]

Here, \(b\) denotes the size of the minibatch, and \(\mathbf{X}\sps{1} \in \mathbb{R}^{p \times b}\) and \(\mathbf{X}\sps{2} \in \mathbb{R}^{q \times b}\) are the data matrices for the samples from \(X\sps{1}\) and \(X\sps{2}\), respectively. The bar over \(\mathbf{X}\sps{1}\) and \(\mathbf{X}\sps{2}\) signifies that these are centered versions of the matrices, i.e., the mean has been subtracted from each column.

\textbf{Practical Implications:} Using sample covariance matrices introduces some estimation error but allows us to apply the methods in real-world scenarios where population-level data are unattainable. Additionally, the use of minibatches provides a computationally efficient way to estimate these covariances in large-scale problems, at the cost of some additional statistical noise.

\textbf{Connection to Previous Methods:} The use of sample covariance matrices is directly applicable to algorithms like \acrshort{cca} and LDA. When replacing the population covariances \(\Sigma\sps{ij}\) with sample estimates, the optimization problems remain structurally similar but are solved using the sample data.

This dual perspective—considering both population and sample covariance matrices—enables a more robust and flexible approach to the methods discussed, bridging the gap between theoretical analysis and practical application.


\section{Practical Frameworks for Multiview Learning}

\subsection{Machine Learning and Statistical Inference}

\subsubsection{Machine Learning Evaluation Framework}

\paragraph{Training, Validation, and Test Sets}

In machine learning, data is typically partitioned into training, validation, and test sets, each serving a specific purpose in the model development process:

\begin{itemize}
\item Training Set: Used for fitting the model.
\item Validation Set: Assists in model parameter tuning.
\item Test Set: Evaluates the model's generalization capability.
\end{itemize}

\paragraph{Cross-Validation}

A fundamental technique in machine learning, cross-validation involves dividing the training dataset into smaller subsets for training and validation. This approach provides insights into the model's performance across different data segments.

\paragraph{Holdout Method}

The holdout method involves using a separate dataset, not involved in training or validation, for final model assessment. This ensures an unbiased performance evaluation.

\paragraph{Out of Sample Correlation}

Specific to canonical correlation analysis, this involves measuring the correlation between latent variables in new datasets, assessing the model's ability to uncover relationships in unseen data.

\paragraph{Downstream Tasks}

Evaluating model performance on downstream tasks like classification or prediction can offer practical insights into the utility of the learned representations.

\subsubsection{Statistical Inference Evaluation Framework}

Statistical inference approaches provide a contrasting perspective to machine learning methods, focusing on understanding and quantifying the underlying data structure:

\begin{itemize}
\item Parameter Estimation: Involves estimating model parameters and their uncertainties.
\item Hypothesis Testing: Assesses the statistical significance of the relationships found by the model.
\item Confidence Intervals: Provide ranges within which the true parameter values are likely to fall, considering uncertainty.
\item Permutation Testing: A non-parametric method that evaluates the significance of models by comparing model performance on original and randomly shuffled data.
\end{itemize}

\subsection{Components and Subspaces in CCA: A Subspace Perspective}

\subsection{Components and Subspaces in CCA: A Subspace Perspective}

\subsubsection{Context: Eigenvalue Problems in CCA}\label{subsec:orthogonality}

While our focus so far has primarily been on the top-1 eigenvector-eigenvalue pair, it's important to note that the methodology also extends to the top-k subspace problem. This broader approach involves identifying the top-k eigenvectors and their corresponding eigenvalues.

\subsubsection{Addressing the Top-k Problem}

Transitioning from a focus on the top-1 component to exploring the top-k subspace introduces additional complexities. One common method to solve the top-k problem is to identify the top-1 component and then apply a deflation process to find subsequent orthogonal components. However, this is not the only approach, and other methods may also be employed.

\subsubsection{Deflation Methods in Subspace Identification}

While Hotelling's Deflation and Projection Deflation are crucial for ensuring orthogonality in the deflation process, it's important to recognize that they represent just one approach to identifying multiple components in CCA. These methods, although effective, can be sensitive to the initial conditions of the data and can influence the outcome of the top-k subspace identification.

\subsubsection{Non-Uniqueness of Components}

Furthermore, non-uniqueness is a significant challenge in CCA, particularly when eigenvectors have repeated eigenvalues. Imagine a scenario where the top-1 eigenvalue is repeated \(k\) times. In this case, there are \(k\) possible eigenvectors that can be associated with the top-1 eigenvalue. While this is unlikely to occur in practice, the eigenvalues can be very close to each other, leading to numerical instability and non-uniqueness in the components. Particularly true in cross-validation settings, this non-uniqueness can lead to instability in the components, complicating their interpretation and comparison across different data analyses.

This non-uniqueness aligns with the probabilistic perspectives on PCA and CCA, where the latent variables are considered unique only up to a rotation. This perspective further reinforces the subspace approach, emphasizing the identification of a subspace rather than specific directions within it.

\paragraph{Thesis Approach: Concentrating on the Top-1 Component}

In this thesis, we focus on the top-1 component in CCA to align with and facilitate comparison with typical componentwise studies in brain-behavior research. This choice is driven by the complexity associated with the top-k problem and the variety of methods available to address it. Under the assumption of a significant eigengap\footnote{An 'eigengap' refers to the difference in magnitude between consecutive eigenvalues in an eigenvalue problem. A significant eigengap between the first and second eigenvalues suggests that the first eigenvalue (and its corresponding eigenvector) is distinctly more significant than the next, lending credence to its uniqueness and importance.}, the first component can be considered equivalent to the top-1 subspace. This equivalence allows for a clear and interpretable analysis, making the top-1 subspace a straightforward and reliable choice for studying multivariate data. It's important to note that while we focus on the top-1 component, the later sections of the thesis introduce a method for simultaneously solving the complete subspace, addressing broader subspace analyses.

\section{Multiview Learning in Neuroimaging}

There have been a number of applications of \acrshort{cca} and related methods to multiview problems in neuroimaging.
Using resting state fMRI data, modes of correlation have been found that relate to differences in sex and age relating to drug and alcohol abuse, depression and self harm \citep{mihalik2019brain}.
A similar mode relating to `positive-negative' wellbeing has been found across studies \citep{smith2015positive}suggesting that mental wellbeing has a relationship (though not necessarily causally) with functional connectivity between networks in the brain.
Later in this dissertation we will replicate and build on the findings from this paper by using regularised and non-linear \acrshort{cca} methods.

CCA has also been used as a preprocessing step in order to identify groups of subjects in the latent variable space.
In particular, \acrshort{cca} and clustering have been used to identify depression using fMRI data\citep{dinga2019evaluating} \citep{drysdale2017resting}.
CCA has also been used in the manner we described to denoise two \gls{views} of a dataset such as separate measures of neuroimaging data \citep{zhuang2020technical} to remove artefacts.
Deep \acrshort{cca} has recently been used to extract features for the diagnosis of schizophrenia\citep{qi2016deep}.

\section{Open challenges in Multiview Learning and CCA}

\subsection{Interpretability and Regularization}



\subsection{Efficient Algorithms for High-Dimensional Data}

The challenges of high-dimensional data often manifest when solving Generalized Eigenvalue Problems (GEPs) for Canonical Correlation Analysis (CCA). The computational burden of solving these problems becomes daunting as the number of features grows.
To combat this issue, various efficient algorithms have been developed to reduce the complexity.
In this section, we will explore some of these strategies.

\subsubsection{Challenges in Solving Generalized Eigenvalue Problems}

The GEP is often represented as \( Ax = \lambda Bx \), where \( A \)and \( B \)are matrices whose dimensions depend on the method in use.
For instance, in \acrshort{pca}, \( A \) and \( B \)are \( p \times p \); in \acrshort{pls} and \acrshort{cca}, they are \( (p+q) \times (p+q) \).

% Table summarizing the definitions of A and B for various methods
\begin{table}[h]
    \centering
    \begin{tabular}{|c|c|c|c|c|}
        \hline
        Method         & \( A \)                                                                                    & \( B \)                                                                & \( x \)                                                  & Dimensions               \\
        \hline
       \acrshort{pca} & \( \Sigma_{11} \)                                                                          & \( I \)                                                                & \( u\sps{1} \)                                           & \( p \times p \)         \\
        \hline
        LDA            & \( S_B \)                                                                                  & \( S_W \)                                                              & \( u\sps{1} \)                                           & \( p \times p \)         \\
        \hline
       \acrshort{cca} & \( \begin{pmatrix} \Sigma_{11} & \Sigma_{12} \\ \Sigma_{21} & \Sigma_{22} \end{pmatrix} \) & \( \begin{pmatrix} \Sigma_{11} & 0 \\ 0 & \Sigma_{22} \end{pmatrix} \) & \( \begin{pmatrix} u\sps{1} \\ u\sps{2} \end{pmatrix} \) & \( (p+q) \times (p+q) \) \\
        \hline
       \acrshort{pls} & \( \begin{pmatrix} 0 & \Sigma_{12} \\ \Sigma_{21} & 0 \end{pmatrix} \)                     & \( I \)                                                                & \( \begin{pmatrix} u\sps{1} \\ u\sps{2} \end{pmatrix} \) & \( (p+q) \times (p+q) \) \\
        \hline
    \end{tabular}
    \caption{Definitions and dimensions of \( A \)and \( B \)for different subspace learning methods.}
    \label{tab:subspace}
\end{table}

To solve the GEP, one common technique is to transform it into a standard eigenvalue problem\( B^{-\frac{1}{2}} A B^{-\frac{1}{2}} y = \lambda y \), followed by eigendecomposition.
However, this approach has computational complexity \(O(n^3)\)and may suffer from numerical instability.

\subsubsection{\acrshort{pca}-CCA}

One way to reduce the complexity of solving GEPs is to use the \acrshort{pca}-CCA method, which first applies \acrshort{pca} to the data
and then solves the GEP in the reduced space.
An important advantage of using \acrshort{pca}-CCA is computational efficiency, especially for high-dimensional data.
The overall complexity of \acrshort{pca}-CCA is\(O(p^2K + p^3)\), where \( K \)is the number of reduced components.

\acrshort{kcca} is computationally efficient for high-dimensional data (\(p>n\)) because its complexity scales with the number of samples\(n\), not the number of features\(p\).
However, it requires access to all training data at test time, raising efficiency concerns.

\subsection{Non-linear \acrshort{cca}}

\subsubsection{Kernel \acrshort{cca}}

Kernel \acrshort{cca} (\acrshort{kcca}) is a non-linear extension of \acrshort{cca} that uses the kernel trick to find non-linear relationships between variables.

\begin{align}
     & \alpha_{\text{opt}}=\underset{\alpha_{\text{opt}}}{\mathrm{argmax}}\{ \alpha_1^{\top}K\spstop{1}K\sps{2}\alpha_2  \} \\
     & \text{subject to:} \notag                                                                                            \\
     & \alpha_1^{\top}K\spstop{1}K\sps{1}\alpha_1=1 \notag                                                                  \\
     & \alpha_2^{\top}K\spstop{2}K\sps{2}\alpha_2=1 \notag
\end{align}

\subsubsection{Deep \acrshort{cca}}

Deep \acrshort{cca} (\acrshort{dcca}) is a non-linear extension of \acrshort{cca} that uses deep neural networks to find non-linear relationships between variables.





