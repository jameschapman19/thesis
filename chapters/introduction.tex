\chapter{Introduction}\label{chap:introduction}
It was June 2021, and I had self-referred to the Community Living Well service in London, UK, seeking help for my mental health. Each week, I met with my therapist and dutifully filled out the questionnaires, rating my mood and answering questions about my well-being. Yet, I couldn't shake the feeling that these snapshots were inadequate in capturing the complexity of my mental state. As a keen sportsperson, I relied on my Garmin watch to track my heart rate, sleep, and activity levels, providing a continuous stream of biometric data that painted a more nuanced picture of my physical health. Moreover, as a type 1 diabetic, my continuous glucose monitor offered real-time insights into my blood sugar levels, helping me fine-tune my insulin management. These diverse data streams, each offering unique perspectives on my overall health, highlight the potential of learning meaningful representations from disparate data sources to gain a more comprehensive understanding of an individual's well-being.

In biomedical research, there is a growing need to develop methods that can effectively combine and analyze data from various sources, such as electronic health records, imaging data, and patient-reported outcomes. By leveraging the power of self-supervised learning, a machine learning approach that learns from unlabeled data, we can potentially uncover hidden patterns and relationships in these complex datasets. Self-supervised learning is particularly well-suited for this task, as it can learn robust and generalizable representations from vast amounts of unlabeled data, which is abundant in the biomedical domain.

This thesis focuses on developing and applying novel machine learning methods to address the challenge of integrating diverse health metrics through representation learning; that is, learning meaningful low-dimensional representations from complex, high-dimensional, and potentially multimodal data sources. 
A key approach explored in this work is Canonical Correlation Analysis (CCA), a powerful multiview learning technique that aims to find linear transformations of two or more datasets such that the transformed variables are maximally correlated. By learning these transformations, CCA can uncover latent structures and relationships between disparate data sources, making it a valuable tool for representation learning in the biomedical domain. Through improved methods for multiview and self-supervised learning, particularly centered around CCA, we hope to improve the analysis and comprehension of biomedical data, ultimately enhancing our ability to understand and manage personal health.

The main contributions of this thesis are fourfold:
\begin{enumerate}
\item Developing a framework for regularized Canonical Correlation Analysis (CCA) using structured priors to learn more interpretable and biologically meaningful representations;
\item Unifying simulated data generation methods for CCA under a latent variable model perspective to facilitate the evaluation and comparison of representation learning algorithms;
\item Formulating a new gradient descent approach for CCA and other generalized eigenvalue problems, tailored for learning representations from large datasets;
\item Extending the gradient descent approach to Deep CCA and Joint Embedding Self-Supervised Learning to learn more complex representations from complex, high-dimensional data;
\item Developing CCA-Zoo, an open-source Python package for Canonical Correlation Analysis, which provides a unified interface for various CCA methods and facilitates their application in representation learning.
\end{enumerate}
These contributions have significant practical implications, from aiding in the diagnosis and treatment of mental health and neurological disorders to enabling efficient analysis of extensive health databases like the UK Biobank \citep{biobank2014uk}.

\section*{Thesis Structure}

The thesis is structured as follows:
\begin{itemize}
\item \textbf{Chapter \ref{ch:background}} reviews multiview and self-supervised learning techniques, focusing on their application in learning meaningful representations from biomedical data.
\item \textbf{Chapter \ref{ch:als}} introduces a method to regularize CCA using structured priors, demonstrated with Human Connectome Project and Alzheimer's Disease Neuroimaging Initiative data to showcase the potential of learning structured representations.
\item \textbf{Chapter \ref{ch:loadings}} examines the relationship between loadings and weights in CCA, using simulated data to show the advantages of loadings for interpreting learned representations.
\item \textbf{Chapter \ref{ch:gradient_descent}} presents a new gradient descent algorithm for generalized eigenvalue problems, tailored for learning representations from large datasets, demonstrated with Multiview CCA and PLS. We show how our algorithm can be applied to large datasets, using the UK Biobank as an example.
\item \textbf{Chapter \ref{ch:deep_learning}} extends the algorithm from Chapter \ref{ch:gradient_descent} to deep learning, showing its application in scaling deep CCA to learn hierarchical representations from complex, high-dimensional data. We demonstrate state-of-the-art results on CIFAR-10 and CIFAR-100 benchmarks, illustrating the potential of Deep CCA in Self-Supervised Learning.
\item \textbf{Chapter \ref{ch:software}} introduces CCA-Zoo, a Python package implementing the methodologies of this thesis, and discusses its role in the Python ecosystem and biomedical research, particularly in facilitating representation learning.
\item \textbf{Chapter \ref{ch:discussion}} discusses the implications, challenges, and future directions for the research presented in this thesis.
\end{itemize}

Through this thesis, we aspire to bridge the gap between the potential of biomedical data and the current capabilities of analytical methods. By developing novel, scalable, and interpretable machine learning approaches for representation learning, we aim to unlock the full potential of diverse health metrics, paving the way for advancements in biomedical research and personalized healthcare.