\chapter{Conclusion}\label{ch:discussion}

This chapter provides a summary of the findings of this thesis, discusses their implications, and outlines potential directions for future work. We reflect on the contributions of this thesis to the field of canonical correlation analysis (CCA) and multiview learning, highlighting the opportunities and challenges that lie ahead. By examining the impact of our work on both theory and practice, we aim to inspire further research in this exciting and rapidly evolving field.

\section{Summary of Findings}



\section{Implications}

\section{Future Work}

\subsection{Applications}

One of the most rewarding aspects of this thesis has been seeing the software we developed, such as the CCA-Zoo package, being used by researchers across different fields to tackle a wide range of problems. It has been truly inspiring to witness the impact of our work and to see how it has enabled others to push the boundaries of multiview learning.

While the applications presented in this thesis, particularly the UK Biobank analysis in Chapter \ref{ch:deep_learning}, have demonstrated the potential of our methods, there is still vast untapped potential in applying these techniques to even larger and more diverse datasets. The ABCD dataset, for instance, offers a rich source of multimodal data that could benefit from the regularized and scalable CCA methods developed in this thesis. Preliminary results on this dataset have shown promise, and we believe that further exploration will yield valuable insights into brain development and its associated factors.

Moreover, the integration of non-imaging modalities, such as Electronic Health Records and audio transcripts, with neuroimaging data presents an exciting avenue for future research. The success of transformer architectures in natural language processing suggests that these methods could be powerful tools for analyzing and integrating textual and auditory data with brain imaging. By leveraging the capabilities of deep learning and the multiview learning framework of CCA, we can uncover complex relationships across modalities and gain a more comprehensive understanding of human health and behavior.

\subsection{Methods}

From a methodological perspective, there are several directions in which the work presented in this thesis could be extended. One promising avenue is the further development of regularization techniques for CCA, building upon the ideas introduced in Chapters \ref{ch:als} and \ref{ch:loadings}. Exploring alternative priors and regularization schemes could lead to more interpretable and robust CCA models, particularly in high-dimensional settings.

Another area of interest is the adaptation of the scalable algorithms presented in Chapters \ref{ch:gradient_descent} and \ref{ch:deep_learning} to other multiview learning methods beyond CCA. Techniques such as Partial Least Squares (PLS) and multi-view clustering could potentially benefit from the efficient optimization and deep learning approaches developed in this thesis. Extending these methods to a broader range of multiview learning problems could have a significant impact on various fields, from bioinformatics to social network analysis.

\section{Closing Remarks}

As I reflect on my journey from struggling with the concepts of eigenvalues and eigenvectors as an undergraduate to delving into the intricacies of CCA as a doctoral researcher, I am struck by the depth and breadth of this field. Far from being a narrow topic, CCA has served as a gateway to explore a wide range of subjects, from optimization and Bayesian statistics to deep learning and software engineering.

One of the most exciting aspects of this journey has been witnessing the rapid growth and evolution of multiview and self-supervised learning during my PhD. The convergence of these fields with CCA has been truly remarkable, and I am thrilled to have been a part of this exciting development. The methods and insights gained from this thesis have the potential to contribute to the continued growth and impact of these areas.

Although we have been limited by access to data, I am confident that the work presented in this thesis has made a meaningful impact on the field of multiview learning. The software tools we have developed, such as CCA-Zoo, have already been adopted by researchers across various domains, enabling them to push the boundaries of what is possible with CCA and related methods.

With its rich history dating back to the 1930s, CCA has been a powerful tool for discovering new associations across various domains. The work presented in this thesis aims to contribute to the continued relevance and applicability of CCA in the era of big data and deep learning. By developing interpretable, regularized, and scalable methods, we have sought to unlock the full potential of CCA and multiview learning.

As we look to the future, I am excited by the prospects of further advancements in this field. The integration of deep learning with CCA, the application of these methods to ever-larger and more diverse datasets, and the potential for uncovering groundbreaking insights all point to a bright future for multiview learning. I hope that this thesis has not only made valuable contributions to the field but also inspired others to explore the fascinating world of CCA and its applications.

Thank you for reading.