\graphicspath{{chapters/introduction/}}
\chapter{Introduction}\label{chap:introduction}

\epigraph{We are drowning in information, while starving for wisdom. The world henceforth will be run by synthesizers, people able to put together the right information at the right time, think critically about it, and make important choices wisely.}{\textit{E. O. Wilson}}

In an era marked by an overwhelming abundance of data, the ability to distill wisdom from information becomes not just a skill but a necessity. This thesis addresses the critical gap between the immense volumes of biomedical data available and our current capacity to synthesize this data into actionable insights for healthcare.

\section{Setting the Stage: The Challenge of Big Data in Biomedical Research}

The advent of advanced technologies in biomedical research has led to a deluge of data.
From genomic sequences to patient health records, this data holds the key to unlocking mysteries of human health and disease.
However, the challenge lies not in data collection but in its analysis and interpretation – in turning this vast information into wisdom.

\textbf{This thesis presents innovative methodologies for scaling multimodal data fusion to massive datasets}, aiming to transform the way biomedical data is analyzed and understood.
By leveraging advancements in self-supervised learning and data integration techniques, it seeks to bridge the gap between data abundance and knowledge creation.

\section{Thesis Structure and Contributions}

This thesis offers three primary contributions:

\begin{itemize}
    \item Developing a regularization method for CCA using structured priors, including the Elastic Net, to improve interpretability.
    \item Proposing the use of loadings over weights in CCA for better interpretability and relevance to biomedical data generation processes.
    \item Creating a new gradient descent-based formulation for CCA and generalized eigenvalue problems, suitable for large datasets.
\end{itemize}

\subsection{Chapter Summaries}

\textbf{Chapter \ref{chap:background}} reviews multiview and self-supervised learning techniques, focusing on their application in biomedical data.

\textbf{Chapter \ref{chap:als}} introduces a method to regularize CCA using structured priors, demonstrated with Human Connectome Project and Alzheimer's Disease Neuroimaging Initiative data.

\textbf{Chapter \ref{chap:loadings}} examines the relationship between loadings and weights in CCA, using simulated data to show the advantages of loadings for interpretability.

\textbf{Chapter \ref{chap:gradient_descent}} presents a new loss function for generalized eigenvalue problems, applicable to CCA and SSL methods, and demonstrates its effectiveness with various benchmarks.

\textbf{Chapter \ref{chap:software}} introduces CCA-Zoo, a Python package implementing the methodologies of this thesis, and discusses its role in the Python ecosystem and biomedical research.

\textbf{Chapter 7} discusses the implications, challenges, and future directions for the research presented in this thesis.

This thesis aims to address the gap between the potential of biomedical data and the capabilities of current analytical methods.
