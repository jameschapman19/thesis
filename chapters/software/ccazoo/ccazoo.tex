\section{Python: CCA-Zoo: A collection of regularised, Deep Learning based, Kernel, and Probabilistic CCA methods in a scikit-learn style framework}\label{sec:ccazoo}

\subsection{Statement of need}
The Python ecosystem for multiview learning currently provides a few options for implementing CCA and PLS models. scikit-learn \cite{pedregosa2011scikit} contains standard implementations of both CCA and PLS for two-view data which plug into their mature API. pyrcca \cite{bilenko2016pyrcca} contains implementations of ridge regularised and kernelized two-view CCA. The embed module of mvlearn \cite{perry2020mvlearn} is perhaps the closest relative of cca-zoo, containing implementations of ridge regularised and kernelized multi-view CCA. cca-zoo builds on the mvlearn API by providing an additional range of regularised models and in particular sparsity inducing models which have found success in multiomics. Building on the reference implementation in mvlearn, cca-zoo further provides a number of deep learning models with a modular design to enable users to supply their own choice of neural network architectures.

Standard implementations of state-of-the-art models help as benchmarks for methods development and easy application to new datasets. cca-zoo extends the existing ecosystem with a number of sparse regularised CCA models. These variations have found popularity in genetics and neuroimaging where signals are contained in a small subset of variables. With applications like these in mind, cca-zoo simplified the access to the learnt model weights to perform further analysis in the feature space. Furthermore, the modular implementations of deep CCA and its multiview variants allow the user to focus on architecture tuning. Finally, cca-zoo adds generative models including variational \cite{wang2007variational} and deep variational CCA \cite{wang2016deep} as well as higher order canonical correlation analysis with tensor \cite{kim2007tensor} and deep tensor CCA \cite{wong2021deep}.

\subsection{Implementation}
cca-zoo adopted a similar API to that used in scikit-learn. The user first instantiates a model object and its relevant hyperparameters. Next they call the model's fit() method to apply the data. After fitting, the model object contains its relevant parameters such as weights or dual coefficients (for kernel methods) which can be accessed for further analysis. For models that fit with iterative algorithms, the model may also contain information about the convergence of the objective function. After the model has been fit, its transform() method can project views into latent variables and score() can be used to measure the canonical correlations.

The deep and probabilistic models are supported by PyTorch and NumPyro respectively. Due to the size of these dependencies, these two classes of variations are not in the default installation. Instead, we provide options [deep] and [probabilistic] for users. The list bellow provides the complete collection of models along with their installation tag is provided below.

\subsection{Model List}

A complete model list at the time of publication:

\begin{center}
\begin{tabular}{|| p{0.2\textwidth}|p{0.4\textwidth}|p{0.1\textwidth}|p{0.1\textwidth} ||}
    \hline
    Model Class & Model Name & Number of Views & Install\\
    \hline \hline
    CCA & Canonical Correlation Analysis & 2 & standard\\
    \hline
    rCCA & Canonical Ridge & 2 & standard\\
    \hline
    KCCA & Kernel Canonical Correlation Analysis & 2 & standard\\
    \hline
    MCCA & Multiset Canonical Correlation Analysis & >=2 & standard\\
    \hline
    KMCCA & Kernel Multiset Canonical Correlation Analysis & >=2 & standard\\
    \hline
    GCCA & Generalized Canonical Correlation Analysis & >=2 & standard\\
    \hline
    KGCCA & Kernel Generalized Canonical Correlation Analysis & >=2 & standard\\
    \hline
    PLS & Partial Least Squares & >=2 & standard\\
    \hline
    CCA\_ALS & Canonical Correlation Analysis by Alternating Least Squares) \cite{golub1995canonical} & >=2 & standard\\
    \hline
    PLS\_ALS & Partial Least Squares by Alternating Least Squares) & >=2 & standard\\
    \hline
    PMD & Sparse CCA by Penalized Matrix Decomposition & >=2 & standard\\
    \hline
    ElasticCCA & Sparse Penalized CCA \cite{waaijenborg2008quantifying} & >=2 & standard\\
    \hline
    ParkhomenkoCCA & Sparse CCA \cite{parkhomenko2009sparse} & >=2 & standard\\
    \hline
    SCCA & Sparse Canonical Correlation Analysis by Iterative Least Squares \cite{mai2019iterative} & >=2 & standard\\
    \hline
    SCCA\_ADMM & Sparse Canonical Correlation Analysis by Altnerating Direction Method of Multipliers \cite{suo2017sparse} & >=2 & standard\\
    \hline
    SpanCCA & Sparse Diagonal Canonical Correlation Analysis \cite{asteris2016simple} & >=2 & standard\\
    \hline
    SWCCA & Sparse Weighted Canonical Correlation Analysis \cite{wenwen2018sparse} & >=2 & standard\\
    \hline
    TCCA & Tensor Canonical Correlation Analysis & >=2 & standard\\
    \hline
    KTCCA & Kernel Tensor Canonical Correlation Analysis \cite{kim2007tensor} & >=2 & standard\\
    \hline
    DCCA & Deep Canonical Correlation Analysis & >=2 & deep\\
    \hline
    DCCA\_NOI & Deep Canonical Correlation Analysis by Non-Linear Orthogonal Iterations \cite{wang2015stochastic} & >=2 & deep\\
    \hline
    DCCAE & Deep Canonically Correlated Autoencoders \cite{wang2015deep} & >=2 & deep\\
    \hline
    DTCCA & Deep Tensor Canonical Correlation Analysis & >=2 & deep\\
    \hline
    SplitAE & Split Autoencoders \cite{ngiam2011multimodal} & 2 & deep\\
    \hline
    DVCCA & Deep Variational Canonical Correlation Analysis & >=2 & deep\\
    \hline
    ProbabilisticCCA & Probabilistic Canonical Correlation Analysis & 2 & probabilistic\\
    \hline
\end{tabular}
\end{center}


\subsection{Documentation}

The package is accompanied by documentation (https://cca-zoo.readthedocs.io/en/latest/index.html) and a number of tutorial notebooks which serve as both guides to the package as well as educational resources for CCA and PLS methods.

\subsection{Conclusion}

cca-zoo fills many of the gaps in the multiview learning ecosystem in Python, including a flexible API for deep-learning based models, regularised models for high dimensional data (and in particular those that induce sparsity), and generative models.cca-zoo will therefore help researchers to apply and develop Canonical Correlation Analysis and Partial Least Squares models. We continue to welcome contributions from the community.

\subsection{Contribution}

I was the first author of two.