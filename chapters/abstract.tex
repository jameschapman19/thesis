\chapter*{Abstract} % the Asterix (*) indicates that this section will be added to the table of contents but no number will be present beside it.
% \addcontentsline{toc}{chapter}{Abstract}

Biomedical data are critical for enhancing our understanding and practices in medicine and healthcare. Yet, the complexity, heterogeneity, high-dimensionality, and label scarcity in these datasets present significant analytical challenges. This thesis introduces innovative approaches to self-supervised learning (SSL), focusing on multiview SSL, where data are represented by multiple distinct feature groups or modalities. To overcome these challenges, self-supervised learning (SSL) has emerged as a promising
paradigm for learning from unlabeled data by leveraging inherent structures or patterns in the data. SSL methods can exploit different forms of supervision signals
derived from the data itself, such as contrastive learning, reconstruction, prediction,
or clustering. SSL methods can also benefit from deep neural networks that can
learn expressive and flexible representations from complex and high-dimensional
data

Central to this thesis are four research questions addressing the enhancement of multiview SSL: (1) How can regularization or prior knowledge be integrated into subspace learning methods for improved quality and robustness? (2) How can the data generation process aid in interpreting multiview models and validating their quality? (3) How can subspace learning methods be scaled up to large datasets using gradient-based optimization techniques? (4) How can these methods be extended to nonlinear functions with deep neural networks?

Our contributions include:

A framework for incorporating various forms of regularization or prior knowledge into subspace learning, enhancing the quality and robustness of these methods. A unification of simulated data generation literature for multiview learning, facilitating model interpretation and quality validation. A scalable and flexible subspace learning method for multiview SSL, adaptable to large-scale datasets through modern optimization techniques. An innovative extension of subspace learning to nonlinear functions using deep neural networks. A high quality open source software implementation of the canonical correlation analysis family of methods, enabling reproducible research and facilitating adoption by the community.

This research advances the field of biomedical data analysis by providing scalable, flexible, and interpretable solutions for multiview SSL challenges, harnessing the power of modern computational techniques and deep learning and making them accessible through open source software.