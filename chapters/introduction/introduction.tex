\graphicspath{{chapters/introduction/}}
\chapter{Introduction}\label{chap:introduction}

In the middle of my PhD journey, in June 2021, I self-referred to the Community Living Well service in London, UK, for help with my mental health. I was assigned a therapist, who I met with weekly for 12 weeks. During our sessions, we discussed my mental health and the challenges I was facing. I was also asked to complete a questionnaire at the beginning and end of each session, which asked me to rate my mood and answer questions about my mental health. Each time I did this, I questioned how well these subjective numbers truly represented my feelings.

A keen sportsperson, I also wear a Garmin watch that tracks my heart rate, my sleep, and my activity levels. I use this data to monitor my health and fitness, and I have found it to be a useful tool in my training. Using a physical `stress level' metric based on Heart Rate Variability (HRV), I can see how alcohol affects my sleep\footnote{badly}, how well I have slept, and I know I am about to get sick before I feel it.

As a type 1 diabetic, I rely on a continuous glucose monitor for my blood sugar levels. This device measures my blood sugar every five minutes, and I can see the results on my phone. I can also see how my blood sugar changes over time, and I can use this information to adjust my insulin doses and improve my control.

This thesis stems from a simple idea: What if we could combine different kinds of health data to get a clearer, more complete picture of a person's health?

\section{Thesis Structure and Contributions}

This thesis presents innovative methodologies for scaling multiview data fusion to massive datasets, aiming to transform the way biomedical data is analyzed and understood. By leveraging advancements in self-supervised learning and multiview learning, the research herein explores the integration of diverse data sources, similar to how my mental health, physical activity, and diabetes data each provide unique insights into my well-being.

The overarching aim is to develop methodological improvements that are practical and user-friendly. We strive to create tools and methods that are theoretically robust yet intuitive and straightforward to use in real-life scenarios. The goal is to empower practitioners in biomedical research and other fields to fully leverage their data, without requiring deep technical expertise in data analysis algorithms.

This thesis offers three primary contributions:

\begin{itemize}
    \item Developing a regularization method for CCA using structured priors, including the Elastic Net, to improve interpretability.
    \item Proposing the use of loadings over weights in CCA for better interpretability and relevance to biomedical data generation processes.
    \item Creating a new gradient descent-based formulation for CCA and generalized eigenvalue problems, suitable for large datasets.
\end{itemize}

\subsection{Chapter Summaries}

\textbf{Chapter \ref{chap:background}} reviews multiview and self-supervised learning techniques, focusing on their application in biomedical data.

\textbf{Chapter \ref{chap:als}} introduces a method to regularize CCA using structured priors, demonstrated with Human Connectome Project and Alzheimer's Disease Neuroimaging Initiative data.

\textbf{Chapter \ref{chap:loadings}} examines the relationship between loadings and weights in CCA, using simulated data to show the advantages of loadings for interpretability.

\textbf{Chapter \ref{chap:gradient_descent}} presents a new loss function for generalized eigenvalue problems, applicable to CCA and SSL methods, and demonstrates its effectiveness with various benchmarks.

\textbf{Chapter \ref{chap:software}} introduces CCA-Zoo, a Python package implementing the methodologies of this thesis, and discusses its role in the Python ecosystem and biomedical research.

\textbf{Chapter 7} discusses the implications, challenges, and future directions for the research presented in this thesis.

I hope that this thesis and the work it represents will help to bridge the gap between the potential of biomedical data and the capabilities of current analytical methods.
