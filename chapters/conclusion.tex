\chapter{Conclusion}

In this thesis, we have motivated the need for developing scalable, flexible, and interpretable multiview SSL methods for biomedical data, especially for population studies of mental health. We have highlighted the challenges and opportunities of working with large-scale and multimodal data sources that can provide diverse and comprehensive information on mental health status and related factors. We had three research questions and objectives that guided our work:

\begin{itemize} 
\item How can we scale classical multiview SSL to the increasingly massive biomedical data available to researchers 
\item How can we incorporate expert priors and interpretability into the models
\item How  can we benefit from the power of deep learning in this space
\end{itemize}

In Chapter 2, we reviewed the existing literature on multiview learning and SSL, and identified the gaps and limitations of the current methods. We also introduced the main concepts and techniques that are relevant for our work, such as subspace learning, regularization, deep neural networks, and stochastic optimization.

In Chapter 3, we reformulated the classical CCA method as an unconstrained optimization problem that can be solved by SGD and therefore scales to large datasets

In Chapter 4, we added regularisation and showed that it improved generalisation (priors) and could be more easily intepreted (sparsity).

In chapter 5 we extended the model to include deep learning functions for learning oomplex highly flexible  functions

