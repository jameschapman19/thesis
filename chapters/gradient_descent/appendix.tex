% \graphicspath{{chapters/gradient_descent/}}
% \chapter{Proofs and Additional Results for Chapter \ref{ch:gradient_descent}}\label{app:gradient_descent}

% The work shown in this appendix is contributed to my co-author, Lennie Wells, and is printed here in order to give context to the work in Chapter \ref{ch:gradient_descent}.

% \section{Eckhart-Young characterization of GEP subspace}\label{supp:proofs}
% \subsection{Formal definitions}
% There are various different notations and conventions for GEPs and SVDs.
% We largely follow the standard texts on Matrix Analysis \citep{stewart_matrix_1990,bhatiamatrix1997} but seek a more careful handling of the equality cases of certain results.
% To help, we use the following non-standard definitions, largely inspired by \citet{carlssonvon2021}.

% \begin{definition}[Top-$K$ subspace]
%     Let the GEP $(A,B)$ on $\R^d$ have eigenvalues $\lambda_1 \geq \dots \geq \lambda_d$. Then \textbf{a} top-$K$ subspace is that spanned by some $w_1,\dots,w_K$, where $w_k$ is a $\lambda_i$-eigenvector of $(A,B)$ for $k=1,\dots,K$.
% \end{definition}

% \begin{definition}[$B$-orthonormality]
%     Let $B \in \R^{d \times d}$ be strictly positive definite. Then we say a collection $w_1,\dots,w_K \in \R^{d}$ of vectors is $B$-orthonormal if $w_k^T B w_l = \delta_{kl}$ for each $k,l \in \{1,\dots,K\}$.
% \end{definition}

% \begin{definition}[Top-$K$ matrix]
%     We say $W \in \R^{d \times K}$ is a top-$K$ matrix for a GEP $(A,B)$ if the $k^{\text{th}}$ column $w_i$ of $W$ is a $\lambda_k$-eigenvector for each $k$ and the columns are $B$-orthonormal.
% \end{definition}


% \subsection{Standard Eckhart--Young inequality}

% \begin{theorem}[Eckhart--Young]\label{thm:eckhart-young}
% Let $M \in \R^{p\times q}$. Then $\hat{M}$ minimises $\|M-\tilde{M}\|_F$ over matrices $\tilde{M}$ of rank at most $K$ if and only if $\hat{M} = A_K R_K B_K^\top$ where $(A_K,R_K,B_K)$ is some top-$K$ SVD of the target $M$.
% \end{theorem}
% \begin{proof}
%     Let $M,\tilde{M}$ have singular values $\sigma_k,\tilde{\sigma}_k$ respectively.
%     Since $\tilde{M}$ has rank at most $K$ we must have $\tilde \sigma_k = 0 \text{ for } k> K$.

%     Then by von Neumann's trace inequality \citep{carlssonvon2021},
%     \begin{equation*}
%         \langle M, \tilde{M} \rangle_F \leq \sum_{k=1}^K \sigma_k \tilde{\sigma}_k
%     \end{equation*}
%     with equality if and only if $M,\tilde{M}$ `share singular vectors'; the notion of sharing singular vectors is defined as in \citet{carlssonvon2021} and in this case means that $\tilde{M} = A_K \tilde{R}_K B_K$ where $(A_K,R_K,B_K)$ is some top-$K$ SVD of $M$ and $\tilde{R}_K$ is a diagonal matrix with decreasing diagonal elements $\tilde{\sigma}_1\geq \dots \geq \tilde{\sigma}_K$.

%     Expanding out the objective and applying this inequality gives
%     \begin{align*}
%         \norm{\tilde{M} - M}_F^2
%         &\geq \sum_{k=1}^d \sigma_k^2 - 2 \sum_{k=1}^K \sigma_k \tilde{\sigma}_k + \sum_{k=1}^K \tilde\sigma_k^2 \\
%         &= \sum_{k=K+1}^d \sigma_k^2 + \sum_{k=1}^K (\sigma_k - \tilde{\sigma}_k)^2 \\
%         &\geq \sum_{k=K+1}^d \sigma_k^2
%     \end{align*}
%     so indeed to have equality in both cases requires $\sigma_k = \tilde{\sigma}_k$ for each $k\leq K$ so indeed $\tilde{R}_K = R_K$ and so $\hat{M}$, as defined in the statement of the theorem, minimises $\|M-\tilde M\|_F$ over matrices $\tilde M$ of rank at most $K$.
% \end{proof}


% \subsection{Supporting Results}
% \begin{lemma}[Matrix square root lemma]\label{lem:square-root-lemma}
% Suppose we have two full rank matrices $E,F \in \R^{d \times K}$ where $K \leq d$ and such that $E E^T = F F^T$; then there exists an orthogonal matrix $O \in \R^{K \times K}$ with $E = FO$.
% \end{lemma}
% \begin{proof}
%     Post multiplying the defining condition gives $E E^T E = F F^T E$.
%     Then right multiplying by $(E^T E)^{-1}$ gives
%     \begin{align*}
%         E = F F^T E (E^T E)^{-1} \eqdef F O
%     \end{align*}
%     to check that $O$ as defined above is orthogonal we again use the defining condition to compute
%     \begin{align*}
%         O^T O = (E^T E)^{-1} E^T F F^T E (E^T E)^{-1} = (E^T E)^{-1} E^T E E^T E (E^T E)^{-1} = I_K
%     \end{align*}
% \end{proof}

% \begin{corollary}[PSD Eckhart--Young for square root matrix]\label{prop:psd-eckhart-young-factor}
% Let $M \in \R^{d\times d}$ be symmetric positive semidefinite.
% Then
% \begin{align*}
%     \argmin_{\tilde{Z} \in \R^{d \times K}} \| M - \tilde{Z} \tilde{Z}^T \|_F^2
% \end{align*}
% is precisely the set of $\tilde{Z}$ of the form $\tilde{Z} = Z_K \Lambda_K^\half O_K$ for some top-$K$ eigenvector-matrix $Z_K$ of the GEP $(M,I)$ and some orthogonal $O_K \in \mathcal{O}(K)$, and where $\Lambda_K$ is a diagonal matrix of the top-$K$ eigenvalues.
% \end{corollary}
% \begin{proof}
%     First note that when $M$ is positive semi-definite the SVD coincides with the eigendecomposition.

%     Second note that taking $\tilde{Z} = Z_K \Lambda_K^\half O_K$ attains the minimal value by the Eckhart--Young inequality, Theorem \ref{thm:eckhart-young}.

%     Next note that if $\tilde{Z}$ attains the minimal value then it must have $\tilde{Z} \tilde{Z}^T = Z_K \Lambda_K Z_K^T$ by the equality case of Eckhart--Young.
%     Then by matrix square root Lemma \ref{lem:square-root-lemma} we must indeed have $\tilde{Z} = Z_K \Lambda_K^\half O_K$ for some orthogonal $O_K$.
% \end{proof}

% \begin{corollary}[Symmetric Eckhart--Young for square root matrix]\label{cor:sym-eckhart-young-factor}
% Let $M \in \R^{d\times d}$ be symmetric with eigenvalues $\lambda_1\geq \dots \geq \lambda_d$ such that $\lambda_K > 0$.
% Then
% \begin{align*}
%     \argmin_{\tilde{Z} \in \R^{d \times K}} \| M - \tilde{Z} \tilde{Z}^T \|_F^2
% \end{align*}
% is precisely the set of $\tilde{Z}$ of the form $\tilde{Z} = Z_K \Lambda_K^\half O_K$ for some top-$K$ eigenvector-matrix $Z_K$ of the GEP $(M,I)$ and some orthogonal $O_K \in \mathcal{O}(K)$, and where $\Lambda_K$ is a diagonal matrix of the top-$K$ eigenvalues.
% \end{corollary}
% \begin{proof}
%     Let $\tilde{Z} \in \R^{d\times K}$.
%     Because $M$ is symmetric it has some eigen-decomposition; separate this into strictly positive and non-positive eigenvalues $M = M_+ + M_- = Z_+ \Lambda_+ Z_+^T + Z_- \Lambda_- Z_-^T$, with rank $d_+,d_-$ respectively.
%     % In the above we separate > 0 from \leq 0 so that have P_+ + P_- = Id, which makes notation cleaner
%     Let the corresponding projections be $P_+ = Z_+ Z_+^T, P_- = Z_- Z_-^T$.

%     Now define $\tilde{Z}_+ = P_+ \tilde{Z}, \tilde{Z}_- = P_- \tilde{Z}$. Then note by orthogonality of the projections we have for any matrix $A$ that
%     \begin{align*}
%         \norm{A}^2 = \norm{(P_+ + P_-) A (P_+ + P_-)}^2 = \norm{P_+ A P_+}^2 + \norm{P_+ A P_-}^2 + \norm{P_- A P_+}^2 + \norm{P_- A P_-}^2
%     \end{align*}

%     So we can expand out
%     \begin{align}
%         \norm{M - \tilde{Z}\tilde{Z}^T}^2
%         &= \norm{(P_+ + P_-) (M - \tilde{Z}\tilde{Z}^T) (P_+ + P_-)}^2 \nonumber \\
%         &= \underbrace{\norm{M_+ - \tilde{Z}_+ \tilde{Z}_+^T}^2}_{\geq \sum_{k=K+1}^{d_+} \lambda_k^2} + \underbrace{\norm{M_- - \tilde{Z}_- \tilde{Z}_-^T}^2}_{\geq \norm{M_-}^2}
%         + \underbrace{\norm{\tilde{Z}_+ \tilde{Z}_-^T}^2}_{\geq 0} + \underbrace{\norm{\tilde{Z}_- \tilde{Z}_+^T}^2}_{\geq 0}
%         \geq \sum_{k=K+1}^d \lambda_k^2
%         \label{eq:chasing-equality}
%     \end{align}
%     where the first inequality follows from the previous Corollary \ref{prop:psd-eckhart-young-factor} and the second inequality is just from
%     \begin{align*}
%         \norm{M_- - \tilde{Z}_- \tilde{Z}_-^T}^2 - \norm{M_-}^2 = - 2 \tr\left( \tilde{Z}_-^T M_- \tilde{Z}_- \right)  + \norm{\tilde{Z}_- \tilde{Z}_-^T}^2 \geq 0
%     \end{align*}
%     because $M_-$ has negative eigenvalues.

%     Moreover equality in (\ref{eq:chasing-equality}) requires the equality case of all the component inequalities; the first gives $\tilde{Z}_+ = Z_K \Lambda_K^\half O_K$ for some $Z_K,O_K$ as in the statement of Corollary \ref{prop:psd-eckhart-young-factor}, and the second that $\tilde{Z}_- = 0$; so indeed combining $\tilde{Z} = \tilde{Z}_+ + \tilde{Z}_-$ gives the result.
% \end{proof}

% \subsection{GEP-EY Objective}
% \begin{proposition}[GEP-EY-Objective]
%     Consider the GEP $(A,B)$ with $A$ symmetric and $B$ positive definite; suppose there are at least $K$ strictly positive (generalized) eigenvalues. Then:
%     \begin{align*}
%         \tilW \in \argmax_{\tilW \in \R^{d\times k}} \: \tr\left\{2 \left(\tilW^T A \tilW\right) - \left(\tilW^T B \tilW\right) \left(\tilW^T B \tilW\right) \right\}
%     \end{align*}
%     if and only if $\tilW = W_K \Lambda_K^\half O_K$ for some top-$K$ matrix $W_K$ of the GEP and some orthogonal $O_K \in \mathcal{O}(k)$, where $\Lambda_K$ is a diagonal matrix of the top-$K$ eigenvalues.

%     Moreover, the maximum value is precisely $\sum_{k=1}^K \lambda_k^2$.
% \end{proposition}
% \begin{proof}
%     First recall that there is a bijection between eigenvectors $w$ for the GEP $(A,B)$ and eigenvectors $z = B^\half w$ for the GEP $(M,I)$ where $M \defeq B^\mhalf A B^\mhalf$ (e.g. see \citet{chapman2022generalized}).

%     Now consider how the Eckhart--Young objective from Corollary \ref{cor:sym-eckhart-young-factor} transforms under the bijection $Z = B^\half W$.

%     We get
%     \begin{align*}
%         \| M - \tilde{Z} \tilde{Z} \|_F^2
%         &= \| B^\mhalf A B^\mhalf - B^\half \tilde{W} \tilde{W}^T B^\half \|_F^2 \\
%         &= \| B^\mhalf A B^\mhalf \|_F^2 - 2\tr\left( B^\mhalf A B^\mhalf B^\half \tilde{W} \tilde{W}^T B^\half \right) \\
%         &\quad + \tr\left(B^\half \tilde{W} \tilde{W}^T B^\half \: B^\half \tilde{W} \tilde{W}^T B^\half \right) \\
%         &= \| B^\mhalf A B^\mhalf \|_F^2 - \tr\left\{2 \left(\tilW^T A \tilW\right) - \left(\tilW^T B \tilW\right) \left(\tilW^T B \tilW\right) \right\},
%     \end{align*}

%     where the first term is independent of $\tilde{W}$, so we can conclude by Corollary \ref{cor:sym-eckhart-young-factor}.

%     The moreover conclusion can follow from computing the objective at any maximiser of the form above. We note that
%     \begin{align*}
%         \tilW^T A \tilW &= O_K^T \Lambda_K^\half W_K^T A W_K \Lambda_K O_K = O_K^T \Lambda_K^2 O_K \\
%         \tilW^T B \tilW &= O_K^T \Lambda_K^\half W_K^T B W_K \Lambda_K O_K = O_K^T \Lambda_K O_K
%     \end{align*}
%     plugging into the objective gives
%     \begin{align*}
%         \tr \left(2 \: (\tilW^T A \tilW) - (\tilW^T B \tilW)^2\right) = \tr \left(2 \,O_K^T \Lambda_K^2 O_K - O_K^T \Lambda_K^2 O_K\right)
%         %=\tr \left(O_K^T \Lambda_K^2 O_K\right)
%         = \sum_{k=1}^K \lambda_k^2
%     \end{align*}
%     because the trace of a symmetric matrix is equal to the sum of its eigenvalues.
% \end{proof}


% \section{Tractable Optimization - no spurious local minima}\label{supp:tractable-optimization}\label{app:tractable-optimization}
% First in \cref{sec:no-spurious-local-minima-qualitative} we prove that for general $A, B$ our loss $\LEY(U)$ has no spurious local minima.
% Then in \cref{sec:no-spurious-local-minima-quantitative} we apply a result from \citet{ge_no_2017}.
% This application is somewhat crude, and we expect that a quantitative result with tighter constants could be obtained by adapting the argument of \cref{sec:no-spurious-local-minima-qualitative}; we leave such analysis to future work.

% \subsection{Qualitative results}\label{sec:no-spurious-local-minima-qualitative}
% First we prove an auxillary result.
% \begin{lemma}\label{lem:least-squares-projected-signal}
% Let $M \in \R^{D \times D}$ be a symmetric matrix and let $U \in \R^{D \times K}$.
% Let
% \begin{align*}
%     \hat{\Gamma} \defeq \argmin_{\Gamma \in \R^{K \times K}} \norm{M - U \Gamma U^T}_F^2
% \end{align*}
% Then $U \hat{\Gamma} U^T = \mathcal{P}_U M \mathcal{P}_U$ and the minimum value is precisely
% \begin{align}\label{eq:minimum-attained-low-rank-approximation}
% \norm{M}_F^2 - \norm{\mathcal{P}_U M \mathcal{P}_U}_F^2
% \end{align}

% Moreover, if $U$ has orthonormal columns then $\hat{\Gamma} = U^T M U$, and $\norm{\mathcal{P}_U M \mathcal{P}_U}_F^2 = \norm{\hat{\Gamma}}_F^2$
% \end{lemma}
% \begin{proof}
%     Simply complete the square to give
%     \begin{align*}
%         \norm{M - U \Gamma U^T}_F^2
%         &= \tr (U^T U) \Gamma^T (U^T U) \Gamma - 2 \tr D (U^T M U) + \norm{M}_F^2 \\
%         &= \norm{(U^T U)^\half \Gamma (U^T U)^\half - (U^T U)^\mhalf (U^T M U) (U^T U)^\mhalf}_F^2 + \norm{M}_F^2 - \norm{\mathcal{P}_U M \mathcal{P}_U}_F^2
%     \end{align*}
%     from which we can read off that the minimum is attained precisely when
%     \begin{align*}
%         \Gamma = (U^T U)^{-1} (U^T M U) (U^T U)^{-1}
%     \end{align*}
%     and that the optimal value is precisely the value of \cref{eq:minimum-attained-low-rank-approximation} as claimed.
%     Finally, if $U$ has orthonormal columns, $U^T U = I_K$ so $\Gamma^*$ is of the form claimed, and the final equality comes from expanding out the trace form of the Frobenius norm.
% \end{proof}

% \begin{lemma}\label{lem:clever-basis-choice-for-orthogonality}
% Let $M \in \R^{D \times D}$ be a symmetric matrix and $\mathcal{U}$ a subspace of $\R^D$ of dimension $L$.
% Then there exists an orthonormal basis $u_1, \dots, u_L$ for $\mathcal{U}$ such that
% \begin{align*}
%     u_L \perp M u_l \: \text{ for } l \in \{1, \dots, L-1 \}
% \end{align*}
% \end{lemma}
% \begin{proof}
%     Consider the action of $\tilde{M} \defeq \mathcal{P}_\mathcal{U} M \mathcal{P}_\mathcal{U}$ on $\mathcal{U}$.
%     Then $\tilde{M}$ is symmetric matrix whose range is a subspace of $\mathcal{U}$  and so there exists an orthonormal set of eigenvectors $u_1, \dots, u_L$ that give a basis for $\mathcal{U}$ with corresponding eigenvalues $\tilde{\lambda}_1 \geq \dots \geq \tilde{\lambda}_L$.
%     Then we can read off
%     \begin{align*}
%         \langle u_L, M u_l \rangle = \langle u_L, \tilde{M} u_l \rangle = \tilde{\lambda}_l \langle u_L, u_l \rangle = 0
%     \end{align*}
%     as required.
% \end{proof}

% % Note, this is no longer very consistent with the main text.
% \begin{proposition}[No spurious local minima]
%     The (population) objective $\mathcal{L}^{EY}$ has no spurious local minima.
%     That is, any matrix $\bar{W}$ that is a local minimum of $\mathcal{L}^{EY}$ must in fact be a global minimum of the form described in \cref{prop:EY-charac}.
% \end{proposition}
% \begin{proof}
%     We shall show that for any matrix $W$ that is not a global optimum, there is a (continuous) path of solutions $W_t$ with:
%     \begin{equation*}
%         W_0 = W, \quad W_1 = \hat{W}, \quad W_t \to W \text{ as } t \to 0, \quad \text{and} \quad \mathcal{L}^{EY}(W_t) < \mathcal{L}^{EY}(W) \: \forall t > 0
%     \end{equation*}

%     As in the proof of \cref{prop:EY-charac} we first reduce to the $B=I$ setting by defining $Z \defeq B^{\mhalf} W$ and $M = B^\mhalf A B^\mhalf$.
%     Let the eigendecomposition of $M$ be $M = V^* D^* {V^*}^\top$.
%     Define the loss
%     \begin{align*}
%         l(Z) \defeq \norm{M - Z Z^\top}_F^2
%     \end{align*}

%     It is now sufficient to show that: for any matrix $Z \in \R^{D \times K}$ that is not of the form $V^*_K D^*_K O_K$ where $V^*_K$ is a matrix whose columns are a set of top-$K$ eigenvectors for $M$, and $O_K \in \R^{K \times K}$ is some arbitrary orthogonal matrix cannot be a local minimum.

%     For notational simplicity we will assume that the $\lambda_K(M) > \lambda_{K+1}(M)$ from now on, such that $V_K^*$ can be made well-defined\footnote{with symmetry breaking for earlier repeated eigenvalues if required.}.

%     Now, take such a $Z$ and suppose, for contradiction that it is a local minimum.
%     We will construct a continuous path of matrices $Z(t): t \in [0,1]$ with $Z(0) = Z$ and $l\left(Z(t)\right) < l(Z) \: \forall t > 0$.

%     % Let $Z Z^T$ have eigen-decomposition $V_K D_K V_K^T = \sum_{k=1}^K d_k v_k v_k^T$.
%     % Then $Z = V_K D_K^\half O_K$ for some orthogonal matrix $O_K$.
%     % We will construct the path $Z(t)$ from certain paths for $V_K(t), D_K(t)$ while keeping $O_K$ fixed.

%     % Firstly, we consider changing $Z$ through the diagonal matrix $D_K$. Extend $V_K$ to an orthogonal matrix $V$. Then rearrange
%     % \begin{align*}
%     %     \norm{M - Z Z^T}_F^2 = \norm{V^T ( M - Z Z^\top) V}_F^2 = \norm{V^T M V - \begin{pmatrix} D_K & 0 \\ 0 & 0 \end{pmatrix}}
%     % \end{align*}
%     % Since $Z$ is a local optimum, $D$ must be a local optimum of this objective and therefore we must have $d_k = v_k^T M v_k$.

%     Then by our assumption on the form of $Z$, we have
%     \begin{align*}
%         \mathcal{V}_K \defeq \cspan{Z} \neq \cspan{V_K^*} \eqdef \mathcal{V}^*_K
%     \end{align*}

%     Now comes the clever part of the proof.
%     Define $\kappa_\cap = \operatorname{dim}\cspan{\mathcal{V}_K \cap \mathcal{V}_K^*}$. Then pick orthonormal bases
%     \begin{itemize}
%         \item $u_1, \dots, u_{\kappa_\cap}$ for $\mathcal{V}_K \cap \mathcal{V}_K^*$
%         \item $u_{\kappa_\cap + 1}, \dots, u_K$ for $\mathcal{V}_K \cap \mathcal{V}_K^*$ such that $u_K \perp M u_k$ for all $k = \kappa_\cap + 1, \dots, K-1$ by \cref{lem:clever-basis-choice-for-orthogonality}
%         \item $u^*_{\kappa_\cap + 1}, \dots, u^*_K$ for $\mathcal{V}_K^* \cap \mathcal{V}_K$
%     \end{itemize}

%     Let $U_K = \begin{pmatrix}u_1 & \dots & u_K\end{pmatrix}$.
%     Then by \cref{lem:least-squares-projected-signal}, for $Z$ to be a local minimum we must have
%     \begin{align*}
%         Z Z^T = U_K (U_K^T M U_K) U_K^T
%     \end{align*}
%     Moreover the objective value must therefore be
%     \begin{align}\label{eq:objective-decreases-as-signal-increases}
%     l(Z) = \norm{M}_F^2 - \norm{U_K^T M U_K}_F^2
%     \end{align}

%     We now make the observation that the second term is the `signal of $M$ captured by the subspace of $U_K$'.
%     So aligning $U_K$ with higher-eigenvalue subspaces of $M$ should increase this amount of signal captured and decrease this loss.

%     We now construct a path $U_K(t)$ which captures this intuition.

%     Let $u_K(t) = \cos(t) u_K + \sin(t) u^*_K$.
%     Then let $U_K(t)$ have columns $u_1, \dots, u_{K-1}, u_K(t)$.
%     By construction this is still an orthonormal set of basis vectors, so $U_K(t)^T U_K = I_K$.
%     Let $\Gamma(t) = U_K(t)^T M U_K(t)$.

%     We are finally ready to construct the path $Z(t)$.
%     Because $U_K$ is a basis for the column space of $Z$, and $Z$ is assumed to be a local optimum, we must have
%     \begin{align*}
%         Z Z^T = U_K \Gamma(0) U_K^T
%     \end{align*}
%     by \cref{lem:least-squares-projected-signal}.
%     So $Z = U_K \Gamma^\half O_K$ for some orthogonal matrix $O_K \in \R^{K \times K}$ where $\Gamma^\half$ is the unique positive semi-definite square root of $\Gamma$.
%     So define
%     \begin{align*}
%         Z(t) = U_K(t) \Gamma(t)^\half O_K
%     \end{align*}
%     where again $\Gamma(t)^\half$ is the unique positive semi-definite square root and therefore both $U_k(t)$ and  $\Gamma(t)^\half$ are continuous functions of $t$ and therefore so is $Z$.

%     Then
%     \begin{align}\label{eq:objective-decreases-as-signal-increases}
%     l(Z(t)) = \norm{M}_F^2 - \norm{U_K(t)^T M U_K(t)}_F^2
%     \end{align}
%     So it is sufficient to show that $\norm{U_K(t)^T M U_K(t)}_F^2 > \norm{U_K^T M U_K}_F^2$ for $t \in [0, \pi/2]$.
%     Indeed, we can compute
%     \begin{align*}
%         \norm{U_K(t)^T M U_K(t)}_F^2 - \norm{U_K^T M U_K}_F^2
%         &= \left(u_K(t)^T M u_K(t)\right)^2 - \left(u_K^T M u_K\right)^2 \\
%         & \quad + 2 \sum_{k=1}^{K-1} \left\{ \left(u_K(t)^T M u_k\right)^2 - \left(u_K^T M u_k\right)^2 \right\} \\
%         &\geq \left(u_K(t)^T M u_K(t)\right)^2 - \left(u_K^T M u_K\right)^2
%     \end{align*}
%     because $u_K^T M u_k = 0$ for $k = 1, \dots, K-1$ by construction.
%     Finally we have
%     \begin{align*}
%         u_K(t)^T M u_K(t)
%         &= \sin^2(t) \langle u_K^*, M u_K^* \rangle + 2 \sin(t)\cos(t) \langle u_K, M u_K^* \rangle + \cos^2(t) \langle u_K, M u_K \rangle \\
%         &= \sin^2(t) \langle u_K^*, M u_K^* \rangle + \cos^2(t) \langle u_K, M u_K \rangle \\
%         &> u_K^T M u_K
%     \end{align*}
%     Here we used that $\langle u_K^*, M u_K^* \rangle \geq \lambda_K > \langle u_K, M u_K \rangle$ and that the middle term vanishes because $M u_K^* \in \mathcal{U}^*_K$ and is therefore orthogonal to $u_K$.

% \end{proof}


% \subsection{Quantitative results}\label{sec:no-spurious-local-minima-quantitative}
% To use the results from \citet{ge_no_2017} we need to introduce their definition of a $(\theta, \gamma, \zeta)$-strict saddle.

% \begin{definition}
%     We say function $l(\cdot)$ is a $(\theta, \gamma, \zeta)$-\textbf{strict saddle} if for any $x$, at least one of the following holds:
%     \begin{enumerate}
%         \item $\norm{\nabla l (x)} \geq \theta$
%         \item $\lambda_{\text{min}}(\nabla^2 l(x)) \leq - \gamma$
%         \item $x$ is $\zeta$-close to $\mathcal{X}^*$ - the set of local minima.
%     \end{enumerate}.
% \end{definition}

% We can now state restate Lemma 13 from \citet{ge_no_2017} in our notation; this was used in their analysis of robust PCA, and directly applies to our PCA-type formulation.
% \begin{lemma}[Strict saddle for PCA]\label{lem:strict-saddle-pca}
% Let $M \in \R^{D \times D}$ be a symmetric PSD matrix, and define the matrix factorization objective over $Z \in \R^{D \times K}$
% \begin{align*}
%     l(Z) = \norm{M - Z Z^\top}^2
% \end{align*}
% Assume that $\lambda^*_K \defeq \lambda_K(M) \geq 15 \lambda_{K+1}(M)$.
% Then
% \begin{enumerate}
%     \item all local minima satisfy $Z Z^T = \mathcal{P}_K(M)$ - the best rank-$K$ approximation to $M$
%     \item the objective l(Z) is $(\epsilon, \Omega(\lambda_K^*), \mathcal{O}(\epsilon / \lambda_K^*)$-strict saddle.
% \end{enumerate}
% \end{lemma}

% However, we do not want to show a strict saddle of $l$ but of $\LEY: U \mapsto l(B^\half U)$. Provided that $B$ has strictly positive minimum and bounded maximum eigenvalues this implies that $\LEY$ is also strict saddle, as we now make precise.

% % think I've got the conclusion the wrong way round because had wrong power of $B$.
% \begin{lemma}[Change of variables for strict saddle conditions]\label{lem:strict-saddle-change-of-variables}
% Suppose that $l$ is $(\theta, \gamma, \zeta)$-strict saddle and let $L: U \mapsto l(B^\half U)$ for $B$ with minimal and maximal eigenvalues $\sigma_\text{min}, \sigma_\text{max}$ respectively.

% Then $L$ is \(\left(\sigma_\text{max}^\half \theta,\, \sigma_\text{min} \gamma,\, \sigma_\text{max}^\half \zeta \right)\)-strict saddle.
% \end{lemma}
% \begin{proof}
%     Write $g(U) = B^\half U$. Then $L = l \circ g$, so by the chain rule:
%     \begin{align*}
%         D_U L = D_{B^\half U} l \circ D_U g \, : \: \delta U \mapsto \langle \nabla l(B^\half U), B^{\half} \delta U \rangle
%         = \langle B^\half \nabla l(B^\half U), \delta U \rangle
%     \end{align*}
%     Therefore
%     \begin{align*}
%         \norm{\nabla L(U)} = \norm{B^\half \nabla l(B^\half U)} \geq \sigma_\text{min}^\half \norm{l(B^\half U)}
%     \end{align*}

%     By a further application of the chain rule we have
%     \begin{align*}
%         D_U^2 L \, : \: \delta U, \delta U \mapsto D^2_{B^\half U} l(B^\half \delta U, B^\half \delta U)
%     \end{align*}

%     Suppose $\lambda_{\text{min}}(\nabla^2 l(Z)) \leq - \gamma$ then by the variational characterization of eigenvalues, there exists some $\delta Z$ such that $\langle \delta Z, \nabla^2 l(Z) \delta Z \rangle \leq - \gamma \norm{\delta Z}^2$.
%     Then taking $\delta U = B^\mhalf \delta Z$ gives
%     \begin{align*}
%         \langle \delta U, \nabla^2 L(U) \delta U \rangle
%         &= \langle B^\half \delta U, \nabla^2 l(B^\half U) B^\half \delta U \rangle \\
%         &= \langle \delta Z, \nabla^2 l(Z) \delta Z \rangle  \\
%         &\leq - \gamma \norm{\delta Z}^2 \\
%         & \leq - \gamma \sigma_{\text{min}} \norm{\delta U}^2
%     \end{align*}

%     Thirdly, suppose that $\norm{B^\half U - Z^*} \leq \zeta$ for some local optimum $Z^*$ of $l$. Then since $B$ is invertible, $U^* \defeq B^\mhalf Z^*$ is a local optimum of $L$. In addition:
%     \begin{align*}
%         \norm{U - U^*} = \norm{B^\half( U - U^*)} \leq \sigma_\text{max}^\half \norm{B^\half U - Z^*} \leq \zeta
%     \end{align*}

%     Finally, consider some arbitrary point $U_0$. Let $Z_0 = B^\half U_0$. Then by the strict saddle property for $l$ one of the following must hold:
%     \begin{enumerate}
%         \item $\norm{\nabla l (Z_0)} \geq \theta \quad \implies \quad \norm{\nabla L(U_0)} \geq \sigma_\text{min}^\half \theta$
%         \item $\lambda_{\text{min}}(\nabla^2 l(Z_0)) \leq - \gamma \quad \implies \quad \lambda_{\text{min}}(\nabla^2 L(U_0) \leq - \sigma_{\text{min}} \gamma$
%         \item $Z_0$ is $\zeta$-close to a local-minimum $Z^*$, which implies that $U_0$ is $(\sigma_\text{max}^\half \zeta)$-close to a local minimum $B^\mhalf Z^*$ of $L$.
%     \end{enumerate}
% \end{proof}

% By combining \cref{lem:strict-saddle-pca} with \cref{lem:strict-saddle-change-of-variables}, we can conclude that our objective does indeed satisfy a (quantitative) strict saddle property.
% This is sufficient to show that certain local search algorithms will converge in polynomial time \cite{ge_no_2017}.

% \newpage

% \section{Fast updates for (Multi-view) Stochastic CCA (and PLS)}\label{supp:fast-updates}
% \subsection{Back-propagation for empirical covariances}
% To help us analyse the full details of back-propagation in the linear case, we first prove a lemma regarding the gradients of the empirical covariance operator.

% \begin{lemma}[Back-prop for empirical covariance]\label{lem:backprop-empirical-covariance}
% Let $e \in \R^{M}, f \in \R^{M}$.
% Then $\empCov(e, f)$ and
% \begin{align*}
%     \frac{\partial \empCov(e, f)} {\partial e}
% \end{align*}
% can both be computed in $\order{M}$ time.
% \end{lemma}
% \begin{proof}
%     Let $\ones{M} \in \R^M$ be a vector of ones and $\mathcal{P}_{\ones{M}}^\perp = I_M - \frac{1}{M} \ones{M}^T \ones{M}$ be the projection away from this vector, then we can write $\bar{e} = \mathcal{P}_{\ones{M}}^\perp e, \bar{f} = \mathcal{P}_{\ones{M}}^\perp f$.
%     Moreover, exploiting the identity-plus-low-rank structure of $\mathcal{P}_{\ones{M}}^\perp$ allows us to compute these quantities in $\order{M}$ time.

%     Then by definition
%     \begin{align*}
%         \empCov(e, f) = \frac{1}{M-1} \bar{e}^T \bar{f}
%     \end{align*}
%     which is again computable in $\order{M}$ time.

%     For the backward pass, first note that
%     \begin{align*}
%         \frac{\partial \bar{e}}{\partial e} \, : \, \delta e \mapsto \mathcal{P}_{\ones{M}}^\perp \delta e
%     \end{align*}
%     So the derivative with respect to $e$ is
%     \begin{align*}
%         \frac{\partial \empCov(e, f)} {\partial e} = \frac{1}{M-1} \frac{\partial \bar{e}^T \bar{f}}{\partial e}
%         = \frac{1}{M-1} \left( \frac{\partial \bar{e}}{\partial e} \bar{f}  \right)
%         = \frac{1}{M-1} \mathcal{P}_{\ones{M}}^\perp \bar{f}
%         = \frac{1}{M-1}\bar{f}
%     \end{align*}
%     because $\bar{f}$ is independent of $e$, and already mean-centred.
%     So all that remains is element-wise division, which again costs $\order{M}$ time.
% \end{proof}


% \subsection*{Forward Pass}
% \begin{enumerate}
%     \item \textbf{Compute the transformed variables \(\Z\):}
%     \begin{equation}
%         \Z\sps{i} = U\sps{i} \X\sps{i},
%     \end{equation}
%     with a complexity of \(\mathcal{O}(MKD)\).

%     \item \textbf{Compute \(\tr\hat{C}(\theta)[\Z]\):} the diagonal elements of $\hat{C}$ are simply
%     \begin{align*}
%         \hat{C}_{kk} = \sum_{i \neq j} \empCov(\Z_k\sps{i}, \Z_k\sps{j})
%     \end{align*}
%     which each summand can be computed in $\order{M}$ time, so summing over $i,j,k$ gives total complexity of \(\mathcal{O}(I^2 K M)\).

%     \item \textbf{Compute \(\hat{V}(\theta)[\Z]\):}
%     For \( \hat{V}_\alpha[\Z] \):
%     \begin{align*}
%         \hat{V}_\alpha(\theta)[\Z] = \sum_i \alpha_i {U\sps{i}}^T U\sps{i} + (1 - \alpha_i) \empVar(\Z\sps{i}),
%     \end{align*}
%     each ${U\sps{i}}^T U\sps{i}$ can be computed with a complexity of $\order{D_i K^2}$\ and the total cost of evaluating all of these is $\order{K^2 D}$.
%     Each summand in the second term costs $\order{M K^2}$ by \cref{lem:backprop-empirical-covariance} so evaluating the full second term costs $\order{I MK^2}$.

%     \item \textbf{Evaluate \(\empLEY[\Z, \Z']\):}
%     \begin{equation}
%         \empLEY[\Z, \Z'] = - 2 \tr \hat{C}[\Z] + \langle \hat{V}_\alpha[\Z], \hat{V}_\alpha[\Z'] \rangle_F.
%     \end{equation}
%     The dominant complexity here is the \(\mathcal{O}(K^2)\) cost of computing the Frobenius inner product.
% \end{enumerate}

% \subsection*{Backward Pass}

% \begin{enumerate}
%     \item \textbf{Gradient with respect to \( \Z\sps{i} \):}
%     Using the chain rule, the gradient will flow back from the final computed value, \( \empLEY[\Z, \Z'] \), through the operations that produced it.

%     \item \textbf{Gradient of \( \tr\hat{C}(\theta)[\Z] \) with respect to \( \Z\sps{i}_k \):}
%     Is precisely
%     \begin{align*}
%         \frac{\partial \hat{C}_{kk}}{\partial \Z_k\sps{i}} = \frac{2}{M-1} \sum_{j \neq i} \bar{\Z}_k\sps{j},
%     \end{align*}
%     where $\bar{\Z}_k\sps{j} = \mathcal{P}_{\ones{M}}^\perp \bar{\Z}_k\sps{j} $, from \cref{lem:backprop-empirical-covariance} and so can be computed in \( \mathcal{O}(I M) \) time.

%     \item \textbf{Gradients of $\langle \hat{V}_\alpha[\Z], \hat{V}_\alpha[\Z'] \rangle_F$ with respect to \( \Z\sps{i}_k \):}
%     By applying \cref{lem:backprop-empirical-covariance}, the gradient of the empirical variance term is
%     \begin{align*}
%         \frac{\partial \empVar(\Z\sps{i})_{l,l'}}{\partial \Z_k\sps{i}} =
%         \begin{cases}
%             \frac{2}{M-1} \Z\sps{i}_k & \text{ if $l=l'=k$}\\
%             \frac{1}{M-1} \Z\sps{i}_l & \text{ if $l \neq l'=k$}\\
%             0 & \text{ otherwise.}
%         \end{cases}
%     \end{align*}
%     and so
%     \begin{align*}
%         \frac{\partial \langle \hat{V}_\alpha[\Z], \hat{V}_\alpha[\Z'] \rangle_F}{\partial \Z_k\sps{i}}
%         &= \frac{(1 - \alpha_i)}{M-1}\left( 2 \hat{V}_\alpha[\Z']_{kk} \Z_k\sps{i} + \sum_l ( \hat{V}_\alpha[\Z']_{lk} \Z_l\sps{i} + \hat{V}_\alpha[\Z']_{kl} \Z_k\sps{i}) \right) \\
%         &= \frac{2(1 - \alpha_i)}{M-1}\sum_{l=1}^K \hat{V}_\alpha[\Z']_{lk} \Z_l\sps{i} \\
%     \end{align*}
%     this can be computed in \( \mathcal{O}(M K) \) time.

%     \item \textbf{Gradients of \( \empLEY[\Z, \Z'] \) with respect to $\Z\sps{i}_k$:} can therefore be computed for a given $\Z\sps{i}_k$ in $\order{M (K + I)}$ time and so, adding up over all $i,k$ gives total $\order{I M (K + I)}$ time.

%     \item \textbf{Gradients of $\langle \hat{V}_\alpha[\Z], \hat{V}_\alpha[\Z'] \rangle_F$ with respect to $U\sps{i}_k$:}
%     is similarly
%     \begin{align*}
%         \frac{2\alpha_i}{M-1}\sum_{l=1}^K (\hat{V}_\alpha[\Z]_{lk} + \hat{V}_\alpha[\Z']_{lk}) U_l\sps{i}
%     \end{align*}
%     so can be computed in $\order{D_i K}$ time.

%     \item \textbf{Finally compute gradients with respect to $U\sps{i}_k$:} simply have $Z\sps{i}_k = {U\sps{i}_k}^\top \X\sps{i}$ so the final gradients are
%     \begin{align}
%         \frac{\partial \empLEY}{\partial U\sps{i}_k} = \left( \frac{\partial \empLEY}{\partial \Z\sps{i}_k} \right)^\top {\X\sps{i}}
%         + \frac{\partial \langle \hat{V}_\alpha[\Z], \hat{V}_\alpha[\Z'] \rangle_F}{\partial U\sps{i}_k}
%     \end{align}
%     so the dominant cost is the $\order{M D_i}$ multiplication.
% \end{enumerate}

% Since $D \gg K, M$, the dominant cost each final gradient is $\order{M D_i}$.
% Summing up over $i,k$ gives total cost $\order{K M \sum D_i} = \order{K M D}$, as claimed.

% \newpage