\chapter{Background: Multiview Machine Learning: Concepts, Methods, and Limitations}
\label{chap:background}
\minitoc

\section{Machine Learning}
Machine learning is a subfield of computer science that automates the process of data analysis, enabling systems to learn from data and make decisions without being explicitly programmed.
It is a cornerstone of modern artificial intelligence, offering a set of tools that extends far beyond traditional statistical methods.
These automated methods are used in a myriad of applications, from filtering spam emails to making medical diagnoses and powering self-driving cars.

This field is generally divided into three primary categories: supervised, unsupervised, and reinforcement learning.
Each of these paradigms is suited to a different type of problem, and the choice of approach often depends on the type and amount of data available.
This thesis will focus on a specialized area of machine learning known as `multiview machine learning," which deals with data sets that contain multiple types or `views" of information.

\subsection{Supervised Learning}

Supervised learning is a machine learning paradigm where the algorithm learns from labeled training data, and makes predictions or decisions based on that data.
This type of learning allows for the mapping of input data to output labels, essentially using annotated examples to discern how to generalize to unseen data.
A classic example of supervised learning is a spam email filter, where a model is trained on a dataset of emails labeled as 'spam' or 'not spam', and the task is to classify new, incoming emails into one of these categories.

The primary components of supervised learning include a set of input-output pairs known as the training set.
The goal is to find a function that approximates the relationship between the corresponding input and output variables.
This function is then applied to new, unseen data to make predictions or decisions.

\subsection{Unsupervised Learning}

Unsupervised learning methods learn a mapping from inputs to outputs without any training targets.
Unsupervised learning methods can be used to learn representations of the data that can be used for downstream tasks such as classification or regression.
They can also be used as a way to discover relationships between the data, such as understanding the underlying structure of the data or finding correlations between different modalities of data.
Some generative models can also be used to generate new data with similar characteristics to the training data.

Perhaps the most well-known example of unsupervised learning is principal components analysis (PCA), which learns a mapping from inputs to outputs based on the directions of maximum variance in the data.
PCA can be used to learn a low-dimensional representation of the data that captures the most variance in the data.

\subsection{Self-Supervised Learning}

Self-supervised learning methods learn a mapping from inputs to outputs based on a training signal that is derived from the data itself.
The transformer model behind the success of many Large Language Models (LLMs) such as BERT \cite{devlin2018bert} and GPT-3 \cite{brown2020language} is trained using a self-supervised learning method called masked language modelling.
In masked language modelling, the model is trained to predict a masked word in a sentence based on the other words in the sentence.
Of closer relevance to this PhD thesis is work on self-supervised learning for computer vision tasks.
In these methods, the model is trained to predict a patch of an image based on the other patches in the image.

Like unsupervised learning methods, self-supervised learning methods can be used to learn representations of the data that can be used for downstream tasks such as classification or regression.

\subsection{Multiview Machine Learning}

Throughout this report we will refer to different modalities of data for the same subject as different `views', consistent with the literature\cite{sun2013survey}.
Multiview machine learning is a branch of machine learning that deals with data that have multiple sources or modalities that describe the same phenomenon or entity.
For example, a person can be represented by their face image, voice, text, and gesture.
Each source or modality is referred to as a view, and different views may provide complementary or redundant information.

Multiview learning methods can be used to generate robust low-dimensional representations for a downstream task such as classification or regression, or to discover relationships between views such as correlation or even causation.
They been widely applied across a range of fields such as neuroimaging\cite{Krishnan2011}, finance\cite{cassel2000measurement}, Imaging Genetics\cite{Hansen2021}, to find associations between views in large datasets.

Multiview machine learning methods can be interpreted as either unsupervised or self-supervised depending on the underlying assumptions about their data-generating process.
Specifically, the presence of a shared latent variable can influence how we categorize these methods.
This interpretation will have important implications for the methods that we consider in this thesis.

\subsection{Unsupervised Multiview Machine Learning}
In unsupervised multiview machine learning, the focus is typically on finding a shared representation that captures the essence of different views without making any assumptions about the nature of these views.
The main idea here is that each view provides a different `angle' on the same object or phenomenon, but there is no explicit modeling of a shared latent variable that generates these views.
Methods such as Canonical Correlation Analysis (CCA) aim to maximize the correlation between the different views in a
common space but do not inherently posit that these views come from a single latent source.

\subsection{Self-Supervised Multiview Machine Learning}
Self-supervised multiview machine learning, on the other hand, often assumes that the different views are generated from a common latent variable.
In this sense, one can argue that the task of learning from multiview data becomes a form of self-supervised learning.

It is important to note that the distinction between unsupervised and self-supervised learning is not always clear-cut.
In particular, we will argue that a number of classical subspace learning algorithms including Canonical Correlation
Analysis (CCA) can be interpreted as both an unsupervised (learning associations between views) and a self-supervised (where the derived target is the subject) method depending on the context.



\section{Learning Representations: Definitions and Notation}

Suppose we have a sequence of vector-valued random variables $X^{(i)} \in \R^{D_i}$ for $i \in \{1, \dots, I \}$
We want to learn meaningful $K$-dimensional representations
\begin{equation}\label{eq:general-form-of-representations}
    Z\sps{i} = f\sps{i}( X\sps{i}; \theta\sps{i}).
\end{equation}
For convenience, define $D = \sum_{i=1}^I D_i$ and $\theta = \left(\theta\sps{i}\right)_{i=1}^I$.
Without loss of generality take $D_1 \geq D_2 \geq \cdots \geq D_I$.
We will consistently use the subscripts $i,j \in [I]$ for views;
$d \in [D_i]$ for dimensions of input variables;
and $l,k \in [K]$ for dimensions of representations - i.e. to subscript dimensions of $Z\sps{i}, f\sps{i}$.
Later on we will introduce total number of samples $N$.

In this report, we will typically refer to $u_k$ as \textbf{weights}, $Z_k = X_k u_k$ as \textbf{representations},
        \textbf{latent
dimensions}, or \textbf{scores} depending on the context. We will sometimes consider the
        matrix $U = \left(u_1, \dots, u_K\right) \in \R^{D \times K}$ of weights, and the
        matrix $Z = \left(Z_1, \dots, Z_K\right) \in \R^{N \times K}$ of representations. We will refer to $\mathbb{E
        }[X_k^\top X_k] u_k$ as loadings.

\subsection{Principal Components Analysis}

Principal Components Analysis (PCA)\cite{hotelling1933analysis} is a classical method in unsupervised machine learning for representation learning.
It is widely used for dimensionality reduction and feature extraction.
The primary goal of PCA is to transform the original high-dimensional data into a new coordinate system defined by orthogonal axes, capturing the most relevant aspects of the data.

In PCA, the representations are constrained to be linear transformations of the form:
\begin{equation}\label{eq:pca-linear-function-def}
    Z_k = X u_k,
\end{equation}
where $u_k$ are the orthonormal basis vectors:
\begin{equation}\label{eq:pca-orthonormality-constraint}
    u_k^\top u_l = \delta_{kl}.
\end{equation}

The primary goal of PCA is to maximize the variance of the projections \(Z_k\). Mathematically, this can be formulated
as:
\begin{align}
    u_{\text{opt}} &= \underset{u}{\text{argmax}} \left( u^\top X^\top Xu \right) \\
    \text{subject to:} \notag \\
    u^\top u &= 1 \notag
\end{align}

\paragraph{Optimization and Solution}
The Lagrangian for this problem is:
\begin{equation}
    f(u,\lambda) = u^\top X^\top Xu + \lambda(1 - u^\top u),
\end{equation}
where \(\lambda\) is the Lagrange multiplier. Differentiating the Lagrangian yields the first-order conditions:
\begin{align}
    X^\top X u &= \lambda u, \\
    u^\top u &= 1.
\end{align}

This transforms the problem into an eigenvalue equation for the covariance matrix \(X^\top X\), which can be efficiently solved using standard libraries such as scikit-learn\cite{pedregosa2011scikit}.

The first principal component corresponds to the eigenvector associated with the largest eigenvalue \(\lambda\). Subsequent components are ordered by their corresponding eigenvalues.

\textbf{Limitations: }However, when applying PCA to datasets such as high-dimensional neuroimaging and behavioral
data, PCA's main limitation arises: it only accounts for variance within a single dataset, potentially discarding features that are relevant for cross-modal analysis.

\subsection{Partial Least Squares}

Partial Least Squares (PLS)\cite{wold1975path} aims to maximize the shared covariance between two paired sets of data, referred to as `views". PLS can be seen as a generalization of PCA, where PCA becomes a special case when the two views are identical. The optimization problem for PLS can be formulated as:

\begin{align}
     u\sps{1}_{\text{opt}} &= \underset{u\sps{1}}{\mathrm{argmax}} \{ u\sps{1}_{1}^{\top} X\sps{1}^{\top} X\sps{2} u\sps{2} \} \\
     \text{subject to:} \notag \\
     u\sps{1}_1^{\top}u\sps{1}_1 &= 1 \notag \\
     u\sps{2}_1^{\top}u\sps{2}_1 &= 1 \notag
\end{align}

where \( X\sps{1} \in \mathbb{R}^{n \times p_1} \) and \( X\sps{2} \in \mathbb{R}^{n \times p_2} \), meaning we have two views with the same number of samples but potentially different number of features.

\subsubsection{Eigenvalue Problem}

The Lagrangian for this optimization problem can be formulated as:

\begin{equation}
f(u\sps{1}, \lambda) = u\sps{1}_{1}^{\top} X\sps{1}^{\top} X\sps{2} u\sps{2} + \lambda_1 (1 - u\sps{1}_1^{\top}u\sps{1}_1) + \lambda_2 (1 - u\sps{2}_1^{\top}u\sps{2}_1)
\end{equation}

Upon deriving the first order conditions, we get:

\begin{align}
    X\sps{2}^{\top} X\sps{1} u\sps{1}_1 &= \lambda_2 u\sps{2}_1 \\
    X\sps{1}^{\top} X\sps{2} u\sps{2}_1 &= \lambda_1 u\sps{1}_1 \\
    u\sps{1}_1^{\top}u\sps{1}_1 &= 1 \\
    u\sps{2}_1^{\top}u\sps{2}_1 &= 1
\end{align}

By substituting the constraint conditions into these equations, we find that \( \lambda_1 = \lambda_2 = \lambda \) by symmetry. Further simplification yields:

\begin{align}
    X\sps{2}^{\top}X\sps{1} X\sps{1}^{\top} X\sps{2} u\sps{2}_1 &= \lambda^2 u\sps{2}_1 \\
    X\sps{1}^{\top}X\sps{2} X\sps{2}^{\top} X\sps{1} u\sps{1}_1 &= \lambda^2 u\sps{1}_1
\end{align}

Thus, solving these equations will yield the \( u\sps{1}_1 \) and \( u\sps{2}_1 \) vectors as the eigenvectors of \( X\sps{1}^{\top} X\sps{2} X\sps{2}^{\top} X\sps{1} \) and \( X\sps{2}^{\top} X\sps{1} X\sps{1}^{\top} X\sps{2} \), respectively \cite{hoskuldsson1988pls}.

\textbf{Limitations: } The problem with applying PLS to neuroimaging and behavioural modalities is that PLS is not scale invariant and
is therefore biased towards the largest principal components in the data \cite{helmer2020stability}.
This is particularly problematic when there is a low signal to noise ratio since PLS may find directions in either dataset which correspond to the largest directions of noise in the other.
Additionally, PLS assumes that the structures contributing to variance in both datasets are linearly related, which
may not be the case in complex biological systems like the brain or in intricate behavioral patterns \cite{rosipal2005overview}.
The linearity assumption can sometimes be overly restrictive, failing to capture more complicated, nonlinear relationships between the data modalities.
Another issue is the lack of sparsity in the PLS solution.
Traditional PLS methods do not provide sparse weight vectors, which makes the interpretation of results challenging in high-dimensional settings such as neuroimaging where only a subset of features might be relevant \cite{leurgans1993canonical}.
There are sparse variants of PLS available, but these typically introduce additional complexity and may require fine-tuning of regularization parameters \cite{chun2010sparse}.
Furthermore, PLS can be sensitive to outliers, which are not uncommon in neuroimaging data due to motion artifacts or other sources of noise.
Since the method aims to maximize covariance, extreme values in one dataset can disproportionately affect the resulting latent variables \cite{wold1975path}.

\subsection{Canonical Correlation Analysis}\label{sec:cca}

In CCA, we aim to find the directions that maximize correlation, as opposed to maximizing covariance between two views of a dataset.
This nuance renders CCA invariant to feature scale. The optimization problem for CCA can be expressed as:

\begin{align}
     & u_{\text{opt}}=\underset{u}{\mathrm{argmax}}\{ u\sps{1}^{\top}X\sps{1}^{\top}X\sps{2}u\sps{2} \} \\
     & \text{subject to:} \notag \\
     & u\sps{1}^{\top}X\sps{1}^{\top}X\sps{1}u\sps{1}=1 \notag \\
     & u\sps{2}^{\top}X\sps{2}^{\top}X\sps{2}u\sps{2}=1 \notag
\end{align}

Although non-convex, numerous methods exist for solving the CCA problem, such as eigenvalue problems, generalized eigenvalue problems, block coordinate descent via alternating least squares regressions \cite{golub1995canonical,sun2008least} , and gradient descent \cite{via2007learning}.

\subsubsection{Eigenvalue Problem}

The first-order conditions derived in the same manner as the PLS case are:

\begin{align}\label{CCA:FOCs}
     & X\sps{2}^{\top}X\sps{1}u\sps{1}=\lambda\sps{2} X\sps{2}^{\top}X\sps{2}u\sps{2} \\
     & X\sps{1}^{\top}X\sps{2}u\sps{2}=\lambda\sps{1} X\sps{1}^{\top}X\sps{1}u\sps{1} \\
     & u\sps{1}^{\top}X\sps{1}^{\top}X\sps{1}u\sps{1}=1 \\
     & u\sps{2}^{\top}X\sps{2}^{\top}X\sps{2}u\sps{2}=1
\end{align}

Substituting the second two conditions into the first two, we get \(\lambda\sps{1}=\lambda\sps{2}=\lambda\). Then, recognizing \(X_i^{\top}X_i\) as the covariance matrix \(\Sigma_{ii}\) and \(X_i^{\top}X_j\) as the cross-covariance matrix \(\Sigma_{ij}\), we obtain another pair of eigenvalue problems:

\begin{align}
     & \Sigma\sps{11}^{-1}\Sigma\sps{12}\Sigma\sps{22}^{-1}\Sigma\sps{21}u\sps{1}=\lambda^2u\sps{1} \notag \\
     & \Sigma\sps{22}^{-1}\Sigma\sps{21}\Sigma\sps{11}^{-1}\Sigma\sps{12}u\sps{2}=\lambda^2u\sps{2} \notag
\end{align}

An alternative form of the CCA problem can be developed by reparameterizing \(u^*_i=(X_i^{\top}X_i)^{-\frac{1}{2}}u_i\). The optimization problem then becomes:

\begin{align}
     & u^*_{\text{opt}}=\underset{u^*}{\mathrm{argmax}}\{ u^{*T}\sps{1}(X\sps{1}X\sps{1}^{\top})^{-\frac{1}{2}}X\sps{1}^{\top}X\sps{2}(X\sps{2}^{\top}X\sps{2})^{-\frac{1}{2}}u^*\sps{2} \} \\
     & \text{subject to:} \notag \\
     & u^{*T}\sps{1}u^*\sps{1}=1 \notag \\
     & u^{*T}\sps{2}u^*\sps{2}=1 \notag
\end{align}

This reparameterized form will later underpin Deep Canonical Correlation Analysis (DCCA).

This form also shows that PLS and CCA can be made equivalent by whitening the data matrices before constructing the covariance matrix. When the number of features exceeds the number of samples (\(p>n\)), CCA becomes degenerate because the within-view covariance matrices cannot be inverted—contrasting with PLS, which is always computable.

\subsubsection{Generalized Eigenvalue Problems}

We can also represent the system of equations in equation \ref{CCA:FOCs} as a matrix equation:

\begin{align}
    \begin{pmatrix}
        0                    & \Sigma\sps{12} \\
        \Sigma\sps{21} & 0
    \end{pmatrix}
    \begin{pmatrix}
        u\sps{1} \\
        u\sps{2}
    \end{pmatrix}
    =
    \lambda
    \begin{pmatrix}
        \Sigma\sps{11} & 0 \\
        0                    & \Sigma\sps{22}
    \end{pmatrix}
    \begin{pmatrix}
        u\sps{1} \\
        u\sps{2}
    \end{pmatrix}
\end{align}

Which is of the form $\mathbf{A v} = \lambda \mathbf{B v}$. CCA is therefore often referred to as a generalized eigenvalue problem for which there are a number of publicly available solvers.

\subsubsection{LDA as a Special Case of CCA}

Linear Discriminant Analysis (LDA) can be viewed as a special case of Canonical Correlation Analysis (CCA) where \(X^{(2)}\) is a one-hot encoded matrix representing the class labels.
This allows us to draw a connection between the unsupervised learning framework of CCA and the supervised framework of LDA, thus expanding the understanding of both algorithms.

\textbf{Intuition:} In LDA, the aim is to find a lower-dimensional subspace where the classes are maximally separated. This objective can be viewed through the lens of CCA, where the optimal directions \(u^{(1)}\) and \(u^{(2)}\) in the original and one-hot encoded spaces aim to maximize correlation. In the LDA context, \(u^{(1)}\) would maximize the separation between classes.

Mathematically, LDA is reduced to solving a generalized eigenvalue problem involving the between-class scatter matrix \(\mathbf{S}_B\) and the within-class scatter matrix \(\mathbf{S}_W\):

\[
    \hat{\mathbf{S}_B} = \sum_{i=1}^{c} n_i (\mu_i - \mu)(\mu_i - \mu)^T
\]

\[
    \hat{\mathbf{S}_W} = \sum_{i=1}^{c} \sum_{x \in X_i} (x - \mu_i)(x - \mu_i)^T
\]

\textbf{Connection to CCA:} When \(X^{(2)}\) is the one-hot encoded matrix of class labels, the CCA problem effectively tries to maximize the correlation between the feature vectors and their corresponding labels.
This turns out to be equivalent to maximizing the between-class variance in LDA while minimizing the within-class variance.
Thus, LDA can be thought of as a constrained form of CCA, tailored to classification tasks.

This perspective unifies the two algorithms and shows that the core objective—finding meaningful relationships or directions in the data—is shared between both CCA and LDA.

\textbf{Multi-view CCA} is a straightforward extension of CCA to the case of 3-or more datasets.
The optimization problem for MCCA can be stated as:
\begin{align}
     & u_{\text{opt}} = \underset{u}{\mathrm{argmax}} \sum_{i=1}^{m} \sum_{j=1, j \neq i}^{m} u\sps{i\top} X\sps{i\top} X\sps{j} u\sps{j} \\
     & \text{subject to:} \notag \\
     & \sum_{i=1}^{m} u\sps{i\top} X\sps{i\top} X\sps{i} u\sps{i} = 1 \notag
\end{align}

The generalized eigenvalue problem (GEP) can be written in matrix form as follows:

\begin{align}
    \mathbf{A} \mathbf{U} &= \lambda \mathbf{B} \mathbf{U} \\
    \mathbf{A} &= \begin{pmatrix}
        \mathbf{0} & \Sigma\sps{12} & \cdots & \Sigma\sps{1m} \\
        \Sigma\sps{21} & \mathbf{0} & \cdots & \Sigma\sps{2m} \\
        \vdots & \vdots & \ddots & \vdots \\
        \Sigma\sps{m1} & \Sigma\sps{m2} & \cdots & \mathbf{0}
    \end{pmatrix}, \\
    \mathbf{B} &= \begin{pmatrix}
        \Sigma\sps{11} & \mathbf{0} & \cdots & \mathbf{0} \\
        \mathbf{0} & \Sigma\sps{22} & \cdots & \mathbf{0} \\
        \vdots & \vdots & \ddots & \vdots \\
        \mathbf{0} & \mathbf{0} & \cdots & \Sigma\sps{mm}
    \end{pmatrix}, \\
    \mathbf{U} &= \begin{pmatrix}
        u\sps{1} \\
        u\sps{2} \\
        \vdots \\
        u\sps{m}
    \end{pmatrix}.
\end{align}

\subsection{Sample Covariance and Population Covariance}
In the previous sections, the methods were described in terms of population covariance matrices such as \(\Sigma\sps{11}=\mathbb{E}[X\sps{1}^T X\sps{1}]\), \(\Sigma\sps{22}=\mathbb{E}[X\sps{2}^T X\sps{2}]\), and \(\Sigma\sps{12}=\mathbb{E}[X\sps{1}^T X\sps{2}]\). These population covariances assume an underlying probability distribution from which the data are drawn.

\textbf{Sample Covariance:} In practical settings, we often do not have access to the entire population but only to a sample. Hence, we can utilize the Sample Average Approximation to estimate these covariances:

\[
    \hat{\Sigma}\sps{12} = \frac{1}{b-1} \bar{\mathbf{X}\sps{1}} \bar{\mathbf{X}\sps{2}}^T
\]

Here, \(b\) denotes the size of the minibatch, and \(\mathbf{X}\sps{1} \in \mathbb{R}^{p \times b}\) and \(\mathbf{X}\sps{2} \in \mathbb{R}^{q \times b}\) are the data matrices for the samples from \(X\sps{1}\) and \(X\sps{2}\), respectively. The bar over \(\mathbf{X}\sps{1}\) and \(\mathbf{X}\sps{2}\) signifies that these are centered versions of the matrices, i.e., the mean has been subtracted from each column.

\textbf{Practical Implications:} Using sample covariance matrices introduces some estimation error but allows us to apply the methods in real-world scenarios where population-level data are unattainable. Additionally, the use of minibatches provides a computationally efficient way to estimate these covariances in large-scale problems, at the cost of some additional statistical noise.

\textbf{Connection to Previous Methods:} The use of sample covariance matrices is directly applicable to algorithms like CCA and LDA. When replacing the population covariances \(\Sigma\sps{ij}\) with sample estimates, the optimization problems remain structurally similar but are solved using the sample data.

This dual perspective—considering both population and sample covariance matrices—enables a more robust and flexible approach to the methods discussed, bridging the gap between theoretical analysis and practical application.

\subsection{Multiple Effects, Orthogonality, and Deflation}\label{subsec:orthogonality}

Understanding multiple correlated or covarying effects is fundamental in multivariate data analysis methods such as Principal Component Analysis (PCA), Partial Least Squares (PLS), and Canonical Correlation Analysis (CCA). Although these methods generally aim to find a single maximal correlation or covariance (referred to as the top-1 problem), they are capable of identifying more than one orthogonal effect, a scenario often termed the top-$k$ problem, where $k$ denotes the number of orthogonal effects desired.

\subsection{Challenges in Identifying Multiple Components}

Identifying multiple components using iterative methods introduces specific challenges.
Firstly, to discover subsequent effects that are orthogonal to the maximally correlated or covarying one, a deflation step is commonly applied.
This is essential because the second component is conditional on the first in deflation models, meaning that interpreting the second component as an independent form of variation becomes untenable if the wrong first component is identified.

\subsection{Deflation Methods for Orthogonality}

Two prevalent deflation methods, Hotelling's Deflation and Projection Deflation, serve to ensure the orthogonality of the components.

\textbf{Hotelling's method}  is mainly used for methods based on the power method for the Singular Value
Decomposition (SVD).

\begin{align}
     & d = \mathbf{w^{\top}_1X^{\top}_1X_2w_2}                                            \\
     & \mathbf{\Sigma^{(i+1)}_{12}}= \mathbf{\Sigma^{(i)}_{12}} - d\mathbf{w_1w^{\top}_2}
\end{align}

\textbf{Projection Deflation} is common in algorithms like Wold's NIPALS and ensures that the projection vectors
themselves are orthogonal. These methods work by systematically reducing the data dimensions, so that each new component is orthogonal to the ones found before it. However, these methods come with their own limitations and are sensitive to the condition of the original problem, including issues related to regularization.

\begin{align}
     & P(\mathbf{X})= \frac{\mathbf{Xw}}{\|\mathbf{Xw}\|}\mathbf{w^{\top}X^{\top}X}
\end{align}

\begin{align}
     & P^\perp(\mathbf{X})= \mathbf{X} - \frac{\mathbf{Xw}}{\|\mathbf{Xw}\|}\mathbf{w^{\top}X^{\top}X} = (I - \frac{\mathbf{Xw}}{\|\mathbf{Xw}\|}\mathbf{w^{\top}X^{\top})X}
\end{align}

\subsection{Non-Uniqueness of Components}

Another critical consideration is the non-uniqueness of the identified components.
The eigenvectors of covariance matrices in these methods are only defined up to a sign, or up to an orthogonal transformation when eigenvalues are repeated. This lack of uniqueness can lead to inconsistencies, especially when comparing solutions across different samples or data folds.

\subsection{Thesis Perspective}

Given these complexities, this thesis adopts a subspace perspective, focusing on either the top-1 or top-$k$ solutions, considering these as the most reliable and interpretable representations of the data's underlying structure.
This approach aims to mitigate the limitations of deflation methods and the non-uniqueness of solutions, providing a more robust framework for multivariate data analysis.

\section{Open challenges in CCA}

\subsection{Efficient Algorithms for High-Dimensional Data}

The challenges of high-dimensional data often manifest when solving Generalized Eigenvalue Problems (GEPs) for Canonical Correlation Analysis (CCA). The computational burden of solving these problems becomes daunting as the number of features grows. To combat this issue, various efficient algorithms have been developed to reduce the complexity. In this section, we will explore some of these strategies.

\subsubsection{Challenges in Solving Generalized Eigenvalue Problems}

The GEP is often represented as\( Ax = \lambda Bx \), where \( A \)and \( B \)are matrices whose dimensions depend on the method in use.
For instance, in PCA, \( A \)and \( B \)are\( p \times p \); in PLS and CCA, they are\( (p+q) \times (p+q) \).

% Table summarizing the definitions of A and B for various methods
\begin{table}[h]
    \centering
    \begin{tabular}{|c|c|c|c|c|}
        \hline
        Method & \( A \)&\( B \)&\( X \)& Dimensions\\
        \hline
        PCA& \( \Sigma\sps{11} \)&\( \mathbf{I} \)&\( U\sps{1} \)&\( p \times p \) \\
        \hline
        LDA& \( \mathbf{S}_B \)&\( \mathbf{S}_W \)&\( U\sps{1} \)&\( p \times p \) \\
        \hline
        CCA& \( \begin{pmatrix} \Sigma\sps{11} & \Sigma\sps{12} \\ \Sigma\sps{21} & \Sigma\sps{22} \end{pmatrix} \)&\( \begin{pmatrix} \Sigma\sps{11} & \mathbf{0} \\ \mathbf{0} & \Sigma\sps{22} \end{pmatrix} \)&\( \begin{pmatrix} U\sps{1} \\ U\sps{2} \end{pmatrix} \)&\( (p+q) \times (p+q) \) \\
        \hline
        PLS& \( \begin{pmatrix} \mathbf{0} & \Sigma\sps{12} \\ \Sigma\sps{21} & \mathbf{0} \end{pmatrix} \)&\( \mathbf{I} \)&\( \begin{pmatrix} U\sps{1} \\ U\sps{2} \end{pmatrix} \)&\( (p+q) \times (p+q) \) \\
        \hline
    \end{tabular}
    \caption{Definitions and dimensions of \( A \)and \( B \)for different subspace learning methods.}
    \label{tab:subspace}
\end{table}

To solve the GEP, one common technique is to transform it into a standard eigenvalue problem\( B^{-\frac{1}{2}} A B^{-\frac{1}{2}} y = \lambda y \), followed by eigendecomposition.
However, this approach has computational complexity \(O(n^3)\)and may suffer from numerical instability.

\subsubsection{PCA-CCA}

The primary advantage of using PCA-CCA is computational efficiency, especially for high-dimensional data.
The overall complexity of PCA-CCA is\(O(p^2K + p^3)\), where \( K \)is the number of reduced components.
This method is particularly beneficial when \( p \)is large.

\subsubsection{KCCA}

Kernel Canonical Correlation Analysis (KCCA) extends CCA to capture nonlinear relationships. The optimization problem for KCCA is:

\begin{align}
    & \bold{\alpha_{opt}}=\underset{\bold{\alpha_{opt}}}{\mathrm{argmax}}\{ \bold{\alpha_1^{\top}K_1^{\top}K_2\alpha_2}  \}\\
    & \text{subject to:} \notag\\
    & \bold{\alpha_1^{\top}K_1^{\top}K_1\alpha_1}=1 \notag\\
    & \bold{\alpha_2^{\top}K_2^{\top}K_2\alpha_2}=1 \notag
\end{align}

KCCA is computationally efficient for high-dimensional data (\(p>n\)) because its complexity scales with the number of samples\(n\), not the number of features\(p\).
However, it requires access to all training data at test time, raising efficiency concerns.

\subsection{Regularisation for High-Dimensional and Structured Data}

Regularised solutions to the CCA problem are desirable both to provide a solution in the case where the number of features, \( p \) exceeds the number of observations, \( n \)as well as to improve the robustness of the projections in the case where we expect noisy observations \cite{branco2005robust} and/or to produce sparse solutions for better interpretability\cite{parkhomenko2009sparse}.

\subsubsection{Ridge regularisation}\label{subsec:ridge-regularisation}

Vinod proposed the `Canonical Ridge' which combined the PLS and CCA constraints in a single constrained optimisation \cite{vinod1976canonical}:

\begin{align}
     & u\sps{1}_{\text{opt}} = \underset{u\sps{1}}{\mathrm{argmax}} \{ u\sps{1}^\top X\sps{1}^\top X\sps{2} u\sps{2} \} \\
     & \text{subject to:} \notag \\
     & (1 - \tau_1) u\sps{1}^\top X\sps{1}^\top X\sps{1} u\sps{1} + \tau_1 u\sps{1}^\top u\sps{1} = 1 \notag \\
     & (1 - \tau_2) u\sps{2}^\top X\sps{2}^\top X\sps{2} u\sps{2} + \tau_2 u\sps{2}^\top u\sps{2} = 1 \notag
\end{align}

Where \( \tau_i \) is a mixing hyperparameter that makes the solution more or less CCA-like (\( c_i = 0 \)) or PLS-like (\( c_i = 1 \)) depending on the constraint.
By once again forming the Lagrangian and taking partial derivatives, we have the first-order conditions:

\begin{align}
    & X\sps{1}^\top X\sps{2} u\sps{2} + \lambda_1 ((1-\tau_1) X\sps{1}^\top X\sps{1} u\sps{1} + \tau_1 u\sps{1} - 1) = 0 \\
    & X\sps{2}^\top X\sps{1} u\sps{1} + \lambda_2 ((1-\tau_2) X\sps{2}^\top X\sps{2} u\sps{2} + \tau_2 u\sps{2} - 1) = 0
\end{align}

And this gives us the eigenvalue problems \cite{rosipal2005overview}:

\begin{align}
    & ((1-\tau_1) X\sps{1}^\top X\sps{1} + \tau_1 I)^{-1} X\sps{1}^\top X\sps{2} ((1-\tau_2) X\sps{2}^\top X\sps{2} + \tau_2 I)^{-1} X\sps{2}^\top X\sps{1} u\sps{1} = \lambda^2 u\sps{1} \notag \\
    & ((1-\tau_2) X\sps{2}^\top X\sps{2} + \tau_2 I)^{-1} X\sps{2}^\top X\sps{1} ((1-\tau_1) X\sps{1}^\top X\sps{1} + \tau_1 I)^{-1} X\sps{1}^\top X\sps{2} u\sps{2} = \lambda^2 u\sps{2}
\end{align}

The main difference between this eigenvalue problem and the CCA eigenvalue problem is the substitution of the matrices \(X\sps{1}^\top X\sps{1}\) and \(X\sps{2}^\top X\sps{2}\) for the matrices \( ((1-\tau_1) X\sps{1}^\top X\sps{1} + \tau_1 I) \) and \( ((1-\tau_2) X\sps{2}^\top X\sps{2} + \tau_2 I) \).
We can therefore see that this regularisation is equivalent to adding a constant to the diagonal of the covariance matrix \(X\sps{i}^\top X\sps{i}\).

\subsubsection{Sparse CCA}

Sparse CCA methods aim to find sparse vectors \(u\sps{1}\) and \(v\sps{2}\) that optimize the correlation while maintaining interpretability.

\textbf{Sparse PLS: Penalized Matrix Decomposition}Witten's Penalized Matrix Decomposition (PMD) \cite{
    witten2009penalized} provides an approximate solution to the sparse CCA problem by altering the constraints of the classical CCA formulation.
Specifically, PMD replaces the constraint \(\bold{u\sps{1}\top X\sps{1}\top X\sps{1} u\sps{1}}=1\) with \(\bold{u\sps{1}\top u\sps{1}}=1\).
The optimization problem for PMD is then given by:

\begin{align}
    \label{eq:pmd}
    & \bold{u\sps{opt}}=\underset{\bold{u}}{\mathrm{argmax}}\{ \bold{u\sps{1}\top X\sps{1}\top X\sps{2} v\sps{2}} \} \\
    & \text{subject to:} \notag \\
    & \bold{u\sps{1}\top u\sps{1}} \leq 1 \notag \\
    & \bold{v\sps{2}\top v\sps{2}} \leq 1 \notag \\
    & P(\bold{u\sps{1}}) \leq c_1 \notag \\
    & P(\bold{v\sps{2}}) \leq c_2 \notag
\end{align}

\textbf{Penalized CCA} Parkhomenko et al.\ \cite{parkhomenko2009sparse} also proposed a sparse CCA variant that
operates directly on correlation matrices.
Their optimization problem is formulated as:

\begin{align}
    \label{eq:parkho}
    & \bold{u\sps{opt}}=\underset{\bold{u}}{\mathrm{argmax}}\{ \bold{(X\sps{1}X\sps{1}\top)^{-\frac{1}{2}}X\sps{1}\top X\sps{2}(X\sps{2}\top X\sps{2})^{-\frac{1}{2}}} \}
\end{align}

\textbf{Iterative Penalized Least Squares} Some approaches to sparse Canonical Correlation Analysis (CCA) are related to alternating regression forms, such as sparse alternating regressions~\cite{wilms2015sparse}.
Mai et al.~\cite{mai2019iterative} proved that these methods can yield true global solutions to projection-length constrained lasso regression problems.
However, this holds true only for one-homogeneous penalties, meaning the regularization function must take the form \( P((\mu + 1)\boldsymbol{\theta}) = (\mu + 1)P(\boldsymbol{\theta}) \). Consequently, we need \( P(\boldsymbol{\theta}) \) to be a one-homogeneous function to use regularized alternating least squares for solving regularized CCA objectives.

\begin{align}
    \label{eq:mai}
    \boldsymbol{u} &= \underset{\boldsymbol{u}}{\mathrm{argmin}} \left\{ \|\boldsymbol{Xu} - \boldsymbol{y}\|_2^2 + P(\boldsymbol{u}) \right\} \\
    \text{subject to:} \notag \\
    \boldsymbol{u^{(1)\top} X^{(1)\top} X^{(1)} u^{(1)}} &= 1 \notag
\end{align}

However, the one homogenous penalty in practice limits the flexibility of the method.
For example, the elastic net penalty is not one-homogenous and therefore cannot be used with this method.

\textbf{Sparse Diagonal CCA}\cite{asteris2016simple} propose a combinatorial algorithm for sparse diagonal CCA. Their algorithm operates on a low-rank approximation of the input data and scales linearly with the number of input variables.
The algorithm precisely controls the sparsity of the extracted canonical vectors and offers data-dependent global approximation guarantees based on the spectrum of the input data.

Formally, they optimize the following problem:

\begin{aligned}
    \label{eq:asteris}
    \max_{u\sps{1} \in U\sps{1}, u\sps{2} \in U\sps{2}} \quad & (u\sps{1})^{\top}\Sigma_{X\sps{1}X\sps{2}}u\sps{2} \\
    \text{where } & U\sps{1} = \left\{ u\sps{1} \in \mathbb{R}^{m} : ||u\sps{1}||_2 = 1, ||u\sps{1}||_0 \leq s_x \right\}, \\
    & U\sps{2} = \left\{ u\sps{2} \in \mathbb{R}^{n} : ||u\sps{2}||_2 = 1, ||u\sps{2}||_0 \leq s_y \right\},
\end{aligned}

Their approach balances between exhaustive search and thresholding methods by focusing on a principal subspace of the input matrix $\Sigma_{X\sps{1}X\sps{2}}$ spanned by its leading \( r \geq 1 \) singular vector pairs.
This allows them to find an (approximately) optimal pair of supports without evaluating the entire set of possible supports, thus making it both efficient and parallelizable.

\subsubsection{Critical Commentary on Existing Methods and Motivation for the Present Work}

Existing methods in regularized Canonical Correlation Analysis (CCA) have provided valuable insights, but there is still much room for improvement, especially when handling high-dimensional data—a pressing concern that has garnered significant attention in contemporary research.

Ridge CCA is a stabilizing force in the solution space but falls short in providing sparse solutions.
This makes the interpretation of the canonical vectors challenging, a shortcoming that is especially glaring in high-dimensional settings where interpretability is crucial.
Additionally, the method demands a careful selection of the regularization parameter, a process that is not always straightforward.

PCA-CCA simplifies computational demands by first reducing dimensionality using PCA. However, this risks overlooking subtle but crucial relationships between variables, as it focuses on leading principal components.
This method, too, faces the challenge of interpretability, a recurring theme in PCA-based approaches.

Witten's PMD, despite its influence, effectively performs Sparse PLS, diverging from its initial framing as sparse CCA due to its use of an identity covariance assumption.
Similarly, Penalized CCA \cite{parkhomenko2009sparse} hasn't gained substantial traction, possibly because its penalty terms make sparsity control cumbersome.
This issue also manifests in the Iterative Penalized Least Squares (IPLS) approach, which adds another layer of complexity: the selection of appropriate regularization parameters.
Moreover, the computational expense of solving high-dimensional IPLS problems is considerable, particularly when iterative steps are required.

Lastly, the sparse diagonal CCA proposed by \cite{asteris2016simple} is limited in its ability to capture complex relationships between variables due to its assumption of diagonal covariance matrices.

These challenges collectively point to an urgent need for CCA models that are not only more interpretable and flexible but also computationally efficient for high-dimensional data.
This need strongly motivates the contributions of the present thesis.


\section{Multiview Learning in Neuroimaging}

There have been a number of applications of CCA and related methods to multiview problems in neuroimaging.
Using resting state fMRI data, modes of correlation have been found that relate to differences in sex and age relating to drug and alcohol abuse, depression and self harm \cite{mihalik2019brain}.
A similar mode relating to `positive-negative' wellbeing has been found across studies \cite{smith2015positive}suggesting that mental wellbeing has a relationship (though not necessarily causally) with functional connectivity between networks in the brain.
Later in this dissertation we will replicate and build on the findings from this paper by using regularised and non-linear CCA methods.

CCA has also been used as a preprocessing step in order to identify groups of subjects in the latent variable space.
In particular, CCA and clustering have been used to identify depression using fMRI data\cite{dinga2019evaluating} \cite{drysdale2017resting}.
CCA has also been used in the manner we described to denoise two views of a dataset such as separate measures of neuroimaging data \cite{zhuang2020technical} to remove artefacts.
Deep CCA has recently been used to extract features for the diagnosis of schizophrenia\cite{qi2016deep}.





